% based on a template made by the university of cologne
% http://www.mi.uni-koeln.de/wp-MIEDV/wp-content/uploads/2016/07/LaTeX-Vorlage.zip - 2023-11-02
\documentclass[12pt,a4paper]{scrartcl}

\addtokomafont{sectioning}{\rmfamily}
\usepackage[ngerman]{babel}% deutsches Sprachpaket wird geladen
\usepackage[T1]{fontenc} % westeuropäische Codierung wird verlangt
\usepackage[utf8]{inputenc}% Umlaute werden erlaubt
\usepackage[usenames]{color} % Erlaubt die Benutzung der namen im Farbpaket und deren Änderung
\usepackage{amsmath} % Erweiterung für den Mathe-Satz
\usepackage{amssymb} % alle Zeichen aus msam und msmb werden dargestellt
\usepackage{graphicx} % Graphiken und Bilder können eingebunden werden
%\usepackage{multirow} % erlaubt in einer Spalte einer Tabelle die Felder in mehreren Zeilen zusammenzufassen
\usepackage{enumerate} % erlaubt Nummerierungen
\usepackage{xurl} % Dient zur Auszeichnung von URLs; setzt die Adresse in Schreibmaschinenschrift.
\usepackage[center]{caption}  % Bildunterschrift wird zentriert
%\usepackage{subfigure} % mehrere Bilder können in einer fugure-Umgebung verwendet werden
%\usepackage{longtable} % Diese Umgebung ist ähnlich definiert wie die tabular-Umgebung, erlaubt jedoch mehrseitige Tabellen.
%\usepackage{paralist} % Modifikation der bereits bestehenden Listenumgebungen
\usepackage{lmodern}% Für die Schrift
\usepackage[hidelinks]{hyperref} % Links und Verweise werden innerhalb von PDF Dokumenten erzeugt
%\usepackage{wrapfig} % Das Paket ermöglicht es von Schrift umflossene Bilder und Tabellen einzufügen.
\usepackage{latexsym} % LaTeX-Symbole werden geladen
\usepackage{tikz} % Erlaubt es mit tikz zu zeichnen
\usepackage{tabularx} % Erlaubt Tabellen
\usepackage{algorithm} % Erlaubt Pseudocode
\usepackage{color} % Farbpaket wird geladen
%\usepackage{stmaryrd} % St Mary Road Symbole werden geladen
\usepackage{physics}
\usepackage{mhchem} % Chemie: \ce & \pu

\numberwithin{equation}{section} % Nummerierungen der Gleichungen, die durch equation erstellt werden, sind gebunden an die section
\newcommand{\HRule}{\rule{\linewidth}{0.7mm}}
\newcommand{\pu}[1]{\ensuremath{\mathrm{#1}}}

% disable commands
\renewcommand{\[}{} % math block start
\renewcommand{\]}{\noindent} % math block end
\newcommand{\tightlist}{} % created in enumerations

\hypersetup{
  pdftitle={B 3.3},
  pdfcreator={LaTeX via pandoc}}
  
  \hyphenation{Ener-gie-stragg-ling}

\setcounter{secnumdepth}{6}
\setcounter{tocdepth}{6}

\begin{document}
\begin{titlepage}
	\pagestyle{empty}

	\begin{center}

	\textsc{\LARGE Universität zu Köln }\\ [0.4cm]
	\textsc{Mathematisch-Naturwissenschaftliche Fakultät} \\[1.5cm]

	\includegraphics[width=0.45\textwidth]{../media/uni.jpg}\\[1.5cm]  % Uni-Logo wird geladen

	\textsc{\Large Praktikum~B}\\[2mm]
	\textsc{}\\[10mm]
	\HRule \\[0.4cm]

		{	\Huge \bfseries B 3.3}\\[0.4cm]
			{	\huge \bfseries Reichweite von \(\pmb{\alpha}\)-Strahlen}\\[0.3cm]
	
	\HRule \\[3cm]

			\textsc{\Large Catherine Tran } \\[3pt]
		\textsc{\Large Carlo Kleefisch } \\[3pt]
		\textsc{\Large Oliver Filla } \\[3pt]
		
% 	\begin{center}
% 	\textsc{\Large Catherine~Tran } \\[3pt]
% 	\textsc{\Large Carlo~Kleefisch } \\[3pt]
% 	\textsc{\Large Oliver~Filla } \\[3pt]
% 	\end{center}
	\end{center}
\end{titlepage}

\newpage
\tableofcontents
\newpage

\hypertarget{einleitung}{%
\section{Einleitung}\label{einleitung}}

In diesem Versuch wird die Wechselwirkung von \(\alpha\)-Teilchen mit
den Elektronen der Atomhülle und der damit verbundene Abbremsung durch
inelastische Stöße untersucht. Des Weiteren werden das Phänomen des
\(\alpha\)-Zerfalls, Bremsvermögen, Reichweite in Luft und Folie sowie
Energie-Straggling durch die Aufnahme von \(\alpha\)-Spektren mithilfe
eines Sperrschichtdetektors studiert.

\hypertarget{theoretische-grundlagen}{%
\section{Theoretische Grundlagen}\label{theoretische-grundlagen}}

\hypertarget{alpha-zerfall}{%
\subsection{\texorpdfstring{\(\alpha\)-Zerfall}{\textbackslash alpha-Zerfall}}\label{alpha-zerfall}}

Der \(\alpha\)-Zerfall ist eine Form des radioaktiven Zerfalls, bei dem
ein Heliumkern \(\ce{^4_2He}\) emittiert wird. Nach der Nuklidkarte
findet der Zerfall hauptsächlich bei massereichen Kernen. \([3]\)

\[
\begin{eqnarray}
    \ce{^A_ZX -> ^{A-4}_{Z-2}Y + ^4_2He}
\end{eqnarray}
\]

\hypertarget{q-wert}{%
\subsubsection{\texorpdfstring{\(Q\)-Wert}{Q-Wert}}\label{q-wert}}

Der Energiedifferenz zwischen Ausgangs- und Endprodukt ist durch den
\(Q\)-Wert gegeben, der die Massendifferenz zwischen Mutterkern
\(\ce{^A_ZX}\), Tochterkern \(\ce{^{A-4}_{Z-2}Y}\) sowie Heliumkern
\(\ce{^4_2He}\) darstellt. Diese wird durch die Masse-Energie-Relation
\(E=mc^2\) mittels der Lichtgeschwindigkeit \(c\) in eine Energie
umgerechnet.

\[
\begin{eqnarray}
    Q &=&
        \left(m_{A}
            \left(\ce{^A_ZX}\right)
            - m_{A}\left(\ce{^{A-4}_{Z-2}Y}\right)
            - m_{A}\left(\ce{^4_2He}\right)
        \right) \cdot c^2
\end{eqnarray}
\]

\hypertarget{energie-der-alpha-teilchen}{%
\subsubsection{\texorpdfstring{Energie der
\(\alpha\)-Teilchen}{Energie der \textbackslash alpha-Teilchen}}\label{energie-der-alpha-teilchen}}

Mithilfe der die Atommassen \(m_{A}\) der Stoffe kann über die
Impulserhaltung die kinetische Energie \(E_\alpha\) des
\(\alpha\)-Teilchens ermittelt werden.

\[
\begin{eqnarray}
    E_\alpha
        &=& \frac{m_{A}\left(\ce{^{A-4}_{Z-2}Y}\right) \cdot Q}
            {m_{A}\left(\ce{^{A-4}_{Z-2}Y}\right)+m_{A}\left(\ce{^4_2He}\right)}
\end{eqnarray}
\]

\hypertarget{kernpotential}{%
\subsubsection{Kernpotential}\label{kernpotential}}

Das Kernpotential beschreibt die potentielle Energie innerhalb eines
Atomkerns, die die Nukleonen zusammenhält. Es beruht auf der starken
Wechselwirkung sowie der Coulombwechselwirkung innerhalb des Kernes.

Um den Kern herum bewirkt die elektromagnetische Wechselwirkung eine
Abstoßung zwischen einem positiv geladenen Teilchen und dem ebenso
geladenen Kern. Beide Potentiale wirken zusammen und bilden ein
quasibindenes Potenzial mit einer endlichen Coulombbarriere.

Positronen und Neutronen sind in schweren Kerne mit einer Energie bis zu
\(7\mathrm{\,MeV}\) gebunden und können daher nicht einzeln den Kern
verlassen. Deshalb ist eine Emission eines gebundene System
wahrscheinlicher, da zusätzliche Bindungsenergie zur Verfügung steht.
Die Bildung eines \(\alpha\) -Teilchens mit einer Bindungsenergie von
etwa \(7.1\mathrm{\,MeV}\) ermöglicht das Verlassen des Kerns durch die
Coulombbarriere \(V_C\).

Dennoch ist die Energie des \(\alpha\)-Teilchens nicht groß genug, um
die Potentialbarriere zu überwinden. Deswegen muss es hindurch
\emph{tunneln}. Dieser Prozess wird im Folgenden beschrieben.

\hypertarget{tunneleffekt}{%
\subsubsection{Tunneleffekt}\label{tunneleffekt}}

Wie schon erwähnt muss das Teilchen die energetisch höhere
Coulomb-Barriere überwinden. Dies wird durch die
Tunnelwahrscheinlichkeit \(T\) bestimmt, die von dem Gamow-Faktor \(G\)
abhängt. Dieser wiederum hängt von dem Coulomb-Potential \(V_C\), der
Energie der \(\alpha\)-Teilchen \(E_\alpha\) sowie deren Masse
\(m_\alpha\) und Position der Barriere von \(r_1\) bis \(r_2\) ab.
\([7]\)

\[
\begin{eqnarray}
    T &=& T_0 \cdot e^{-G} \\
    G &=&
        \frac{2\sqrt{2m_\alpha}}{\hbar}
        \int_{r_{1}}^{r_{2}}\sqrt{V_{C}-E_{\alpha}}
        \,\mathrm dr
\end{eqnarray}
\]

\hypertarget{zerfallswahrscheinlichkeit}{%
\subsubsection{Zerfallswahrscheinlichkeit}\label{zerfallswahrscheinlichkeit}}

Damit ein Zerfall stattfindet, müssen drei Ereignisse in Folge
stattfinden, die jeweils mit unterschiedlichen Wahrscheinlichkeiten
geschehen.

Zuerst muss sich ein \(\alpha\)-Teilchen im Kern bilden. Dann muss
Teilchen am Rand des Kerns gegen die Coulomb-Barriere stoßen, die den
Kern zusammenhält. Dann muss das \(\alpha\)-Teilchen durch die Barriere
tunneln. Die Wahrscheinlichkeiten für diese drei Prozesse multiplizieren
sich zu der gesamten Zerfallswahrscheinlichkeit für den Kern.

\hypertarget{energiespektrum}{%
\subsubsection{Energiespektrum}\label{energiespektrum}}

Zerfällt ein Mutterkern, so können außer dem Grundzustand noch andere
angeregte Zustände des Tochterkerns besetzt werden. Man erhält ein
diskretes Linienspektrum. Bei einer Messung wird jede Linie mit einer
gewissen Wahrscheinlichkeit gemessen, dabei wird jede Linie durch eine
Gaußkurve angenähert.

Das Spektrum des Isotops \(\ce{^241Am}\) hat vier Linien bei Energien
von \(5388\mathrm{\,keV}\) und \(5545\mathrm{\,keV}\). \([1]\) Es ist in
Abbildung \(1\) dargestellt.

\begin{figure}
	\centering
	\includegraphics[width=0.5\textwidth]{../media/B3.3/Am241_Spektrum.pdf}
	\caption{gemessenes Spektrum von \(\ce{^241Am}\) \([1]\)}
	\label{abb:Spektrum 241Am}
\end{figure}

\hypertarget{weizsuxe4cker-massenformel}{%
\subsubsection{Weizsäcker
Massenformel}\label{weizsuxe4cker-massenformel}}

Die \emph{Weizsäcker Formel} gibt die Bindungsenergie eines Atomkerns
an. Sie basiert einerseits auf empirischen Daten, andererseits auf dem
Tröpfchenmodell. Aus ihr wird die Weizsäcker Massenformel ermittelt.

Das Tröpfchenmodell beschreibt einen Atomkern als einen inkompressiblen,
kugelförmigen Fluidtropfen, der zur Energieminimierung den Atomradius
\(R\) aufweist. Dabei ist die Dichte überall konstant.

Die Bindungsenergie \(E_B\) wird aus fünf verschiedenen Termen
ermittelt, auf die im Folgenden eingegangen wird. Diese Terme werden
durch die empirisch ermittelten Faktoren \(a_i\) sowie die Nukleonenzahl
/ Massenzahl \(A\), Protonenzahl \(Z\) und Neutronenzahl \(N\)
beschrieben. Hierbei handelt es sich um den Volumenterm \(E_V\)
\(\eqref{Volumenterm}\), den Oberflächenterm \(E_O\)
\(\eqref{Oberflächenterm}\), den Coulombterm \(E_C\)
\(\eqref{Coulombterm}\), den Symmetrieterm \(E_S\)
\(\eqref{Symmetrieterm}\) und den Paarungsterm \(E_P\)
\(\eqref{Paarungsterm}\).

\[
\begin{eqnarray}
    E_B &=& E_V + E_O + E_C + E_S + E_P \\
\end{eqnarray}
\]

Die Bindungsenergie \(E_B\) verringert die Masse des Atomkernes. Daher
kann die Kernmasse \(m\) durch die Protonenmasse \(m_P\), die
Neutronenmasse \(n_N\) sowie \(E_B\) beschrieben werden, wobei die
Lichtgeschwindigkeit \(c\) die Energie in Masse konvertiert. Dies ergibt
die \emph{Weizsäcker Massenformel}.

\[
\begin{eqnarray}
    m &=& N\cdot m_N + P\cdot m_P - \frac{E_B}{c^2}
\end{eqnarray}
\]

In der Darstellung der einzelnen Terme ist weiterhin relevant, dass der
Atomradius \(R\) kann durch den Radius \(r\) eines Nukleons und die
Nukleonenzahl \(A\) beschrieben werden kann.

\[
\begin{eqnarray}
    R &=& r \cdot \sqrt[3]{A}
\end{eqnarray}
\]

\hypertarget{volumenterm}{%
\paragraph{Volumenterm}\label{volumenterm}}

Der Volumenterm \(E_V\) beschreibt die Anziehung der Nukleonen durch die
starke Wechselwirkung.

Diese hat eine Reichweite von \(2.5\mathrm{\,fm}\), weswegen sie nur auf
die nächsten Nachbarn eines Nukleons wirkt. Da die Dichte im Kern nach
dem Tröpfchenmodell konstant ist, ist die gesamte Bindungsenergie durch
die starke Wechselwirkung proportional zum Kernvolumen. Dieses wiederum
ist proportional zu \(R^3\propto A\).

\[
\begin{eqnarray}
    E_V &=& + a_V\cdot A \label{Volumenterm} \\
    a_V &=& 15.85\mathrm{\,MeV}
\end{eqnarray}
\]

\hypertarget{oberfluxe4chenterm}{%
\paragraph{Oberflächenterm}\label{oberfluxe4chenterm}}

Da die Atome an der Oberfläche des Atomkerns weniger Nachbarn haben als
die Nukleonen im Kern, wird sind die ersteren schwächer gebunden. Daher
beschreibt der Oberflächenterm \(E_O\) eine Korrektur des Volumenterms.
Diese ist proportional zur Oberfläche einer Kugel mit dem Atomradius
\(R\), also proportional zu \(\sqrt[3]{A^2}\).

\[
\begin{eqnarray}
    E_O &=& - a_O\cdot \sqrt[3]{A^2} \label{Oberflächenterm} \\
    a_V &=& 18.34\mathrm{\,MeV}
\end{eqnarray}
\]

\hypertarget{coulombterm}{%
\paragraph{Coulombterm}\label{coulombterm}}

Der Coulombterm \(E_C\) beschreibt die elektrostatische Abstoßung der
Protonen voneinander, die die Bindungsenergie senkt. Jedes der \(Z\)
Protonen wird von den anderen \((Z-1)\) Protonen abgestoßen. Die
Coulombwechselwirkung ist proportional zu
\(R^{-1}\propto\left(\sqrt[3]{A}\right)^{-1}\).

\[
\begin{eqnarray}
    E_C &=& - a_C\cdot \frac{Z(Z-1)}{\sqrt[3]{A}} \label{Coulombterm} \\
    a_C &=& 0.71\mathrm{\,MeV}
\end{eqnarray}
\]

Für große Kerne mit \(Z\approx(Z-1)\) kann der Term
\(Z(Z-1)\approx Z^2\) vereinfacht werden.

\hypertarget{symmetrieterm}{%
\paragraph{Symmetrieterm}\label{symmetrieterm}}

Der Symmetrieterm \(E_S\) beschreibt die Verringerung der
Bindungsenergie durch ein Ungleichgewicht von Protonen und Neutronen.

Die Ursache kann quantenmechanisch erklärt werden. Protonen und
Neutronen werden als Fermigas in einem Potentialtopf betrachtet. Beide
Gase teilen sich denselben Potentialtopf und füllen Einteilchenniveaus
bis zu ihrer jeweiligen Fermienergie auf. Sind genau gleich viele beider
Teilchensorten vorhanden, so alle Zustände bis zur Fermienergie besetzt.

Gibt es jedoch ein Teilchen mehr von einer Sorte, so müssen höherer
Energieniveaus besetzt werden. Sei z.B. ein Proton mehr vorhanden, so
muss ein Proton ein höheres Energieniveau als alle anderen Nukleonen
besetzen. Dies benötigt mehr Energie.

Wandelt man nun in einem symmetrischen Kern ein Nukleon um, so erhöht
man die eine Fermienergie und senkt die andere ab. Dieser Prozess kostet
Energie, der Betrag der Energie ist die Differenz zwischen den
Ferminiveaus. Wenn man die Energiedifferenz in einer Tabelle aufträgt,
sieht man, dass der Term erst am Anfang mit \((N-Z)\) wächst und bei
Umschichtungen von drei Nukleonen eine besser Beschreibung das Wachstum
mit \((N-Z)/2\) ist. Wenn man dann noch betrachtet, dass Abstand der
Einteilchenniveaus mit steigendem Volumen sinkt, erhält man mit der
Proportionalität zwischen Volumen und Nukleonenzahl \(A\) folgende
Formel.

\[
\begin{eqnarray}
    E_S &=& - a_S\cdot \frac{(N-Z)^2}{4A} \label{Symmetrieterm} \\
    a_S &=& 2.86\mathrm{\,MeV}
\end{eqnarray}
\]

\hypertarget{paarungsterm}{%
\paragraph{Paarungsterm}\label{paarungsterm}}

Der Paarungsterm \(E_P\) beschreibt das Phänomen, dass gerade Anzahlen
von Protonen bzw. Neutronen in einem Kern stabilere Kerne produzieren.
Paare von Protonen oder Neutronen sind stärker gebunden als ein
ungepaartes Proton oder Neutron.

Deswegen wird zwischen \(\mathrm{gerade}\)-\(\mathrm{gerade}\)-Kernen
\((\mathrm{gg})\), \(\mathrm{gerade}\)-\(\mathrm{ungerade}\)-Kernen
\((\mathrm{gu})\) und \(\mathrm{ungerade}\)-\(\mathrm{gerade}\)-Kernen
\((\mathrm{ug})\) sowie
\(\mathrm{ungerade}\)-\(\mathrm{ungerade}\)-Kernen \((\mathrm{uu})\)
unterschieden. Erstere haben jeweils eine gerade Anzahl von Protonen und
Neutronen, während letztere jeweils ungerade Anzahlen haben.
\(\mathrm{ug}\)- und \(\mathrm{gu}\)-Kerne haben eine Nukleonensorte in
gerader und die andere in ungerader Menge.

Bei einer geraden Anzahl derselben Nukleonensorte heben sich die Spins
auf, bei einer ungeraden Anzahl nicht. Auf diese Weise kann das Phänomen
mithilfe des Schalenmodells erklärt werden.

Der Paarungsterm wird betragsmäßig kleiner, je größer die Nukleonenzahl
\(A\) ist. DIes wird durch die folgende Gleichung beschrieben.

\[
\begin{eqnarray}
    E_P &=&
        \begin{cases}
            + a_P\cdot \frac{1}{\sqrt{A}} & \text{gg} \\
            0 & \text{gu} \\
            0 & \text{ug} \\
            - a_P\cdot \frac{1}{\sqrt{A}} & \text{uu} \\
        \end{cases}
        \label{Paarungsterm} \\
    a_P &=& 11.46\mathrm{\,MeV}
\end{eqnarray}
\]

Beide Nukleonensorten liefern betragsmäßig den gleichen Beitrag zu
\(E_P\). Bei \(\mathrm{gg}\)- und \(\mathrm{uu}\)-Kernen addieren sich
diese Werte zu einer nicht-verschwindenden Energie. Bei \(\mathrm{gu}\)-
und \(\mathrm{ug}\)-Kernen heben sich die Terme dagegen auf, weswegen
der Paarungsterm hier verschwindet.

\hypertarget{bremsvermuxf6gen}{%
\subsection{Bremsvermögen}\label{bremsvermuxf6gen}}

\hypertarget{bethe-bloch-gleichung}{%
\subsubsection{Bethe-Bloch-Gleichung}\label{bethe-bloch-gleichung}}

Bewegte und geladene Teilchen werden durch Interaktion mit Materie
abgebremst, indem sie durch Stöße mit Atomkernen sowie Elektronen
wechselwirken. Schwere Teilchen mit einer Ruhemasse \(M_0\gg m_e\)
deutlich größer der Elektronen-Ruhemasse \(m_e\) werden primär durch die
Wechselwirkung mit Atomkernen gebremst, wodurch die Atome angeregt und
ionisiert werden können.

Die Bethe-Bloch-Gleichung beschreibt den Verlust von Energie \(E\) pro
Strecke \(x\) durch das Durchfliegen eines homogenen Bremsmediums.

Dazu werden die Dichte \(\rho\), die Atommassenzahl \(A\) und die
Ladungszahl \(Z\) des Bremsmediums benötigt. Dabei wird von einem
homogenen Medium mit \(N\) Atomen pro Kubikzentimeter und der
Kernladungszahl \(Z\cdot e\) ausgegangen, wobei \(e\) die
Elementarladung darstellt. \(\beta\) ist der Quotient aus
Geschwindigkeit \(v\) und Lichtgeschwindigkeit \(c\), der auch in der
Relativitätstheorie verwendet wird.

\[
\begin{eqnarray}
    N &=& \frac{\rho\cdot N_A}{A} \\
    \beta &=& \frac{v}{c}
\end{eqnarray}
\]

Ebenso werden die Ladungzahl \(z\) und Geschwindigkeit \(v\) des
Projektils sowie die Elektronen-Ruhemasse \(m_e\) verwendet. Weiterhin
sind das mittlere Ionisationspotential \(\bar I\), gemittelt über alle
Atomschalen des Bremsmediums, sowie eine Korrektur \(c_K\) notwendig.
Letztere beschreibt den fehlenden Beitrag der \(K\)-Schalen-Elektronen
bei kleinen Geschossenergien.

\[
\begin{eqnarray}
    -\frac{\mathrm dE}{\mathrm dx} &=&
        \frac{4\pi z^2 e^4}{m_e v^2} NZ
        \left[
            \ln\left(\frac{2mv^2}{\bar I}\right)
            - \ln\left(1 - \beta^2\right)
            - \beta^2
            - \frac{c_K}{Z}
        \right]
        \label{BetheBloch}
\end{eqnarray}
\]

\hypertarget{herleitung}{%
\paragraph{Herleitung}\label{herleitung}}

Im Folgenden werde die Bethe-Bloch-Gleichung für schwere, schnelle und
geladene Projektile wie \(\alpha\)-Teilchen hergeleitet.

Hierbei wird eine quasi-klassische Betrachtung des Stoßvorganges
angenommen. Da das Projektil sehr schwer im Vergleich zu Elektronen ist,
kann seine Bewegung als näherungsweise linear angenommen werden.
Weiterhin wird das Elektron als schwach gebunden und ruhend angenommen.
Diese Annahmen können durch die hohe Geschwindigkeit und Masse des
Projektils getätigt werden.

Da das Projektil das Elektron passiert, heben sich sämtliche
Wechselwirkungen parallel zur Flugbahn auf. Dadurch muss nur die
orthogonale Komponente der Coulomb-Kraft \(\vec F\) betrachtet werden,
die durch die Ladungen des Projektils \(Q=ze\) und des Elektrons
\(q=-e\) im Abstand \(\vec r\) erzeugt wird. Der Betrag des Abstands
kann durch die Wegstrecke \(x\) des Projektils sowie den orthogonalen
Abstand \(b\) der Flugbahn und des Elektrons als \(r^2=x^2+b^2\)
beschrieben werden.

\[
\begin{eqnarray}
    \vec F &=& \frac{Qq}{r^2} \frac{\vec{r}}{\left|\vec r\right|} \\
    \vec F &=& -\frac{ze^2}{x^2+b^2} \frac{\vec{r}}{\left|\vec r\right|}
\end{eqnarray}
\]

Weiterhin kann die Kraft durch das elektrische Feld \(\vec E\) des
Projektils und die Ladung des Elektrons \(q=-e\) beschrieben werden
\(\eqref{F=E}\). Diese Gleichung wird integriert, um den Betrag des
Impulsübertrages \(\left|\Delta p_e\right|\) zu ermitteln. Dabei wird
die Integration nach der Zeit durch eine Integration nach dem Ort
substituiert, was durch die konstante Geschwindigkeit \(v\) ermöglicht
wird. Weiterhin wird die Symmetrie ausgenutzt, wodurch nur noch über die
orthogonale Komponente integriert werden muss.

\[
\begin{eqnarray}
    \vec F &=& -e \vec E \label{F=E} \\
    \left|\Delta p_e\right| &=& \int \vec F \mathrm dt \\
    \left|\Delta p_e\right| &=& \frac{e}{v} \int E_\perp \mathrm dx
\end{eqnarray}
\]

Darauf wird der Gauß'sche Integralsatz angewendet. Weiterhin wird der
Energieübertrag \(\Delta E\) durch die kinetische Energie
\(E=\frac{p^2}{2m_e}\) des Elektrons dargestellt. Dann kann über einen
hohlen Zylinder vom Radius \(b_\mathrm{min}\) bis \(b_\mathrm{max}\)
integriert werden. Sinnvolle Integrationsgrenzen sind notwendig, da das
Integral sowohl bei \(x=0\) als auch bei \(x=\infty\) divergieren würde.

\[
\begin{eqnarray}
    -\left(\frac{\mathrm dE}{\mathrm dx}\right)
        &=& \frac{4\pi z^2 e^4}{m_ev^2}
            \ln\left[\frac{b_\mathrm{max}}{b_\mathrm{min}}\right]
            \propto \frac{z^2}{v^2}
\end{eqnarray}
\]

Nun werden relativistische Korrekturen durchgeführt, die zu der
vollständigen Bethe-Bloch-Gleichung \(\eqref{BetheBloch}\) führen.

\hypertarget{diskussion-des-kurvenverlaufs}{%
\subsubsection{Diskussion des
Kurvenverlaufs}\label{diskussion-des-kurvenverlaufs}}

Bei niedrigen Energien steigt die Kurve beinahe linear an. Dies ist
darauf zurückzuführen, dass ein langsames \(\alpha\)-Teilchen aufgrund
der langen Wirkzeit beim Durchqueren des Mediums zufällig Elektronen
aufnimmt und abgibt. Dies wiederum reduziert die die effektive Ladung
des \(\alpha\)-Teilchens und somit den Energieverlust.

Für \(\alpha\)-Teilchen findet sich bei kinetischen Energien von etwa
\(0.5-0.6\mathrm{\,MeV}\) ein Peak. Bei der Verbreiterung des Peaks der
Verteilung sind nicht-statistische Effekte von höherer Relevanz, als das
statistische Energie-Straggling.

Nach dem Peak sinkt die Kurve erstmal relativ stark ab. Die Energien
sind noch gering genug, dass die relativistische Korrektur
vernachlässigbar klein ist, daher ist der Energieverlust proportional zu
\(\frac{\ln(E)}{E}\).

Werden die kinetischen Energien größer, so wird logarithmische Anteil
langsam näherungsweise konstant, dann dominiert der
\(\frac{1}{E}\)-Anteil.

Bei der Ruheenergie des \(\alpha\)-Teilchens weist die Kurve ein Minimum
auf. Ab diesem Punkt ist die relativistische Korrektur zu
berücksichtigen. Physikalisch lässt sich der Verlauf nach dem Peak
dadurch erklären, dass das Projektil noch lange den Coulomb-Feldern der
Kerne des Bremsmediums ausgesetzt ist und dadurch stark abgebremst wird.
Mit steigender kinetischer Energie wird diese Beeinflussung immer
kürzer, bis irgendwann der Bereich eintritt, in welchem die
relativistischen Effekte eine dominante Rolle einnehmen.

\hypertarget{geltungsbereich}{%
\subsubsection{Geltungsbereich}\label{geltungsbereich}}

Die Bethe-Bloch-Gleichung gilt weder für sehr kleine, noch für sehr
große Projektilenergien.

Bei sehr kleinen Energien kann nicht mehr davon ausgegangen werden, dass
die Elektronen relativ zum Projektil in Ruhe liegen.

Bei sehr großen Energien kann z.B. die Wechselwirkung des Projektils mit
dem Atomkern relevant werden, die in der hiesigen Betrachtung
vernachlässigbar war.

Weiterhin muss das Projektil sehr schwer im Vergleich zu Elektronen
sein, da ansonsten die Näherung einer geraden Flugbahn des Projektils
nicht mehr angenommen werden kann.

\hypertarget{bragg-kurve}{%
\subsubsection{Bragg-Kurve}\label{bragg-kurve}}

Die Bragg-Kurve beschreibt den gesamten Energieverlust eines geladenen
Teilchens abhängig von der in einem Bremsmedium zurückgelegten Strecke.
Damit wird sie durch die integrierte Bethe-Bloch-Gleichung beschrieben.

\[
\begin{eqnarray}
    \frac{\Delta E}{\mathrm dx}(x) &=&
        \int_0^x \left(\frac{\mathrm dE}{\mathrm dx}\right) \mathrm dx^\prime
\end{eqnarray}
\]

Je weiter das Projektil in das Bremsmedium eindringt, desto größer wird
der Energieverlust. Bei der mittleren Reichweite \(\bar R\) des
Projektils ist ein Maximum erreicht, dann fällt die Kurve nahezu
senkrecht ab. In diesem Bereich kommt das Projektil zum Stillstand. Da
dies durch Straggling keine feste Grenze hat, flacht die Kurve ganz am
Ende wieder leicht ab.

Extrapoliert man den steilen Abfall, kann man die extrapolierte
Reichweite \(R_\mathrm{ex}\) ermitteln. Dabei wird die Abflachung der
Kurve durch Straggling herausgerechnet.

Für eine feste Eindringtiefe \(x\) kann die Restenergie
\(E_\mathrm{Rest}(x)\) ermittelt werden.

\[
\begin{eqnarray}
    E_\mathrm{Rest}
        &=& E_0
        - \int_0^x \left(\frac{\mathrm dE}{\mathrm dx}\right) \mathrm dx^\prime
        \label{Restenergie}
\end{eqnarray}
\]

\hypertarget{reichweite-von-alpha-teilchen}{%
\subsection{\texorpdfstring{Reichweite von
\(\alpha\)-Teilchen}{Reichweite von \textbackslash alpha-Teilchen}}\label{reichweite-von-alpha-teilchen}}

\hypertarget{abhuxe4ngigkeit-vom-druck}{%
\subsubsection{Abhängigkeit vom Druck}\label{abhuxe4ngigkeit-vom-druck}}

Der Luftdruck in einer Kammer ist einfacher zu variieren als der Abstand
zwischen Quelle und Detektor, insbesondere kleine Änderungen im Druck
sind leichter zu erreichen als minimale Abstandsänderungen. Daher soll
die Abhängigkeit zwischen dem mittleren Druck \(\bar p\) und der
mittleren Reichweite \(\bar R\) der \(\alpha\)-Teilchen ermittelt
werden.

Die ideale Gasgleichung \(\eqref{iGG}\) bringt die Teilchenzahl \(N\),
die Temperatur \(T\), den Druck \(p\) und das Volumen \(V\) unter
Verwendung der Boltzmann-Konstante \(k_B\) in Relation. Wenn Volumen und
Temperatur konstant gehalten werden können, dann kann eine
Proportionalität \(N\propto p\) ermittelt werden.

\[
\begin{eqnarray}
    k_B N T &=& pV \label{iGG} \\
    N(p) &=& \frac{V}{k_B T}
\end{eqnarray}
\]

Der Energieverlust pro Weglänge \(-\frac{\mathrm dE}{\mathrm dx}\) wird
durch die Bethe-Bloch-Gleichung \(\eqref{BetheBloch}\) beschrieben. Er
ist proportional zur Teilchenzahl \(N\), folglich auch zum Druck \(p\).

\[
\begin{eqnarray}
    -\frac{\mathrm dE}{\mathrm dx}\propto N\propto p \label{dEtoP}
\end{eqnarray}
\]

Die mittlere Reichweite \(\bar R\) ist die Position, an der
Erwartungswert der Restenergie \(E_\mathrm{Rest}\) verschwindet. Daher
kann sie aus dem Energieverlust pro Weglänge
\(-\frac{\mathrm dE}{\mathrm dx}\) ermittelt werden.

\[
\begin{eqnarray}
    \bar{R} &=&
        \int_{E_0}^{0} -\left(\frac{\mathrm dx}{\mathrm dE}\right) \mathrm dE
\end{eqnarray}
\]

Für eine feste Energie \(E_i \in \{0, E_0\}\) ist die Geschwindigkeit
der Teilchen konstant. Damit ist nur noch der Druck \(p_i\) variabel und
aus Gleichung \(\eqref{dEtoP}\) folgt eine Proportionalität zwischen dem
\(i\)-ten Integranden und dem inversen Druck \(p_i\).

\[
\begin{eqnarray}
    -\left(\frac{\mathrm dE}{\mathrm dx}\right)_i &\propto& p_i \\
    \Leftrightarrow -\left(\frac{\mathrm dx}{\mathrm dE}\right)_i
        &\propto& \frac{1}{p_i}
\end{eqnarray}
\]

Sei der \emph{mittlere Druck} \(\bar p\) der über alle Energien
gemittelte Druck. Zu einem bestimmten Zeitpunkt seien die Drücke \(p_i\)
für alle Energien \(E_i\) näherungsweise konstant, es gelte daher
\(\forall i: p_i \approx \bar p\). Dies ist gewährleistet, wenn der
Druck während einer Messung nicht variiert wird. Dann ist die mittlere
Reichweite \(\bar R\) proportional zu dem inversen mittleren Druck
\(\bar p\).

\[
\begin{eqnarray}
    \bar{R} &=&
        \int_{E_0}^{0} -\left(\frac{\mathrm dx}{\mathrm dE}\right) \mathrm dE \\
        &\approx&
            \quad
            \int_{E_0}^{0}
                \left(
                - \frac{4\pi z^2 e^4}{m_e v^2}
                \frac{\bar pV}{k_BT}
                Z\left[
                    \ln\left(\frac{2mv^2}{\bar I}\right)
                    - \ln\left(1 - \beta^2\right)
                    - \beta^2
                    - \frac{c_K}{Z}
                \right]
            \right)^{-1}
            \mathrm dE \\
        &=&
            \frac{1}{\bar p}
            \int_{E_0}^{0}
                \left(
                - \frac{4\pi z^2 e^4}{m_e v^2}
                \frac{V}{k_BT}
                Z\left[
                    \ln\left(\frac{2mv^2}{\bar I}\right)
                    - \ln\left(1 - \beta^2\right)
                    - \beta^2
                    - \frac{c_K}{Z}
                \right]
            \right)^{-1}
            \mathrm dE \\
    \Rightarrow \bar R &\propto& \frac{1}{\bar p}
\end{eqnarray}
\]

\hypertarget{abhuxe4ngigkeit-von-der-masse}{%
\subsubsection{Abhängigkeit von der
Masse}\label{abhuxe4ngigkeit-von-der-masse}}

Leichte Teilchen folgen einer sehr ähnlichen Formel für den
Energieverlust in Materie, wie schwere Teilchen. Bei gleichen
Geschwindigkeiten sind die Energieverluste pro Weglänge identisch.

Bei gleichen kinetischen Energien \(E\) hingegen ist der Energieverlust
von leichten Teilchen geringer als der von schweren Teilchen, da das
Quadrat der Geschwindigkeit des leichteren Teilchens
\(v_\mathrm{leicht}^2\) der Masse \(m\) um einen Faktor \(\frac{M}{m}\)
kleiner als das des schwereren Teilchens der Masse \(M\) und
Geschwindigkeitsquadrat \(v_\mathrm{schwer}^2\) ist.

\[
\begin{eqnarray}
    E &=& \frac{1}{2}m_i v_i^2 \\
    \Rightarrow \frac{1}{2} m v_\mathrm{leicht}^2
        &\overset{!}{=}& \frac{1}{2} M v_\mathrm{schwer}^2 \\
    \Rightarrow v_\mathrm{leicht}^2 & = &\frac{M}{m} v_\mathrm{schwer}^2
\end{eqnarray}
\]

Aufgrund der inversen Proportionalität des Energieverlustes mit dem
Quadrat der Geschwindigkeit folgt eine Verringerung des Energieverlusts
des leichten Teilchens verglichen mit einem schweren Teilchen um den
Faktor \(\frac{m}{M} < 1\). \([7]\)

\[
\begin{eqnarray}
    - \frac{\mathrm dE}{\mathrm dx} &\propto& \frac{1}{v^2} \\
    \frac{\mathrm dE_\mathrm{leicht}}{\mathrm dx}
        &=& \frac{\mathrm dE_\mathrm{schwer}}{\mathrm dx}
            \cdot \frac{m}{M}
\end{eqnarray}
\]

\hypertarget{abschuxe4tzung-der-anzahl-von-stuxf6uxdfen-in-luft}{%
\subsection{Abschätzung der Anzahl von Stößen in
Luft}\label{abschuxe4tzung-der-anzahl-von-stuxf6uxdfen-in-luft}}

Teilchen mit deutlich größeren Massen als die Elektronenmasse verlieren
ihre Energie vor allem durch inelastische Stöße mit den Elektronen der
Atome im Bremsmedium, wie es im Abschnitt \(\ref{bremsvermuxf6gen}\)
Bremsvermögen erläutert wurde.

Dadurch werden die Elektronen entweder aus ihrer Bindung herausgestoßen
und das Atom bleibt ionisiert zurück, oder die Elektronen werden
angeregt und das Atom erreicht einen höheren energetischen Zustand.

Um nun die Anzahl an Stößen abzuschätzen, nach denen ein
\(\alpha\)-Teilchen zur Ruhe kommt, werden ein paar Annahmen getroffen.
Da der Großteil unserer Luft aus molekularem Stickstoff besteht, sei das
Bremsmedium ein Gas aus \(\ce{^{14}N}\)-Isotopen. und zweitens soll das
\(\alpha\)-Teilchen seine Energie nur abgeben, indem es den Stickstoff
genau einfach ionisiert. Dazu ist eine Ionisationsenergie $E_I\approx 14.534 \mathrm{\,eV}$ nötig. $[8]$

Unsere \(\alpha\)-Quelle \(\ce{^{241}Am}\) erzeugt primär
\(\alpha\)-Teilchen mit einer Anfangsenergie von
\(E_\alpha\approx 5.486 \mathrm{\,MeV}\). Daraus kann die erwartete
Anzahl an Stößen \(\expval{N}\) ermittelt werden.

\[
\begin{eqnarray}
    \expval{N} &=& \frac{E_\alpha}{E_I} \\
    \expval{N} &\approx& 3.78 \cdot 10^5
\end{eqnarray}
\]

\noindent
Ein \(\alpha\)-Teilchen stößt abgeschätzt also mehrere hunderttausend
mal, bevor es zur Ruhe kommt.

\hypertarget{straggling}{%
\subsection{Straggling}\label{straggling}}

Das sogenannte Straggling bezeichnet
eine statistische Streuung der betrachteten Größe mit einer bekannten
Verteilung. In diesem Experiment gibt es Reichweiten-Straggling,
Energie-Straggling und Winkel-Straggling.

Beim \emph{Reichweiten-Straggling} kommt zu einer normalverteilten
Streuung der Reichweiten \(R_i\) um die mittlere Reichweite \(\bar{R}\).
Der Reichweitenstraggling-Parameter \(\alpha^R_0\) und ist
folgendermaßen durch die experimentell gemessene Reichweite
\(R_\mathrm{ex}\) und die mittlere Reichweite \(\bar{R}\) zu bestimmen.

\[
\begin{eqnarray}
    \alpha^R_0 &=& \sqrt{2}\left(R_\mathrm{ex}-\bar{R}\right)
\end{eqnarray}
\]

Die Streuung der Energien der \(\alpha\)-Teilchen zum Zeitpunkt der
Messung wird \emph{Energiestraggling} genannt. Bei einem
monoenergetischen Strahl streuen die Energien nach dem Durchdringen von
Materie statistisch mit einer Gaußverteilung um eine mittlere Energie
\(E\). Die Breite der beobachteten Linie im Spektrum \(\alpha\) wird
durch den Stragglingparameter \(\alpha_E\) und die Auflösung des
Messapperats \(\alpha_\mathrm{res}\) beeinflusst. Berechnet wird dies
durch eine Faltung der Gaußverteilung.

\[
\begin{eqnarray}
    \alpha &=& \sqrt{\alpha_E^2 + \alpha_\mathrm{res}^2}
\end{eqnarray}
\]

Falls man einen Strahl von Teilchen misst, kommt es zudem zu
\emph{Winkelstraggling}. Im Vakuum verläuft ein solcher Strahl
geradlinig, alle Teilchen bewegen sich parallel zueinander in einem
Winkel \(\theta\) zu z.B. der Oberfläche. In Materie stoßen die Teilchen
dagegen mit anderen Atomen, dadurch wird er um den ursprünglichen Winkel
\(\theta\) gestreut.

\hypertarget{oberfluxe4chensperrschichtzuxe4hler}{%
\subsection{Oberflächensperrschichtzähler}\label{oberfluxe4chensperrschichtzuxe4hler}}

Eine Halbleiterdiode besteht aus einer Abfolge von \(p\)- und
\(n\)-dotierten Halbleiterschichten. In einem mittels Akzeptoren
\(p\)-dotierten Bereich gibt es Löcher als bewegliche Ladungen, in einem
mit Donatoren \(n\)-dotierten Halbleiter bilden Elektronen die frei
beweglichen Ladungen.

Im Grenzbereich zwischen diesen Schichten rekombinieren sich Elektronen
und Löcher, daher ist dieser Bereich frei von Ladungsträgern. Deshalb
wird diese Zone \emph{Verarmungszone} genannt, hier sind keine weiteren
Rekombinationen möglich.

\begin{figure}
	\centering
	\includegraphics{../media/B3.3/Oberflaechensperrschichtzaehler.pdf}
	\caption{Abbildung \(1\): Oberflächensperrschichtzähler Quelle: \([5]\)}
\end{figure}

Wird eine äußere Spannung angelegt, wächst oder schrumpft die
Verarmungszone, bei ausreichender Spannung verschwindet sie. In
letzterem Fall fließt Strom, daher nennt man diese Richtung
\emph{Durchlassrichtung}. Wird ein Strom in \emph{Sperrrichtung}
angelegt, so wird die Verarmungszone dagegen vergrößert. Daher kann kein
Strom fließen.

Dringt ein \(\alpha\)-Teilchen in die Verarmungszone ein, entstehen
Elektronen-Loch-Paare, während das \(\alpha\)-Teilchen gebremst wird.
Die Elektronen und Löcher werden durch eine anliegende Spannung getrennt
und sammeln sich an den Enden des jeweiligen Halbleiters. Durch einen
empfindlichen Vorverstärker wird ein Spannungsimpuls erzeugt, der von
der Energie des Teilchen abhängt. Um die Verarmungszone und damit das
Detektionsvolumen zu maximieren, wird eine Spannung in Sperrrichtung
angelegt.

Der \(\mathrm{Si}\)-Oberflächen-Sperrschichtzähler besteht aus einen
relativ dicken \(n\)-dotierten Schicht und einer dünnen \(p\)-dotierten
Schicht. Eine sehr dünne Goldschicht sorgt für ein schnelles und
verlustarmes Eindringen der \(\alpha\)-Teilchen. Der schematische Aufbau
eines Oberflächensperrschichtzählers ist in Abbildung \(1\) dargestellt.

Silizium-Halbleiterdetektoren eignet sich aufgrund ihrer Bandlücke von
\(1.11\mathrm{\,eV}\) sehr gut für \(\alpha\)-Strahlung.
Germanium-Halbleiterdetektoren sind prinzipiell ebenfalls geeignet,
müssen allerdings auf ca. \(70\,\mathrm K\) abgekühlt werden. Bei
Raumtemperatur reicht die thermische Energie aus, um die Bandlücke von
\(0.7\mathrm{\,eV}\) zu überwinden. \([6]\)

\hypertarget{durchfuxfchrung}{%
\section{Durchführung}\label{durchfuxfchrung}}

\hypertarget{versuchsaufbau}{%
\subsection{Versuchsaufbau}\label{versuchsaufbau}}

Eine \(^{241}\mathrm{Am}\)-\(\alpha\)-Strahlungsquelle und ein
Silizium-Oberflächensperrschichtzähler sind in einer geschlossenen
Kammer aufgebaut. Zudem gibt es ein Gerüst, in dem sich drei
verschiedene Folien befinden, die zwischen Quelle und Detektor geschoben
werden können. Durch eine Vakuumpumpe kann der Luftdruck in der Kammer
verringert werden.

Der Abstand zwischen Strahlungsquelle und Detektor kann variiert werden,
wobei ein relativer Abstand \(R\) in Millimetern einstellbar ist.

Das Signal des Detektors wird elektronisch verstärkt. Das verstärkte
Zeitsignal wird an einen digitalen Zähler angeschlossen, das verstärkte
Energiesignal kann entweder an ein Oszilloskop oder an einen
Vielkanaldetektor (VKA) angeschlossen werden.

\hypertarget{eichung}{%
\subsection{Eichung}\label{eichung}}

Zunächst wurde das Signal des VKA geeicht. Dazu wurde die Luft aus der
Kammer abgepumpt, bis ein minimaler Druck von ca.
\(1.2\cdot10^{-2}\mathrm{\,mbar}\) erreicht wurde. Dann wurde eine
Messung bei \(R=(0\pm0.5)\mathrm{\,mm}\) mit dem VKA aufgenommen und mit
\texttt{hdtv} \([4]\) ausgewertet.

Hierbei wurde davon ausgegangen, dass der Kanal \(0\) dem
Energienullpunkt entspricht. Weiter wurde angenommen, dass der so
gemessene Peak bei der Energie der \(\alpha\)-Strahlung von
\(5486\mathrm{\,keV}\) liegt, dies war bei Kanal \(10450.4\) der Fall.
Damit wurde \texttt{hdtv} kalibriert.

\hypertarget{energiestraggling}{%
\subsection{Energiestraggling}\label{energiestraggling}}

Um das Energiestraggling zu untersuchen, wurden bei einem eingestellten
relativen Abstand \(R=(18\pm 0.5)\mathrm{\,mm}\) die Energiespektren bei
verschiedenen Drücken \(p_i\) zwischen \(0\mathrm{\,mbar}\) und
\(1013.25\mathrm{\,mbar}=1\mathrm{\,atm}\) aufgenommen. Es wurden \(10\)
Messungen mit einer Dauer von jeweils \(\Delta t=30\mathrm{\,s}\)
getätigt. Diese Messungen wurden sogleich mit \texttt{hdtv} \([4]\)
ausgewertet.

\hypertarget{reichweite-in-luft}{%
\subsection{Reichweite in Luft}\label{reichweite-in-luft}}

Daraufhin wurde die Reichweite der \(\alpha\)-Teilchen gemessen. Dazu
wurden \(4\) verschiedene relative Abstände \(R_i\) eingestellt und je
\(R_i\) Messungen für \(10\) verschiedene Drücke \(p_{i,j}\)
aufgenommen. Hierbei sollten die \(R_i\) größer als die mittlere
Reichweite \(\bar R\) der \(\alpha\)-Teilchen in Luft bei
\(1\mathrm{\,atm}\) sein.

Dabei wurden mittels des digitalen Zählers die Anzahl Detektionen
\(n_i\) sowie die Dauern der Messungen \(\Delta t_i\) aufgezeichnet,
woraus die Zählraten ermittelt werden können. Weiterhin wurden die
Impulshöhen mithilfe des Oszilloskops gemessen.

Die Messungsdauern für die Detektionen unterscheiden sich voneinander,
da versucht wurde, in den meisten Fällen wenigstens \(4500\) Ereignisse
zu messen. Dies soll den statistischen Fehler gering halten. Für die
Messungen mit maximalem Druck wurde dieses Ziel nicht erreicht, hier
wurden maximal \(2\mathrm{\,min}\) lang gemessen.

\hypertarget{metallfolien}{%
\subsection{Metallfolien}\label{metallfolien}}

Zuletzt wurden Folien aus Metall zwischen Strahlungsquelle und Detektor
geschoben. Eine der Folien bestand aus Aluminium, die andere aus Gold.

Die Messungen erfolgten analog zu den Messungen der Reichweite in Luft
(REF), allerdings nur für einen relativen Abstand je Folie. Im Falle von
Aluminium war der relative Abstand \(R_\mathrm{Al}=4\mathrm{\,mm}\), im
Falle von Gold \(R_\mathrm{Au}=8\mathrm{\,mm}\).

\hypertarget{literatur}{%
\section{Literatur}\label{literatur}}

\begin{enumerate}
\def\labelenumi{\arabic{enumi}.}
\tightlist
\item
  K. Bethge, ``Kernphysik: Eine Einführung'', 3. Auflage,
  Springer-Verlag, 2008, \linebreak
  ISBN:~9783540745679, DOI:
  \href{https://doi.org/10.1007/978-3-540-74567-9}{10.1007/978-3-540-74567-9}
\item
  Prior und Rollefson, ``Anomalous energy straggling of alpha
  particles'', American Journal of Physics, Mai 1982,
  \href{https://doi.org/10.1119/1.12834}{DOI 10.1119/1.12834}
\item
  ``Chart of Nuclides'', National Nuclear Data Center,
  \url{https://www.nndc.bnl.gov/nudat3}, \(^{241}_{\ \ 95}\mathrm{Am}\),
  Abruf am 28.01.2024
\item
  Software \href{https://pypi.org/project/hdtv}{\texttt{hdtv}},
  Kurzanleitung unter
  \url{https://www.ikp.uni-koeln.de/fileadmin/data/praktikum/hdtv.pdf},
  Abruf am 28.01.2024
\item
  Lexikon der Physik, Spektrum Verlag,
  \url{https://www.spektrum.de/lexikon/physik/oberflaechensperrschichtzaehler/10568},
  29.01.2024
\item
  G. Knoll, ``Radiation Detection and Measurement'', Wiley, 2010, \linebreak ISBN:~9780470131480
\item
  W. Demtröder, ``Experimentalphysik 4: Kern-, Teilchen- und
  Astrophysik'', \linebreak
  Springer-Spektrum-Verlag, 2017, ISBN: 9783662528839,
  DOI:   \href{https://doi.org/10.1007/978-3-662-52884-6}{10.1007/978-3-662-52884-6}
\item
  NIH National Library of Medicine NCBI,
  ``Ionization Energy in the Periodic Table of Elements'',
  \url{https://pubchem.ncbi.nlm.nih.gov/periodic-table/ionization-energy}, Abruf am 28.01.2024
\item
  LEIFIphysik, ``Alphazerfall und Alphastrahlung'',
  \url{https://www.leifiphysik.de/kern-teilchenphysik/radioaktivitaet-einfuehrung/grundwissen/alphazerfall-und-alphastrahlung},
  Abruf am 01.03.2024
\end{enumerate}
\end{document}
