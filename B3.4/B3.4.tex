% based on a template made by the university of cologne
% http://www.mi.uni-koeln.de/wp-MIEDV/wp-content/uploads/2016/07/LaTeX-Vorlage.zip - 2023-11-02
\documentclass[12pt,a4paper]{scrartcl}

\addtokomafont{sectioning}{\rmfamily}
\usepackage[ngerman]{babel}% deutsches Sprachpaket wird geladen
\usepackage[T1]{fontenc} % westeuropäische Codierung wird verlangt
\usepackage[utf8]{inputenc}% Umlaute werden erlaubt
\usepackage[usenames]{color} % Erlaubt die Benutzung der namen im Farbpaket und deren Änderung
\usepackage{amsmath} % Erweiterung für den Mathe-Satz
\usepackage{amssymb} % alle Zeichen aus msam und msmb werden dargestellt
\usepackage{graphicx} % Graphiken und Bilder können eingebunden werden
%\usepackage{multirow} % erlaubt in einer Spalte einer Tabelle die Felder in mehreren Zeilen zusammenzufassen
\usepackage{enumerate} % erlaubt Nummerierungen
\usepackage{xurl} % Dient zur Auszeichnung von URLs; setzt die Adresse in Schreibmaschinenschrift.
\usepackage[center]{caption}  % Bildunterschrift wird zentriert
%\usepackage{subfigure} % mehrere Bilder können in einer fugure-Umgebung verwendet werden
%\usepackage{longtable} % Diese Umgebung ist ähnlich definiert wie die tabular-Umgebung, erlaubt jedoch mehrseitige Tabellen.
%\usepackage{paralist} % Modifikation der bereits bestehenden Listenumgebungen
\usepackage{lmodern}% Für die Schrift
\usepackage[hidelinks]{hyperref} % Links und Verweise werden innerhalb von PDF Dokumenten erzeugt
%\usepackage{wrapfig} % Das Paket ermöglicht es von Schrift umflossene Bilder und Tabellen einzufügen.
\usepackage{latexsym} % LaTeX-Symbole werden geladen
\usepackage{tikz} % Erlaubt es mit tikz zu zeichnen
\usepackage{tabularx} % Erlaubt Tabellen
\usepackage{algorithm} % Erlaubt Pseudocode
\usepackage{color} % Farbpaket wird geladen
%\usepackage{stmaryrd} % St Mary Road Symbole werden geladen
\usepackage{physics}
\usepackage{mhchem} % Chemie: \ce & \pu

\numberwithin{equation}{section} % Nummerierungen der Gleichungen, die durch equation erstellt werden, sind gebunden an die section
\newcommand{\HRule}{\rule{\linewidth}{0.7mm}}
\newcommand{\pu}[1]{\ensuremath{\mathrm{#1}}}

% disable commands
\renewcommand{\[}{} % math block start
\renewcommand{\]}{\noindent} % math block end
\newcommand{\tightlist}{} % created in enumerations

\hypersetup{
  pdftitle={B3.4},
  pdfcreator={LaTeX via pandoc}}

\setcounter{secnumdepth}{6}
\setcounter{tocdepth}{6}

\begin{document}
\begin{titlepage}
	\pagestyle{empty}

	\begin{center}

	\textsc{\LARGE Universität zu Köln }\\ [0.4cm]
	\textsc{Mathematisch-Naturwissenschaftliche Fakultät} \\[1.5cm]

	\includegraphics[width=0.45\textwidth]{../media/uni.jpg}\\[1.5cm]  % Uni-Logo wird geladen

	\textsc{\Large Praktikum~B}\\[2mm]
	\textsc{}\\[10mm]
	\HRule \\[0.4cm]

		{	\Huge \bfseries B3.4}\\[0.4cm]
			{	\huge \bfseries Positronen--Emissions--Tomografie}\\[0.3cm]
	
	\HRule \\[3cm]

 	\begin{center}
		\textsc{\Large Catherine~Tran } \\[3pt]
		\textsc{\Large Carlo~Kleefisch } \\[3pt]
		\textsc{\Large Oliver~Filla } \\[3pt]
	\end{center}
	\end{center}
\end{titlepage}

\newpage
\tableofcontents
\newpage

\hypertarget{einleitung}{%
\section{Einleitung}\label{einleitung}}

Die Positronen--Emissions--Tomografie (PET) ist ein nuklearmedizinisches
bildgebendes Verfahren, bei denen radioaktive Materialien als
\emph{Tracer} verabreicht werden, die dann im Körper des Patienten
gemessen werden. Dadurch können Bilder von z.B. Krebszellen erzeugt
werden.

In diesem Versuch wird eine radioaktive Quelle in einer Probe mittels
der PET lokalisiert. Weiterhin wird die Ortsauflösung sowie die
Winkelabhängigkeit untersucht.

\clearpage
\hypertarget{theoretische-grundlagen}{%
\section{Theoretische Grundlagen}\label{theoretische-grundlagen}}

\hypertarget{pet}{%
\subsection{PET}\label{pet}}

\hypertarget{betazerfall}{%
\subsection{\texorpdfstring{\(\beta\)--Zerfall}{\textbackslash beta--Zerfall}}\label{betazerfall}}

Der \(\beta\)--Zerfall ist eine der drei Arten radioaktiven Zerfalls.
Wie beim \(\alpha\)--Zerfall wird dabei ein chemisches Element in ein
anderes umgewandelt. Im Unterschied zum \(\alpha\)--Zerfall werden
hierbei keine Nukleonen abgesondert, sondern ein Nukleon in ein anderes
umgewandelt.

Es wird zwischen drei Arten des \(\beta\)--Zerfalls unterschieden. Es
gibt \(\beta^+\)--Zerfall und \(\beta^-\)--Zerfall, weiterhin wird der
Elektroneneinfang dazugezählt.

Radioaktiver Zerfall geschieht, wenn der Tochterkern eine höhere
Bindungsenergie als der Mutterkern erhält. Hierfür ist insbesondere der
Symmetrieterm der Weizsäcker Massenformel \((??)\) interessant. Daran
kann man erklären, dass \(\beta^-\)--Zerfall dann geschehen kann, wenn
der Kern mehr Neutronen als Protonen hat.

Umgekehrt kann es zu \(\beta^+\)--Zerfall kommen, wenn der Kern mehr
Protonen als Neutronen hat. Ebenso ist Elektroneneinfang möglich.
Welcher dieser Prozesse stattfindet hängt von der Energiedifferenz
zwischen Mutterkern und Tochterkern ab: Wenn die Energiedifferenz
kleiner als \(\Delta E_\mathrm{min}=\pu{1.022 MeV}\) beträgt, dann
reicht die Energie nur für Elektroneneinfang. Bei einer Energie von mehr
als \(\Delta E_\mathrm{min}\) ist auch \(\beta^+\)--Zerfall möglich.
\(\Delta E_\mathrm{min}\) entspricht der doppelten Ruheenergie von
Elektronen.

Beim \(\beta^+\)--Zerfall wird ein Proton \(p\) in ein Neutron \(n\)
umgewandelt, dabei entstehen ein Positron \(e^+\) sowie ein
Elektronenneutrino \(\nu_e\). Hierbei wird ein Mutterkern
\(\ce{^A_Z X}\) in einen Tochterkern \(\ce{^A_{Z-1} Y^-}\) umgewandelt.
Beim \(\beta^-\)--Zerfall wird ein Neutron \(n\) in ein Proton \(p\)
umgewandelt, dabei entstehen ein Positron \(e^+\) sowie ein
Anti--Elektronenneutrino \(\bar{\nu}_e\). Hierbei wird ein Mutterkern
\(\ce{^A_Z X}\) in einen Tochterkern \(\ce{^A_{Z+1} Y^+}\) umgewandelt.

Beim Elektroneneinfang (ec) wird ähnlich wie beim \(\beta^+\)--Zerfall
ein Proton \(p\) in ein Neutron \(n\) umgewandelt. Allerdings wird dazu
ein Elektron \(e^-\) verwendet, dass aus der \(K\)--Schale des Atoms
eingefangen wurde. Daher wird nur ein Elektronenneutrino \(\nu_e\)
erzeugt. Hierbei wird ein Mutterkern \(\ce{^A_Z X}\) in einen
Tochterkern \(\ce{^A_{Z-1} Y}\) umgewandelt.

\[
\begin{eqnarray}
    \beta^+:\qquad\,\,\qquad
        \ce{p} &\ce{->}& \quad \ce{n} \quad\, + \ce{e^+} + \ce{\nu_e} \\
        \ce{^A_Z X} &\ce{->}& \ce{^A_{Z-1} Y^-} + \ce{e^+} + \ce{\nu_e} \\
    \beta^-:\qquad\,\,\qquad
        \ce{n} &\ce{->}& \quad \ce{p} \quad\, + \ce{e^-} + \ce{\bar{\nu}_e} \\
        \ce{^A_Z X} &\ce{->}& \ce{^A_{Z+1} Y^+} + \ce{e^-} + \ce{\bar{\nu}_e} \\
    \text{ec}:\qquad
        \ce{p + e^+} &\ce{->}& \quad \ce{n} \ \ \, + \ce{\nu_e} \\
        \ce{^A_Z X} &\ce{->}& \ce{^A_{Z-1} Y} + \ce{\nu_e}
\end{eqnarray}
\]

\hypertarget{paarvernichtung}{%
\subsection{Paarvernichtung}\label{paarvernichtung}}

Der Begriff der Paarvernichtung beschreibt die Zerstrahlung von einem
Teilchen und seinem Antiteilchen, bei der die Teilchen in
\(\gamma\)--Strahlung umgewandelt werden. Die Energie der Strahlung ist
die Summe der kinetischen Energie dieser Teilchen sowie ihrer
Ruhemassen. Die Paarerzeugung ist der dazu inverse Effekt. \([5]\)

Bei der Vernichtung eines Positrons \(e^+\) mit einem Elektron \(e^-\)
entstehen normalerweise zwei \(\gamma\)--Quanten in einem Winkel
\(\theta\). Dieser hängt von der transversalen Impulskomponente \(p_T\)
ab. Weiterhin sind die Ruhemasse \(m_e\) der Teilchen und die
Lichtgeschwindigkeit \(c\) relevant. \([6]\)

\[
\begin{eqnarray}
    \tan(\theta) &=& \frac{p_T}{m_ec}
\end{eqnarray}
\]

Sind die beiden Teilchen in Ruhe zueinander, so ist der Abstahlwinkel
der \(\gamma\)--Quanten \(\pu{180^\circ}\). Haben sie einen relativen
Impuls, so wird der Winkel kleiner. Sind die Teilchen jedoch zu schnell,
dann ist der Wirkungsquerschnitt sehr klein und es ist sehr
unwahrscheinlich, dass Paarvernichtung stattfindet.

\hypertarget{gamma-strahlung}{%
\subsection{\texorpdfstring{\(\gamma\)-Strahlung}{\textbackslash gamma-Strahlung}}\label{gamma-strahlung}}

\ldots{}

\hypertarget{szintillatoren-und-photomultiplier}{%
\subsection{Szintillatoren und Photomultiplier}\label{szintillatoren-und-photomultiplier}}

\clearpage
\hypertarget{durchfuxfchrung}{%
\section{Durchführung}\label{durchfuxfchrung}}

\clearpage
\hypertarget{auswertung}{%
\section{Auswertung}\label{auswertung}}

\clearpage
\hypertarget{fazit}{%
\section{Fazit}\label{fazit}}

\clearpage
\hypertarget{literatur}{%
\section{Literatur}\label{literatur}}

\begin{enumerate}
\def\labelenumi{\arabic{enumi}.}
\tightlist
\item
  Universität zu Köln, ``B3.4: Positronen--Emissions--Tomografie'',
  Januar 2021, Online verfügbar unter
  \url{https://www.ikp.uni--koeln.de/fileadmin/data/praktikum/B3.4\_PET\_de.pdf}
\item
  ``Chart of Nuclides'', National Nuclear Data Center,
  \url{https://www.nndc.bnl.gov/nudat3}, Abruf am 28.03.2024
\item
  ``Positronen Emissions Tomographie'', Deutsche Gesellschaft für
  Nuklearmedizin e.V., Online verfügbar unter
  \url{http://www.nuklearmedizin.de/docs/pet_bro_06.pdf}, Abruf am
  03.04.2024
\item
  W. Demtröder, ``Experimentalphysik 4: Kern-, Teilchen- und
  Astrophysik'', Springer--Spektrum--Verlag, 2017, DOI:
  \href{https://link.springer.com/book/10.1007/978-3-662-52884-6}{10.1007/978-3-662-52884-6}
\item
  Lexikon der Physik, ``Paarvernichtung'',
  \url{https://www.spektrum.de/lexikon/physik/paarvernichtung/10838},
  Abruf am 04.04.2024
\item
  Wikipedia, ``Annihilation'',
  \url{https://de.wikipedia.org/wiki/Annihilation}, Abruf am 04.04.2024
\end{enumerate}

\end{document}
