% based on a template made by the university of cologne
% http://www.mi.uni-koeln.de/wp-MIEDV/wp-content/uploads/2016/07/LaTeX-Vorlage.zip - 2023-11-02
\documentclass[12pt,a4paper]{scrartcl}

\addtokomafont{sectioning}{\rmfamily}
\usepackage[ngerman]{babel}% deutsches Sprachpaket wird geladen
\usepackage[T1]{fontenc} % westeuropäische Codierung wird verlangt
\usepackage[utf8]{inputenc}% Umlaute werden erlaubt
\usepackage[usenames]{color} % Erlaubt die Benutzung der namen im Farbpaket und deren Änderung
\usepackage{amsmath} % Erweiterung für den Mathe-Satz
\usepackage{amssymb} % alle Zeichen aus msam und msmb werden dargestellt
\usepackage{graphicx} % Graphiken und Bilder können eingebunden werden
%\usepackage{multirow} % erlaubt in einer Spalte einer Tabelle die Felder in mehreren Zeilen zusammenzufassen
\usepackage{enumerate} % erlaubt Nummerierungen
\usepackage{xurl} % Dient zur Auszeichnung von URLs; setzt die Adresse in Schreibmaschinenschrift.
\usepackage[center]{caption}  % Bildunterschrift wird zentriert
%\usepackage{subfigure} % mehrere Bilder können in einer fugure-Umgebung verwendet werden
%\usepackage{longtable} % Diese Umgebung ist ähnlich definiert wie die tabular-Umgebung, erlaubt jedoch mehrseitige Tabellen.
%\usepackage{paralist} % Modifikation der bereits bestehenden Listenumgebungen
\usepackage{lmodern}% Für die Schrift
\usepackage[hidelinks]{hyperref} % Links und Verweise werden innerhalb von PDF Dokumenten erzeugt
%\usepackage{wrapfig} % Das Paket ermöglicht es von Schrift umflossene Bilder und Tabellen einzufügen.
\usepackage{latexsym} % LaTeX-Symbole werden geladen
\usepackage{tikz} % Erlaubt es mit tikz zu zeichnen
\usepackage{tabularx} % Erlaubt Tabellen
\usepackage{algorithm} % Erlaubt Pseudocode
\usepackage{color} % Farbpaket wird geladen
%\usepackage{stmaryrd} % St Mary Road Symbole werden geladen
\usepackage{physics}
\usepackage{mhchem} % Chemie: \ce & \pu

\numberwithin{equation}{section} % Nummerierungen der Gleichungen, die durch equation erstellt werden, sind gebunden an die section
\newcommand{\HRule}{\rule{\linewidth}{0.7mm}}
\newcommand{\pu}[1]{\ensuremath{\mathrm{#1}}}

% disable commands
\newcommand{\tightlist}{} % created in enumerations

\hypersetup{
  pdftitle={B3.4},
  pdfcreator={LaTeX via pandoc}}

\setcounter{secnumdepth}{6}
\setcounter{tocdepth}{6}

\begin{document}
\begin{titlepage}
	\pagestyle{empty}

	\begin{center}

	\textsc{\LARGE Universität zu Köln }\\ [0.4cm]
	\textsc{Mathematisch-Naturwissenschaftliche Fakultät} \\[1.5cm]

	\includegraphics[width=0.45\textwidth]{../media/uni.jpg}\\[1.5cm]  % Uni-Logo wird geladen

	\textsc{\Large Praktikum~B}\\[2mm]
	\textsc{}\\[10mm]
	\HRule \\[0.4cm]

		{	\Huge \bfseries B3.4}\\[0.4cm]
			{	\huge \bfseries Positronen--Emissions--Tomografie}\\[0.3cm]
	
	\HRule \\[3cm]

 	\begin{center}
		\textsc{\Large Catherine~Tran } \\[3pt]
		\textsc{\Large Carlo~Kleefisch } \\[3pt]
		\textsc{\Large Oliver~Filla } \\[3pt]
	\end{center}
	\end{center}
\end{titlepage}

\newpage
\tableofcontents
\newpage

\hypertarget{einleitung}{%
\section{Einleitung}\label{einleitung}}

Die Positronen--Emissions--Tomografie (PET) ist ein nuklearmedizinisches
bildgebendes Verfahren, bei denen radioaktive Materialien als
\emph{Tracer} verabreicht werden, die dann im Körper des Patienten
gemessen werden. Dadurch können Bilder von z.B. Krebszellen erzeugt
werden.

In diesem Versuch wird eine radioaktive Quelle in einer Probe mittels
der PET lokalisiert. Weiterhin wird die Ortsauflösung sowie die
Winkelabhängigkeit untersucht.

\clearpage
\hypertarget{theoretische-grundlagen}{%
\section{Theoretische Grundlagen}\label{theoretische-grundlagen}}

\hypertarget{pet}{%
\subsection{PET}\label{pet}}

\hypertarget{betazerfall}{%
\subsection{\texorpdfstring{\(\beta\)--Zerfall}{\textbackslash beta--Zerfall}}\label{betazerfall}}

Der \(\beta\)--Zerfall ist eine der drei Arten radioaktiven Zerfalls.
Wie beim \(\alpha\)--Zerfall wird dabei ein chemisches Element in ein
anderes umgewandelt. Im Unterschied zum \(\alpha\)--Zerfall werden
hierbei keine Nukleonen abgesondert, sondern ein Nukleon in ein anderes
umgewandelt.

Es wird zwischen drei Arten des \(\beta\)--Zerfalls unterschieden. Es
gibt \(\beta^+\)--Zerfall und \(\beta^-\)--Zerfall, weiterhin wird der
Elektroneneinfang dazugezählt.

Beim \(\beta^+\)--Zerfall wird ein Proton \(\ce p\) in ein Neutron
\(\ce n\) umgewandelt, dabei entstehen ein Positron \(\ce{e^+}\) sowie
ein Elektronenneutrino \(\ce{\nu_e}\). Beim \(\beta^-\)--Zerfall wird
dagegen ein Neutron \(\ce n\) in ein Proton \(\ce p\) umgewandelt, dabei
entstehen ein Positron \(e^+\) sowie ein Antielektronenneutrino
\(\ce{\bar{\nu}_e}\).

Beim Elektroneneinfang (ce) wird ähnlich wie beim \(\beta^+\)--Zerfall
ein Proton \(\ce p\) in ein Neutron \(\ce n\) umgewandelt. Allerdings
wird dazu ein Elektron \(\ce{e^-}\) verwendet, dass aus der
\(K\)--Schale des Atoms eingefangen wurde. Daher wird nur ein
Elektronenneutrino \(\ce{\nu_e}\) erzeugt.

\[
\begin{aligned}
    \beta^+: &&\ce{p} &\ce{->}&& \ce{n} + \ce{e^+} + \ce{\nu_e} \\
    \beta^-: &&\ce{n} &\ce{->}&& \ce{p} + \ce{e^-} + \ce{\bar{\nu}_e} \\
    \mathrm{ce}: &&\ce{p + e^+} &\ce{->}&& \ce{n} \qquad\,+ \ce{\nu_e}
\end{aligned}
\]

\hypertarget{paarvernichtung}{%
\subsection{Paarvernichtung}\label{paarvernichtung}}

\ldots{}

\hypertarget{gamma-strahlung}{%
\subsection{\texorpdfstring{\(\gamma\)-Strahlung}{\textbackslash gamma-Strahlung}}\label{gamma-strahlung}}

\ldots{}

\hypertarget{szintillatoren-und-photomultiplier}{%
\subsection{Szintillatoren und Photomultiplier}\label{szintillatoren-und-photomultiplier}}

\clearpage
\hypertarget{durchfuxfchrung}{%
\section{Durchführung}\label{durchfuxfchrung}}

\clearpage
\hypertarget{auswertung}{%
\section{Auswertung}\label{auswertung}}

\clearpage
\hypertarget{fazit}{%
\section{Fazit}\label{fazit}}

\clearpage
\hypertarget{literatur}{%
\section{Literatur}\label{literatur}}

\begin{enumerate}
\def\labelenumi{\arabic{enumi}.}
\tightlist
\item
  Universität zu Köln, ``B3.4: Positronen--Emissions--Tomografie'',
  Januar 2021, Online verfügbar unter
  \url{https://www.ikp.uni-koeln.de/fileadmin/data/praktikum/B3.4_PET_de.pdf}
\end{enumerate}

\end{document}
