% based on a template made by the university of cologne
% http://www.mi.uni-koeln.de/wp-MIEDV/wp-content/uploads/2016/07/LaTeX-Vorlage.zip - 2023-11-02
\documentclass[12pt,a4paper]{scrartcl}

\addtokomafont{sectioning}{\rmfamily}
\usepackage[ngerman]{babel}% deutsches Sprachpaket wird geladen
\usepackage[T1]{fontenc} % westeuropäische Codierung wird verlangt
\usepackage[utf8]{inputenc}% Umlaute werden erlaubt
\usepackage[usenames]{color} % Erlaubt die Benutzung der namen im Farbpaket und deren Änderung
\usepackage{amsmath} % Erweiterung für den Mathe-Satz
\usepackage{amssymb} % alle Zeichen aus msam und msmb werden dargestellt
\usepackage{graphicx} % Graphiken und Bilder können eingebunden werden
%\usepackage{multirow} % erlaubt in einer Spalte einer Tabelle die Felder in mehreren Zeilen zusammenzufassen
\usepackage{enumerate} % erlaubt Nummerierungen
\usepackage{xurl} % Dient zur Auszeichnung von URLs; setzt die Adresse in Schreibmaschinenschrift.
\usepackage[center]{caption}  % Bildunterschrift wird zentriert
%\usepackage{subfigure} % mehrere Bilder können in einer fugure-Umgebung verwendet werden
%\usepackage{longtable} % Diese Umgebung ist ähnlich definiert wie die tabular-Umgebung, erlaubt jedoch mehrseitige Tabellen.
%\usepackage{paralist} % Modifikation der bereits bestehenden Listenumgebungen
\usepackage{lmodern}% Für die Schrift
\usepackage[hidelinks]{hyperref} % Links und Verweise werden innerhalb von PDF Dokumenten erzeugt
%\usepackage{wrapfig} % Das Paket ermöglicht es von Schrift umflossene Bilder und Tabellen einzufügen.
\usepackage{latexsym} % LaTeX-Symbole werden geladen
\usepackage{tikz} % Erlaubt es mit tikz zu zeichnen
\usepackage{tabularx} % Erlaubt Tabellen
\usepackage{algorithm} % Erlaubt Pseudocode
\usepackage{color} % Farbpaket wird geladen
%\usepackage{stmaryrd} % St Mary Road Symbole werden geladen
\usepackage{physics}
\usepackage{mhchem} % Chemie: \ce & \pu

\numberwithin{equation}{section} % Nummerierungen der Gleichungen, die durch equation erstellt werden, sind gebunden an die section
\newcommand{\HRule}{\rule{\linewidth}{0.7mm}}
\newcommand{\pu}[1]{\ensuremath{\mathrm{#1}}}

% disable commands
\renewcommand{\[}{} % math block start
\renewcommand{\]}{\noindent} % math block end
\newcommand{\tightlist}{} % created in enumerations

\hypersetup{
  pdftitle={B3.1},
  pdfcreator={LaTeX via pandoc}}

\setcounter{secnumdepth}{6}
\setcounter{tocdepth}{6}

\begin{document}
\begin{titlepage}
	\pagestyle{empty}

	\begin{center}

	\textsc{\LARGE Universität zu Köln }\\ [0.4cm]
	\textsc{Mathematisch-Naturwissenschaftliche Fakultät} \\[1.5cm]

	\includegraphics[width=0.45\textwidth]{../media/uni.jpg}\\[1.5cm]  % Uni-Logo wird geladen

	\textsc{\Large Praktikum~B}\\[2mm]
	\textsc{}\\[10mm]
	\HRule \\[0.4cm]

		{	\Huge \bfseries B3.1}\\[0.4cm]
			{	\huge \bfseries Statistik der Kernzerfälle}\\[0.3cm]
	
	\HRule \\[3cm]

 	\begin{center}
		\textsc{\Large Catherine~Tran } \\[3pt]
		\textsc{\Large Carlo~Kleefisch } \\[3pt]
		\textsc{\Large Oliver~Filla } \\[3pt]
	\end{center}
	\end{center}
\end{titlepage}

\newpage
\tableofcontents
\newpage

\hypertarget{einleitung}{%
\section{Einleitung}\label{einleitung}}

In diesem Versuch wird die statistische Methode des
\(\chi^2\)--Anpassungstests mithilfe von radioaktiver Strahlung
untersucht. Dazu wird \(\ce{^{137}Cs}\) verwendet, dessen Stahlung mit
einem Geiger--Müller--Zählrohr detektiert wird.

Damit wird die Hypothese getestet, dass die Präparatstärke im Rahmen der
Messung konstant bleibt. Dazu sind die Varianz der gemessenen Werte und
die Totzeit des Detektors von essenzieller Bedeutung.

\clearpage
\hypertarget{theoretische-grundlagen}{%
\section{Theoretische Grundlagen}\label{theoretische-grundlagen}}

\hypertarget{statistische-tests}{%
\subsection{Statistische Tests}\label{statistische-tests}}

\hypertarget{hypothesentest}{%
\subsubsection{Hypothesentest}\label{hypothesentest}}

Ein Hypothesentest oder Statistischer Test dient dazu, durch eine
Hypothese mittels statistischer Messungen zu prüfen.

Dazu verwendet man eine \emph{Nullhypothese}\footnote{\emph{Hypothesis
  to be nullified} \([5]\)} \(H_0\) und eine \emph{Gegenhypothese} oder
\emph{Alternativhypothese} \(H_1\), die sich unterscheiden. Ziel des
Tests ist es, die Alternativhypothese \(H_1\) zu belegen. Falls dies
nicht gelingt, muss man die Nullhypothese \(H_0\) als wahr annehmen.
Diese wird nicht überprüft. \([2]\)

Aufgrund der Zufälligkeit der Ereignisse kann es dabei zwei Arten von
Fehlern geben. Ein \(\alpha\)--Fehler beschreibt das irrtümliche
Ablehnen von \(H_0\), während ein \(\beta\)--Fehler das fälschliche
Annehmen von \(H_0\) bezeichnet.

\hypertarget{fehlerarten}{%
\subsubsection{Fehlerarten}\label{fehlerarten}}

Ein \emph{Fehler erster Art} oder \(\alpha\)--Fehler beschreibt die
fälschliche Ablehnung der Nullhypothese \(H_0\) in einem Statistischen
Test. Man nimmt z.B. an, dass ein Würfel gezinkt ist \((H_1)\), obwohl
er in Wahrheit fair ist \((H_0)\). Hierbei ist die \(H_0\) die Annahme
eines fairen Würfels. Man spricht hier auch von einem
\emph{falsch--positiven} Ergebnis. \([3]\)

Ein \emph{Fehler zweiter Art} oder \emph{\(\beta\)--Fehler} beschreibt
umgekehrt die fälschliche Akzeptanz der Nullhypothese \(H_0\).
Beispielsweise geht man davon aus, dass ein Würfel fair ist \((H_0)\),
obwohl er tatsächlich unfair ist \((H_1)\). Man spricht hier auch von
einem \emph{falsch--negativen} Ergebnis. \([3]\)

Die statistische Signifikanz beschreibt die erlaubte Wahrscheinlichkeit,
einen \(\alpha\)--Fehler zu begehen. \([4]\) In einem
\emph{Alternativtest} dagegen beschreibt die Signifikanz die
Wahrscheinlichkeit, einen \(\alpha\)- oder einen \(\beta\)--Fehler zu
machen. Bei einer Signifikanz \(Y\) sind \(\alpha\)- und
\(\beta\)--Fehler mit einer Wahrscheinlichkeit von je \(\frac{Y}{2}\)
erlaubt.

\hypertarget{der-chi2anpassungstest}{%
\subsubsection{\texorpdfstring{Der
\(\chi^2\)--Anpassungstest}{Der \textbackslash chi\^{}2--Anpassungstest}}\label{der-chi2anpassungstest}}

Der \(\chi^2\)--Anpassungstest dient dazu, eine Verteilung von
Zufallsvariablen \(A\) mit einer theoretischen Verteilung zu
vergleichen. Man kann mithilfe des Tests bewerten, ob die
Zufallsvariablen der Verteilung entsprechen können. Hierbei werden
sowohl Fehler 1. Art als auch Fehler 2. Art berücksichtigt.

Die Grundidee dahinter ist, einen Erwartungswert \(\expval{A}\) und
seine Varianz \(\sigma_A^2\) bewerten zu können. Das Maß für die
Abweichung von der Hypothese wird für einen Freiheitsgrad durch
\(\chi^2\) beschrieben, \footnote{Man könnte auch den Betrag \(|x_i|\)
  anstatt des Quadrates \(x_i^2\) wählen. Dies wird nicht gemacht, weil
  damit schwieriger zu rechnen ist.} was durch die
\(\chi^2\)--Verteilung beschrieben wird.

\[
\begin{eqnarray}
    \chi^2 &=& \sum_i x_i^2
\end{eqnarray}
\]

Mithilfe der \(\chi^2\)--Verteilung kann eine Signifikanz \(Y\)
festgelegt werden. Damit kann ein Intervall
\([\chi^2_\mathrm{min}, \chi^2_\mathrm{max}]\) durch die
Verteilungsfunktion \(F(x, f)\) ermittelt werden. Liegt das ermittelte
\(\chi^2\) in diesem Interval, so kann \(H_1\) als signifikant gültig
angenommen werden.

\[
\begin{eqnarray}
    F(\chi^2_\mathrm{min}, f) &=& \frac{Y}{2} \\
    F(\chi^2_\mathrm{max}, f) &=& 1 - \frac{Y}{2}
\end{eqnarray}
\]

Oft wird die Signifikanz von \(Y=5\,\%\) gefordert, wodurch das
Gültigkeitsintervall durch folgende Gleichungen bestimmt wird.

\[
\begin{eqnarray}
    F(\chi^2_\mathrm{min}, f) &=& 0.025 \label{eq:ChiMinFormula} \\
    F(\chi^2_\mathrm{max}, f) &=& 0.975 \label{eq:ChiMaxFormula}
\end{eqnarray}
\]

\hypertarget{die-chi2verteilung}{%
\subsubsection{\texorpdfstring{Die
\(\chi^2\)--Verteilung}{Die \textbackslash chi\^{}2--Verteilung}}\label{die-chi2verteilung}}

Sei \(A\) standardnormalverteilt\footnote{Diese Annahme ist bei
  ausreichend vielen Messung durch das Gesetz der großen Zahl
  gerechtfertigt.}, dann ist die \(\chi_1^2\)--Verteilung eine
quadrierte Normalverteilung mit einem Freiheitsgrad. Daher ist der
Erwartungswert \(\expval{\chi_1^2}=1\). Gibt es mehrere Freiheitsgrade
\(f\), so müssen \(f\) Erwartungswerte \(\expval{\chi_i^2}\) addiert
werden, um den gesamten Erwartungswert zu ermitteln. Dies wird durch die
Wahrscheinlichkeitsdichte (PDF\footnote{\emph{probability density
  function}}) \(f(x, f)\) beschrieben,\footnote{Achtung: Hier wird zur
  besseren Lesbarkeit \(f(x, 2f)\) angegeben, die Zahl der
  Freiheitsgrade wird in der Funktion halbiert.} wobei die Gammafunktion
\(\Gamma(x)\) benötigt wird.

\[
\begin{eqnarray}
    f(x, 2f) &=&
        \begin{cases}
                \frac{x^{f-1}}{2^f}
                    \frac{\exp[-\frac{x}{2}]}{\Gamma(f)}
                    & : x\ge 0 \\
                0 & : x < 0
        \end{cases} \\
    \Gamma(x) &=& \int_0^\infty t^{x-1}\cdot \mathrm e^t \,\mathrm dt
\end{eqnarray}
\]

Die Verteilungsfunktion (CDF\footnote{\emph{cumulative distribution
  function}}) \(F(x, f)\) ist dabei komplex und hat den Erwartungswert
\(\expval{\chi_f^2}=f\) und die Varianz \(\sigma_{\chi^2}=2f\).

\[
\begin{align}
    F(x, 2f) &=
        \int_0^x
            \frac{y^{f-1}}{2^f}
                \frac{\exp[-\frac{y}{2}]}{\Gamma(f)}
            \,\mathrm dy \\
    \expval{\chi_f^2} &=
        \int_0^\infty x\cdot f(x, f)
            \,\mathrm dx
        &&= f \\
    \sigma_{\chi^2} &=
        \int_0^\infty \left(x - \expval{\chi_f^2}\right)^2 \cdot f(x, f)
            \,\mathrm dx
         &&=2f
\end{align}
\]

\hypertarget{versuchsidee}{%
\subsection{Versuchsidee}\label{versuchsidee}}

In diesem Versuch wird \(\ce{^{137}Cs}\) als radioaktive Probe
verwendet, das eine Halbwertszeit \(t_\frac{1}{2}\approx 30\,\mathrm a\)
hat.

Damit soll die folgende Hypothese getestet, die Präparatstärke sei
konstant und habe den Wert \(\bar n\). Hierbei ist \(\bar n\) der
Mittelwert von vielen Einzelmessungen \(n_i\) über eine kurze Zeit von
\(\Delta t=20\,\mathrm s\), der durch Gleichung \(\eqref{mittlereRate}\)
bestimmt wird. All diese \(N\) Messungen werden in einem Zeitraum von
wenigen Stunden absolviert.

Da der Zeitraum der Messungen sehr kurz gegen die Halbwertszeit ist,
kann man annehmen, dass die Stärke der Probe sich im Rahmen der
Messungenauigkeit nicht verändert.

Damit können die Differenzen zum Mittelwert \((n_i-\bar n)\) ermittelt
werden. Nach dem zentralen Grenzwertsatz sind die relativen Differenzen
standardnormalverteilt. Dadurch kann die Abweichung \(\chi^2\) wie folgt
ermittelt werden.

\[
\begin{eqnarray}
    \bar n &=& \sum_{i=1}^N \frac{n_i}{N} \label{mittlereRate} \\
    \chi^2 &=& \sum_{i=1}^N \frac{(n_i-\bar n)^2}{\bar n} \label{ChiSquared}
\end{eqnarray}
\]

\hypertarget{einfluss-der-totzeit}{%
\subsubsection{Einfluss der Totzeit}\label{einfluss-der-totzeit}}

Die Totzeit der Länge \(\tau\) hat einen Einfluss auf die gemessenen
Zählraten. Anstatt einer Zählrate von \(\frac{n_i}{\Delta t}\) wird eine
totzeitkorrigierte Anzahl \(k_i\) gemessen. Dadurch kann ein
korrigierter Mittelwert \(M\) nach \(\eqref{mittlereRate}\) bestimmt
werden und man erhält eine korrigierte Abweichung
\(\chi^2_\mathrm{korr}\). Die korrigierte Rate wird nach Gleichung
\((??)\) bestimmt.

\[
\begin{eqnarray}
    k_i &=&
        \frac{
            n_i
        }{
            1 - \frac{m}{\Delta t}\tau
        } \label{korrRate} \\
    M &=& \sum_{i=1}^N \frac{k_i}{N} \label{korrAVG} \\
    \chi^2_\mathrm{korr} &=& \sum_{i=1}^N
        \frac{(k_i - M)^2}{M} \label{ChiKorrDef}
\end{eqnarray}
\]

Durch Einsetzen der Gleichungen \(\eqref{korrRate}\)-\(\eqref{korrAVG}\)
sowie \(\eqref{ChiSquared}\) kann man \(\eqref{ChiKorrDef}\)
vereinfachen und man erhält die folgende vereinfachte Relation. Hier
sieht man, dass die korrigierte Abweichung \(\chi^2_\mathrm{korr}\)
kleiner als die nicht--korrigierte Abweichung \(\chi^2\) ist, was
kontraintuitiv wahrgenommen werden kann.

\[
\begin{eqnarray}
    \chi^2_\mathrm{korr} &=&
        \frac{1}{1 - \frac{m}{\Delta t}\tau} \cdot \chi^2
\end{eqnarray}
\]

\clearpage
\hypertarget{durchfuxfchrung}{%
\section{Durchführung}\label{durchfuxfchrung}}

\clearpage
\hypertarget{auswertung}{%
\section{Auswertung}\label{auswertung}}

\hypertarget{poissonverteilung}{%
\subsection{Poissonverteilung}\label{poissonverteilung}}

\hypertarget{gauuxdfverteilung}{%
\subsection{Gaußverteilung}\label{gauuxdfverteilung}}

\hypertarget{intervallverteilung}{%
\subsection{Intervallverteilung}\label{intervallverteilung}}

\hypertarget{chi2test}{%
\subsection{\texorpdfstring{\(\chi^2\)--Test}{\textbackslash chi\^{}2--Test}}\label{chi2test}}

Es werden \(51\) zufällige Messergebnisse gewählt, aus denen \(\chi^2\)
gewählt wird. Da der Mittelwert einen statistischen Freiheitsgrad
bindet, bleiben \(50\) Freiheitsgrade übrig, um die
Gültigkeitsintervalle zu bilden.

\[
\begin{eqnarray}
    \chi^2_\mathrm{min} &=& 32.357 \\
    \chi^2_\mathrm{max} &=& 71.420
\end{eqnarray}
\]

\hypertarget{hypothese-h_1}{%
\subsubsection{\texorpdfstring{Hypothese
\(H_1\)}{Hypothese H\_1}}\label{hypothese-h_1}}

Nach \(\eqref{ChiSquared}\):

\[
\begin{eqnarray}
    \bar x &=& \bar n \\
    \chi^2_1 &=& \sum_i \frac{(x_i-\bar x)^2}{\bar x}
\end{eqnarray}
\]

\hypertarget{hypothese-h_2}{%
\subsubsection{\texorpdfstring{Hypothese
\(H_2\)}{Hypothese H\_2}}\label{hypothese-h_2}}

\[
\begin{eqnarray}
    \bar x^\prime &=& 0.9\cdot\bar n \\
    \chi^2_2 &=& \sum_i \frac{(x_i-\bar x^\prime)^2}{\bar x^\prime}
\end{eqnarray}
\]

\hypertarget{hypothese-h_3}{%
\subsubsection{\texorpdfstring{Hypothese
\(H_3\)}{Hypothese H\_3}}\label{hypothese-h_3}}

Mit \(i\in[0, N]\):

\[
\begin{eqnarray}
    \expval{x(i)} &=& \bar n - i \\
    \chi^2_2 &=& \sum_i \frac{(x_i-(n - i))^2}{(n - i)}
\end{eqnarray}
\]

\hypertarget{totzeit}{%
\subsection{Totzeit}\label{totzeit}}

\clearpage
\hypertarget{fazit}{%
\section{Fazit}\label{fazit}}

\clearpage
\hypertarget{literatur}{%
\section{Literatur}\label{literatur}}

\begin{enumerate}
\def\labelenumi{\arabic{enumi}.}
\tightlist
\item
  Universität zu Köln, ``B3.1: Statistik der Kernzerfälle'', Januar
  2021, Online verfügbar unter
  \url{https://www.ikp.uni-koeln.de/fileadmin/data/praktikum/B3.1_statistik_de.pdf}
\item
  Wikipedia, ``Statistischer Test'',
  \url{https://de.wikipedia.org/wiki/Statistischer_Test}, Abruf am
  18.04.2024
\item
  Wikipedia, ``Fehler 1. und 2. Art'',
  \url{https://de.wikipedia.org/wiki/Fehler_1._und_2._Art}, Abruf am
  18.04.2024
\item
  Wikipedia, ``Statistische Signifikanz'',
  \url{https://de.wikipedia.org/wiki/Statistische_Signifikanz}, Abruf am
  18.04.2024
\item
  G. Gigerenzer, ``Mindless statistics'', 2004, \emph{The Journal of
  Socio--Economics}, p.587-606, DOI
  \href{https://doi.org/10.1016/j.socec.2004.09.033}{0.1016/j.socec.2004.09.033}
\item
  K. C. Kapur \& M. Pecht, ``Reliability Engineering'': Appendix E,
  Wiley 2014, DOI
  \href{https://doi.org/10.1002/9781118841716}{10.1002/9781118841716}
\item
  E. Cramer \& U. Kamps, ``Grundlagen der Wahrscheinlichkeitsrechnung
  und Statistik'', Springer 2020, DOI~\href{https://doi.org/10.1007/978-3-662-60552-3}{10.1007/978-3-662-60552-3}
\item
  J. Puhani, ``Statistik'', Springer 2020, DOI~\href{https://doi.org/10.1007/978-3-658-28955-3}{10.1007/978-3-658-28955-3}
\end{enumerate}

\end{document}
