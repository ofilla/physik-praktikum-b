% based on a template made by the university of cologne
% http://www.mi.uni-koeln.de/wp-MIEDV/wp-content/uploads/2016/07/LaTeX-Vorlage.zip - 2023-11-02
\documentclass[12pt,a4paper]{scrartcl}

\addtokomafont{sectioning}{\rmfamily}
\usepackage[ngerman]{babel}% deutsches Sprachpaket wird geladen
\usepackage[T1]{fontenc} % westeuropäische Codierung wird verlangt
\usepackage[utf8]{inputenc}% Umlaute werden erlaubt
\usepackage[usenames]{color} % Erlaubt die Benutzung der namen im Farbpaket und deren Änderung
\usepackage{amsmath} % Erweiterung für den Mathe-Satz
\usepackage{amssymb} % alle Zeichen aus msam und msmb werden dargestellt
\usepackage{graphicx} % Graphiken und Bilder können eingebunden werden
%\usepackage{multirow} % erlaubt in einer Spalte einer Tabelle die Felder in mehreren Zeilen zusammenzufassen
\usepackage{enumerate} % erlaubt Nummerierungen
\usepackage{xurl} % Dient zur Auszeichnung von URLs; setzt die Adresse in Schreibmaschinenschrift.
\usepackage[center]{caption}  % Bildunterschrift wird zentriert
%\usepackage{subfigure} % mehrere Bilder können in einer fugure-Umgebung verwendet werden
%\usepackage{longtable} % Diese Umgebung ist ähnlich definiert wie die tabular-Umgebung, erlaubt jedoch mehrseitige Tabellen.
%\usepackage{paralist} % Modifikation der bereits bestehenden Listenumgebungen
\usepackage{lmodern}% Für die Schrift
\usepackage[hidelinks]{hyperref} % Links und Verweise werden innerhalb von PDF Dokumenten erzeugt
%\usepackage{wrapfig} % Das Paket ermöglicht es von Schrift umflossene Bilder und Tabellen einzufügen.
\usepackage{latexsym} % LaTeX-Symbole werden geladen
\usepackage{tikz} % Erlaubt es mit tikz zu zeichnen
\usepackage{tabularx} % Erlaubt Tabellen
\usepackage{algorithm} % Erlaubt Pseudocode
\usepackage{color} % Farbpaket wird geladen
%\usepackage{stmaryrd} % St Mary Road Symbole werden geladen
\usepackage{physics}
\usepackage{mhchem} % Chemie: \ce & \pu

\numberwithin{equation}{section} % Nummerierungen der Gleichungen, die durch equation erstellt werden, sind gebunden an die section
\newcommand{\HRule}{\rule{\linewidth}{0.7mm}}
\newcommand{\pu}[1]{\ensuremath{\mathrm{#1}}}

% disable commands
\renewcommand{\[}{} % math block start
\renewcommand{\]}{\noindent} % math block end
\newcommand{\tightlist}{} % created in enumerations

\hypersetup{
  pdftitle={B3.1},
  pdfcreator={LaTeX via pandoc}}

\setcounter{secnumdepth}{6}
\setcounter{tocdepth}{6}

\begin{document}
\begin{titlepage}
	\pagestyle{empty}

	\begin{center}

	\textsc{\LARGE Universität zu Köln }\\ [0.4cm]
	\textsc{Mathematisch-Naturwissenschaftliche Fakultät} \\[1.5cm]

	\includegraphics[width=0.45\textwidth]{../media/uni.jpg}\\[1.5cm]  % Uni-Logo wird geladen

	\textsc{\Large Praktikum~B}\\[2mm]
	\textsc{}\\[10mm]
	\HRule \\[0.4cm]

		{	\Huge \bfseries B3.1}\\[0.4cm]
			{	\huge \bfseries Statistik der Kernzerfälle}\\[0.3cm]
	
	\HRule \\[3cm]

 	\begin{center}
		\textsc{\Large Catherine~Tran } \\[3pt]
		\textsc{\Large Carlo~Kleefisch } \\[3pt]
		\textsc{\Large Oliver~Filla } \\[3pt]
	\end{center}
	\end{center}
\end{titlepage}

\newpage
\tableofcontents
\newpage

\hypertarget{einleitung}{%
\section{Einleitung}\label{einleitung}}

\clearpage
\hypertarget{theoretische-grundlagen}{%
\section{Theoretische Grundlagen}\label{theoretische-grundlagen}}

\hypertarget{der-chi2anpassungstest}{%
\subsection{\texorpdfstring{Der
\(\chi^2\)--Anpassungstest}{Der \textbackslash chi\^{}2--Anpassungstest}}\label{der-chi2anpassungstest}}

Der \(\chi^2\)--Anpassungstest (Chi--Quadrat--Test) dient dazu, eine
Verteilung von Zufallsvariablen mit einer theoretischen Verteilung zu
vergleichen. Man kann mithilfe des Tests bewerten, ob die
Zufallsvariablen der Verteilung entsprechen können.

\hypertarget{hypothesentest}{%
\subsubsection{Hypothesentest}\label{hypothesentest}}

Ein Hypothesentest oder Statistischer Test dient dazu, durch eine
Hypothese mittels statistischer Messungen zu prüfen.

Dazu verwendet man eine \emph{Nullhypothese}\footnote{\emph{Hypothesis
  to be nullified} \([5]\)} \(H_0\) und eine \emph{Gegenhypothese} oder
\emph{Alternativhypothese} \(H_1\), die sich unterscheiden. Ziel des
Tests ist es, die Alternativhypothese \(H_1\) zu belegen. Falls dies
nicht gelingt, muss man die Nullhypothese \(H_0\) als wahr annehmen.
Diese wird nicht überprüft. \([2]\)

Aufgrund der Zufälligkeit der Ereignisse kann es dabei zwei Arten von
Fehlern geben. Ein \(\alpha\)--Fehler beschreibt das irrtümliche
Ablehnen von \(H_0\), während ein \(\beta\)--Fehler das fälschliche
Annehmen von \(H_0\) bezeichnet.

\hypertarget{fehlerarten}{%
\subsubsection{Fehlerarten}\label{fehlerarten}}

Ein \emph{Fehler erster Art} oder \(\alpha\)--Fehler beschreibt die
fälschliche Ablehnung der Nullhypothese \(H_0\) in einem Statistischen
Test. Man nimmt z.B. an, dass ein Würfel gezinkt ist \((H_1)\), obwohl
er in Wahrheit fair ist \((H_0)\). Hierbei ist die \(H_0\) die Annahme
eines fairen Würfels. Man spricht hier auch von einem
\emph{falsch--positiven} Ergebnis. \([3]\)

Ein \emph{Fehler zweiter Art} oder \emph{\(\beta\)--Fehler} beschreibt
umgekehrt die fälschliche Akzeptanz der Nullhypothese \(H_0\).
Beispielsweise geht man davon aus, dass ein Würfel fair ist \((H_0)\),
obwohl er tatsächlich unfair ist \((H_1)\). Man spricht hier auch von
einem \emph{falsch--negativen} Ergebnis. \([3]\)

Die statistische Signifikanz beschreibt die erlaubte Wahrscheinlichkeit,
einen \(\alpha\)--Fehler zu begehen. \([4]\)

\clearpage
\hypertarget{durchfuxfchrung}{%
\section{Durchführung}\label{durchfuxfchrung}}

\clearpage
\hypertarget{auswertung}{%
\section{Auswertung}\label{auswertung}}

\hypertarget{fazit}{%
\clearpage
\section{Fazit}\label{fazit}}

\clearpage
\hypertarget{literatur}{%
\section{Literatur}\label{literatur}}

\begin{enumerate}
\def\labelenumi{\arabic{enumi}.}
\tightlist
\item
  Universität zu Köln, ``B3.1: Statistik der Kernzerfälle'', Januar
  2021, Online verfügbar unter
  \url{https://www.ikp.uni-koeln.de/fileadmin/data/praktikum/B3.1_statistik_de.pdf}
\item
  Wikipedia, ``Statistischer Test'',
  \url{https://de.wikipedia.org/wiki/Statistischer_Test}, Abruf am
  18.04.2024
\item
  Wikipedia, ``Fehler 1. und 2. Art'',
  \url{https://de.wikipedia.org/wiki/Fehler_1._und_2._Art}, Abruf am
  18.04.2024
\item
  Wikipedia, ``Statistische Signifikanz'',
  \url{https://de.wikipedia.org/wiki/Statistische_Signifikanz}, Abruf am
  18.04.2024
\item
  G. Gigerenzer, ``Mindless statistics'', 2004, \emph{The  Journal of Socio--Economics},\\
  DOI~\href{https://doi.org/10.1016/j.socec.2004.09.033}{0.1016/j.socec.2004.09.033}
\end{enumerate}

\end{document}
