% based on a template made by the university of cologne
% http://www.mi.uni-koeln.de/wp-MIEDV/wp-content/uploads/2016/07/LaTeX-Vorlage.zip - 2023-11-02
\documentclass[12pt,a4paper]{scrartcl}

\addtokomafont{sectioning}{\rmfamily}
\usepackage[ngerman]{babel}% deutsches Sprachpaket wird geladen
\usepackage[T1]{fontenc} % westeuropäische Codierung wird verlangt
\usepackage[utf8]{inputenc}% Umlaute werden erlaubt
\usepackage[usenames]{color} % Erlaubt die Benutzung der namen im Farbpaket und deren Änderung
\usepackage{amsmath} % Erweiterung für den Mathe-Satz
\usepackage{amssymb} % alle Zeichen aus msam und msmb werden dargestellt
\usepackage{graphicx} % Graphiken und Bilder können eingebunden werden
%\usepackage{multirow} % erlaubt in einer Spalte einer Tabelle die Felder in mehreren Zeilen zusammenzufassen
\usepackage{enumerate} % erlaubt Nummerierungen
\usepackage{xurl} % Dient zur Auszeichnung von URLs; setzt die Adresse in Schreibmaschinenschrift.
\usepackage[center]{caption}  % Bildunterschrift wird zentriert
%\usepackage{subfigure} % mehrere Bilder können in einer fugure-Umgebung verwendet werden
%\usepackage{longtable} % Diese Umgebung ist ähnlich definiert wie die tabular-Umgebung, erlaubt jedoch mehrseitige Tabellen.
%\usepackage{paralist} % Modifikation der bereits bestehenden Listenumgebungen
\usepackage{lmodern}% Für die Schrift
\usepackage[hidelinks]{hyperref} % Links und Verweise werden innerhalb von PDF Dokumenten erzeugt
%\usepackage{wrapfig} % Das Paket ermöglicht es von Schrift umflossene Bilder und Tabellen einzufügen.
\usepackage{latexsym} % LaTeX-Symbole werden geladen
\usepackage{tikz} % Erlaubt es mit tikz zu zeichnen
\usepackage{tabularx} % Erlaubt Tabellen
\usepackage{algorithm} % Erlaubt Pseudocode
\usepackage{color} % Farbpaket wird geladen
%\usepackage{stmaryrd} % St Mary Road Symbole werden geladen
\usepackage{physics}
\usepackage{mhchem} % Chemie: \ce & \pu

\numberwithin{equation}{section} % Nummerierungen der Gleichungen, die durch equation erstellt werden, sind gebunden an die section
\newcommand{\HRule}{\rule{\linewidth}{0.7mm}}
\newcommand{\pu}[1]{\ensuremath{\mathrm{#1}}}

% disable commands
\renewcommand{\[}{} % math block start
\renewcommand{\]}{\noindent} % math block end

\hypersetup{
  pdftitle={B1.1},
  pdfcreator={LaTeX via pandoc}}

\setcounter{secnumdepth}{6}
\setcounter{tocdepth}{6}

\begin{document}
\begin{titlepage}
	\pagestyle{empty}

	\begin{center}

	\textsc{\LARGE Universität zu Köln }\\ [0.4cm]
	\textsc{Mathematisch-Naturwissenschaftliche Fakultät} \\[1.5cm]

	\includegraphics[width=0.45\textwidth]{../media/uni.jpg}\\[1.5cm]  % Uni-Logo wird geladen

	\textsc{\Large Praktikum~B}\\[2mm]
	\textsc{}\\[10mm]
	\HRule \\[0.4cm]

		{	\Huge \bfseries B1.1}\\[0.4cm]
			{	\huge \bfseries Infrarotabsorption in \(\ce{CO_2}\)}\\[0.3cm]
	
	\HRule \\[3cm]

 	\begin{center}
		\textsc{\Large Catherine~Tran } \\[3pt]
		\textsc{\Large Carlo~Kleefisch } \\[3pt]
		\textsc{\Large Oliver~Filla } \\[3pt]
	\end{center}
	\end{center}
\end{titlepage}

\newpage
\tableofcontents
\newpage

\hypertarget{einleitung}{%
\section{Einleitung}\label{einleitung}}

Mithilfe von Spektroskopie ist es möglich, Absorptions- und
Emissionsspektren von elektromagnetischen Wellen an einer Probe zu
beobachten. Hierzu misst man spezifische Größen in Abhängigkeit von der
Frequenz. Dadurch ist es möglich, Aussagen über die mikroskopischen
Eigenschaften der Probe zu treffen.

Alternativ zur Frequenz werden auch äquivalente Größen wie der
Wellenlänge oder Energie verwendet, um Größen wie die Intensität, die
Strahlungsleistung und die Zählrate der Strahlung zu messen.

Dieser Versuch dient als Einführung in die Infrarotspektroskopie.
Mithilfe eines niedrig auflösenden Absorptionsspektrometers wird die
Infrarot--Absorption von Kohlenstoffdioxid \((\ce{CO_2})\) untersucht.
Dieses Gas gehört zu den sogenannten ``Treibhausgasen''.

\clearpage
\hypertarget{theoretische-grundlagen}{%
\section{Theoretische Grundlagen}\label{theoretische-grundlagen}}

\hypertarget{elektromagnetisches-spektrum}{%
\subsection{elektromagnetisches
Spektrum}\label{elektromagnetisches-spektrum}}

\hypertarget{planckstrahlung}{%
\subsection{Planck--Strahlung}\label{planckstrahlung}}

Das Planck'sche Strahlungsgesetz beschreibt die Energiedichte \(\omega_\mathrm{P}\), die ein schwarzer Körper mit einer Frequenz \(\nu\) bei einer Temperatur \(T\) als Wärmestrahlung aussendet. Dabei finden das Planck'sche Wirkungsquantum \(h\), die Lichtgeschwindigkeit \(c\) und die Boltzmann--Konstante \(k_B\) Verwendung. \cite{Demtröder}

\[
\begin{eqnarray}
    \omega_\mathrm{P}(\nu,T) &=&
        \frac{8\pi\nu^2}{c^3}
        \frac{h\nu}{\exp\left[\frac{h\nu}{k_BT}\right]-1}
        \,\mathrm d\nu
        \label{eq:PlackStrahlung}
\end{eqnarray}
\]

\noindent
Der Faktor \(\frac{8\pi\nu^2}{c^3}\) ist dabei die Dichte der Schwingungsmoden in einem Frequenzintervall, also die Anzahl erlaubter
Schwingungszustände. Der Faktor \(h\nu\cdot\exp[\dots]^{-1}\) beschreibt die mittlere kinetische Energie dieser Zustände. Dies ist in Abbildung \ref{abb:SpektrumSK} dargestellt.

\begin{figure}[h!]
	\centering
	\includegraphics[width=0.5\textwidth]{../media/B1.1/BlackbodySpectrum_lin_150dpi_de.png}
	\caption{Schwarzkörperstrahlung für verschiedene Temperaturen \cite{abb:SpektrumSK}}
	\label{abb:SpektrumSK}
\end{figure}

\hypertarget{strahlungsdichte}{%
\subsubsection{Strahlungsdichte}\label{strahlungsdichte}}

Die Strahldichte oder Strahlungsdichte \(S\) beschreibt die Strahlung, die ein Flächenelement \(\mathrm dA\) eines Strahlers in einen Raumwinkel \(\mathrm d\Omega\) abstrahlt. Sie ist allgemein das Differential der Strahlungsleistung \(\Phi\). \cite{Demtröder,Strahldichte}

\[
\begin{eqnarray}
    S &=& \frac{\mathrm d^2\Phi}{\mathrm dA \cdot \mathrm d\Omega}
\end{eqnarray}
\]

\hypertarget{wiensches-verschiebungsgesetz}{%
\subsubsection{Wien'sches
Verschiebungsgesetz}\label{wiensches-verschiebungsgesetz}}

Das Wien'sche Verschiebungsgesetz beschreibt abhängig von der Temperatur \(T\), bei welcher Wellenlänge \(\hat{\lambda}\) bzw. Frequenz \(\hat{\nu}\) die größte Wärmeleistung abgestrahlt wird. Dadurch beschreibt es die Temperaturabhängigkeit des Maximums des Planck'schen Strahlungsgesetzes \(\eqref{eq:PlackStrahlung}\).

\[
\begin{eqnarray}
    \hat{\lambda}\cdot T &=& 2.898\cdot10^{-3}\mathrm{\,m\cdot K}
        \label{eq:WienLambda}\\
    \hat{\nu}\cdot T &=& 5.879\cdot10^{10} \mathrm{\,m\cdot Hz}
        \label{eq:WienNu}
\end{eqnarray}
\]

\hypertarget{dipole}{%
\subsection{Dipole}\label{dipole}}

\hypertarget{freiheitsgrade}{%
\subsection{Freiheitsgrade}\label{freiheitsgrade}}

\hypertarget{infrarotaktivituxe4t}{%
\subsection{Infrarotaktivität}\label{infrarotaktivituxe4t}}

\hypertarget{photonen}{%
\subsection{Photonen}\label{photonen}}

\hypertarget{mathematische-grundlagen}{%
\subsection{Mathematische Grundlagen}\label{mathematische-grundlagen}}

\hypertarget{das-lambertbeersche-gesetz}{%
\subsection{Das Lambert--Beer'sche
Gesetz}\label{das-lambertbeersche-gesetz}}

\hypertarget{treibhauseffekt}{%
\subsection{Treibhauseffekt}\label{treibhauseffekt}}

\clearpage
\hypertarget{durchfuxfchrung}{%
\section{Durchführung}\label{durchfuxfchrung}}

\clearpage
\hypertarget{auswertung}{%
\section{Auswertung}\label{auswertung}}

\clearpage
\hypertarget{fazit}{%
\section{Fazit}\label{fazit}}

\clearpage
\hypertarget{literatur}{%
\section{Literatur}\label{literatur}}
\renewcommand{\section}[2]{} % remove extra title
\begin{thebibliography}{9}
\bibitem{UzK}
	Universität zu Köln, ``B1.1: Infrarotabsorption in \(\ce{CO_2}\)'', April 2024
\bibitem{Bakan}
  S. Bakan \& E. Raschke, ``Der natürliche Treibhauseffekt'', Promet 28,
	Deutscher Wetterdienst, 2002, Online verfügbar unter
	\url{https://www.dwd.de/DE/leistungen/pbfb_verlag_promet/pdf_promethefte/28_3_4_pdf.pdf}
\bibitem{Demtröder}
	W. Demtröder, ``Experimentalphysik 3: Atome, Moleküle und Festkörper'',
	Springer--Spektrum--Verlag, 5. Auflage 2016, DOI:
	\href{https://doi.org/10.1007/978-3-662-49094-5}{10.1007/978-3-662-49094-5}
\bibitem{Strahldichte}
	Wikipedia, ``Strahldichte'',
	\url{https://de.wikipedia.org/wiki/Strahldichte}, Abruf am 02.05.2024
\bibitem{abb:SpektrumSK}
	Wikipedia, \url{https://commons.wikimedia.org/wiki/File:BlackbodySpectrum_lin_150dpi_de.png},
	Abruf am 02.05.2024
\end{thebibliography}
\end{document}
