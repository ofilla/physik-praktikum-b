% based on a template made by the university of cologne
% http://www.mi.uni-koeln.de/wp-MIEDV/wp-content/uploads/2016/07/LaTeX-Vorlage.zip - 2023-11-02
\documentclass[12pt,a4paper]{scrartcl}

\addtokomafont{sectioning}{\rmfamily}
\usepackage[ngerman]{babel}% deutsches Sprachpaket wird geladen
\usepackage[T1]{fontenc} % westeuropäische Codierung wird verlangt
\usepackage[utf8]{inputenc}% Umlaute werden erlaubt
\usepackage[usenames]{color} % Erlaubt die Benutzung der namen im Farbpaket und deren Änderung
\usepackage{amsmath} % Erweiterung für den Mathe-Satz
\usepackage{amssymb} % alle Zeichen aus msam und msmb werden dargestellt
\usepackage{graphicx} % Graphiken und Bilder können eingebunden werden
%\usepackage{multirow} % erlaubt in einer Spalte einer Tabelle die Felder in mehreren Zeilen zusammenzufassen
\usepackage{enumerate} % erlaubt Nummerierungen
\usepackage{xurl} % Dient zur Auszeichnung von URLs; setzt die Adresse in Schreibmaschinenschrift.
\usepackage[center]{caption}  % Bildunterschrift wird zentriert
%\usepackage{subfigure} % mehrere Bilder können in einer fugure-Umgebung verwendet werden
%\usepackage{longtable} % Diese Umgebung ist ähnlich definiert wie die tabular-Umgebung, erlaubt jedoch mehrseitige Tabellen.
%\usepackage{paralist} % Modifikation der bereits bestehenden Listenumgebungen
\usepackage{lmodern}% Für die Schrift
\usepackage[hidelinks]{hyperref} % Links und Verweise werden innerhalb von PDF Dokumenten erzeugt
%\usepackage{wrapfig} % Das Paket ermöglicht es von Schrift umflossene Bilder und Tabellen einzufügen.
\usepackage{latexsym} % LaTeX-Symbole werden geladen
\usepackage{tikz} % Erlaubt es mit tikz zu zeichnen
\usepackage{tabularx} % Erlaubt Tabellen
\usepackage{algorithm} % Erlaubt Pseudocode
\usepackage{color} % Farbpaket wird geladen
%\usepackage{stmaryrd} % St Mary Road Symbole werden geladen
\usepackage{physics}
\usepackage[version=4]{mhchem} % Chemie: \ce & \pu

\numberwithin{equation}{section} % Nummerierungen der Gleichungen, die durch equation erstellt werden, sind gebunden an die section
\newcommand{\HRule}{\rule{\linewidth}{0.7mm}}

\hypersetup{
  pdftitle={B1.5},
  pdfcreator={LaTeX via pandoc}}

\setcounter{secnumdepth}{6}
\setcounter{tocdepth}{6}

\begin{document}
\begin{titlepage}
	\pagestyle{empty}

	\begin{center}

	\textsc{\LARGE Universität zu Köln }\\ [0.4cm]
	\textsc{Mathematisch-Naturwissenschaftliche Fakultät} \\[1.5cm]

	\includegraphics[width=0.45\textwidth]{../media/uni.jpg}  % Uni-Logo wird geladen

	\textsc{\Large Praktikum~B}\\[2mm]
	\textsc{}\\[10mm]
	\HRule \\[0.4cm]

		{	\Huge \bfseries B1.4}\\[0.4cm]
			{	\huge \bfseries Photoelektrischer Effekt}\\[0.3cm]
	
	\HRule \\[3cm]

 	\begin{center}
		\textsc{\Large Catherine~Tran } \\[3pt]
		\textsc{\Large Carlo~Kleefisch } \\[3pt]
		\textsc{\Large Oliver~Filla } \\[3pt]
	\end{center}
	\end{center}
\end{titlepage}

\newpage
\tableofcontents
\newpage

\clearpage
\hypertarget{einleitung}{%
\section{Einleitung}\label{einleitung}}

\clearpage
\hypertarget{theoretische-grundlagen}{%
\section{Theoretische Grundlagen}\label{theoretische-grundlagen}}
\subsection{Elektrisches Feld und Spannung}
\subsection{Funktionsweise einer Photozelle}
\subsection{Stromfreie Spannungsmessung}
\subsection{Transmissionsgrad, Farbfilter, Graufilter}
\subsection{Austrittsarbeit}
\subsection{Kontaktspannung}
\subsection{Äußerer Photoeffekt, kinetische Energie}

\clearpage
\hypertarget{durchfuxfchrung}{%
\section{Durchführung}\label{durchfuxfchrung}}
Zunächst wird das System genullt, d.h. es wird so eingestellt, dass die Gegenspannung $U$ bei Kurzschluss der Photozelle $0\mathrm{\,V}$ ist.

\subsection{Bestimmung von $h/e$}
Der Widerstand des Elektrometers wurde auf $10^{13}\,\Omega$ eingestellt.

\begin{table}[h!]
	\centering
	\begin{tabular}{c|c|c}
		Filter & Farbe & $\lambda$ $[\mathrm{nm}]$ \\
		\hline
		$1$ & UV & $366$ \\
		$2$ & Violett & $405$ \\
		$3$ & Blau & $436$ \\
		$4$ & Grün & $546$ \\
		$5$ & Gelb & $578$ \\
	\end{tabular}
	\caption{Eigenschaften der Interferenzfilter mit $\Delta \lambda = \pm 7\mathrm{nm}$}
	\label{tab:Interferenzfilter}
\end{table}

\subsubsection{Gegenspannungsmethode}
Die Zeitkonstante wurde auf $0.3\mathrm{\,s}$ und der Messverstärker auf $10$ eingestellt.

\begin{table}[h!]
	\centering
	\begin{tabular}{c|c|c}
		Filter & Farbe & $U_0$ $[V]$ \\
		\hline
		$1$ & UV & $1.769$ \\
		$2$ & Violett & $1.434$ \\
		$3$ & Blau & $1.232$ \\
		$4$ & Grün & $0.705$ \\
		$5$ & Gelb & $0.645$ \\
	\end{tabular}
	\caption{Messergebnisse nach der Gegenspannungsmethode mit $\Delta U_0=\pm 1\mathrm{\,mV}$}
	\label{tab:Gegenspannungsmethode}
\end{table}

\subsubsection{direkte Messmethode}
Schon während der Messung fiel die große Ähnlichkeit der Messwerte mit denen aus der vorherigen Messung auf. Dass die Werte eher minimal kleiner sind liegt vermutlich an einem reduzierten Rauschen aus der Umgebung, da die nicht--abgeschirmten Kabel des Aufbaus noch weiter von der Photozelle entfernt waren.

\begin{table}[h!]
	\centering
	\begin{tabular}{c|c|c}
		Filter & Farbe & $U$ $[V]$ \\
		\hline
		$1$ & UV & $1.770$ \\
		$2$ & Violett & $1.434$ \\
		$3$ & Blau & $1.232$ \\
		$4$ & Grün & $0.696$ \\
		$5$ & Gelb & $0.635$ \\
	\end{tabular}
	\caption{Messergebnisse nach der direkten Methode mit $\Delta U_0=\pm 1\mathrm{\,mV}$}
	\label{tab:direkten Methode}
\end{table}

\subsection{Photostrom}
\begin{table}[h!]
	\centering
	\begin{tabular}{c|c|c|c|c|c|c}
		$\lambda$ $[\mathrm{nm}]$ & \multicolumn{6}{c}{Transmissionsgrad $T$ $[\%]$} \\
		& $1$ & $2$ & $3$ & $4$ & $5$ & $6$ \\
		\hline
		$436$ & $68$ & $48$ & $33$ & $28$ & $20$ & $14$ \\
		$546$ & $67$ & $46$ & $31$ & $23$ & $16$ & $11$
	\end{tabular}
	\caption{Eigenschaften der Graufilter mit $\Delta T=1\,\%$}
	\label{tab:Graufilter}
\end{table}

\subsubsection{Photoströme}
Bei der Messung der Photoströme trat ein Nullpunktsfehler von $1\mathrm{\,mV}$ auf. Der Messverstärker wurde auf $10^4$ gestellt, die Zeitkonstante auf $0.3\mathrm{\,s}$. Der Photostrom wurde indirekt über einen Widerstand $R=10\mathrm{\,k\Omega}$ gemessen.

\begin{table}[h!]
	\centering
	\begin{tabular}{c|c|c|c|c|c|c|c|c}
		 	&& \multicolumn{7}{c}{Photostrom $U_\mathrm{Ph}$ $[\mathrm{V}]$} \\
		Filter & Farbe & ohne & $1$ & $2$ & $3$ & $4$ & $5$ & $6$ \\
		\hline
		$3$ & Blau & $1.777$ & $1.486$ & $1.067$ & $0.807$ & $0.655$ & $0.440$ & $0.338$ \\
		$4$ & Grün & $0.608$ & $0.536$ & $0.357$ & $0.265$ & $0.190$ & $0.126$ & $0.090$
	\end{tabular}
	\caption{Messungen der Photoströme über mit $\Delta U_\mathrm{Ph}=1\mathrm{\,mV}$}
	\label{tab:Photostrom}
\end{table}

\subsubsection{Intensität}
Ohne den Graufilter wurde eine Photospannung von $U_0=0.688\pm0.001\mathrm{\,V}$ gemessen, mit Graufilter $6$ eine von $U_0^\prime=0.693\pm0.001\mathrm{\,V}$. Aufgrund dieser Abweichungen sollte der Fehler auf zumindest $\pm 3\mathrm{\,mV}$ erhöht werden.

\subsection{Untersuchung von LEDs mit der Photozelle}
\begin{table}[h!]
	\centering
	\begin{tabular}{l|c}
		Bezeichnung & $U_0$ $[\mathrm{V}]$ \\
		\hline
		blue & $1.047$ \\
		verde & $0.891$ \\
		true green & $0.807$ \\
	\end{tabular}
	\caption{Messungen der LEDs mit $\Delta U_0=1\mathrm{\,mV}$}
	\label{tab:LEDs}
\end{table}


\clearpage
\hypertarget{auswertung}{%
\section{Auswertung}\label{auswertung}}

\clearpage
\hypertarget{fazit}{%
\section{Fazit}\label{fazit}}

\clearpage
\hypertarget{literatur}{%
\section{Literatur}\label{literatur}}
\renewcommand{\section}[2]{}

\begin{thebibliography}{99}
\bibitem{Demtröder}
	W. Demtröder, ``Experimentalphysik 3'', $5.$ Auflage, Springer Verlag,
	ISBN 000000000
\bibitem{Gerthsen}
	D. Meschede, ``Gerthsen Physik'', $21.$ Auflage, Springer Verlag,
	ISBN 000000000
\bibitem{uni}
		Universität zu Köln, ``B1.4: Photoeffekt: Bestimmung von $h/e$'', Juli 2008,
		Online verfügbar unter \url{https://teaching.astro.uni-koeln.de/sites/default/files/praktikum_b/Anleitung_1.4.pdf},
		Abruf am 10.04.2024
\end{thebibliography}
\end{document}
