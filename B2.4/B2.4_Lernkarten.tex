% based on a template made by the university of cologne
% http://www.mi.uni-koeln.de/wp-MIEDV/wp-content/uploads/2016/07/LaTeX-Vorlage.zip - 2023-11-02
\documentclass[12pt,a4paper]{scrartcl}

\addtokomafont{sectioning}{\rmfamily}
%\usepackage[ngerman]{babel}% deutsches Sprachpaket wird geladen
\usepackage[T1]{fontenc} % westeuropäische Codierung wird verlangt
\usepackage[utf8]{inputenc}% Umlaute werden erlaubt
\usepackage[usenames]{color} % Erlaubt die Benutzung der namen im Farbpaket und deren Änderung
\usepackage{amsmath} % Erweiterung für den Mathe-Satz
\usepackage{amssymb} % alle Zeichen aus msam und msmb werden dargestellt
\usepackage{graphicx} % Graphiken und Bilder können eingebunden werden
%\usepackage{multirow} % erlaubt in einer Spalte einer Tabelle die Felder in mehreren Zeilen zusammenzufassen
\usepackage{enumerate} % erlaubt Nummerierungen
\usepackage{url} % Dient zur Auszeichnung von URLs; setzt die Adresse in Schreibmaschinenschrift.
\usepackage[center]{caption}  % Bildunterschrift wird zentriert
%\usepackage{subfigure} % mehrere Bilder können in einer fugure-Umgebung verwendet werden
%\usepackage{longtable} % Diese Umgebung ist ähnlich definiert wie die tabular-Umgebung, erlaubt jedoch mehrseitige Tabellen.
%\usepackage{paralist} % Modifikation der bereits bestehenden Listenumgebungen
\usepackage{lmodern}% Für die Schrift
\usepackage[hidelinks]{hyperref} % Links und Verweise werden innerhalb von PDF Dokumenten erzeugt
%\usepackage{wrapfig} % Das Paket ermöglicht es von Schrift umflossene Bilder und Tabellen einzufügen.
\usepackage{latexsym} % LaTeX-Symbole werden geladen
\usepackage{tikz} % Erlaubt es mit tikz zu zeichnen
\usepackage{tabularx} % Erlaubt Tabellen
\usepackage{algorithm} % Erlaubt Pseudocode
\usepackage{color} % Farbpaket wird geladen
%\usepackage{stmaryrd} % St Mary Road Symbole werden geladen

\numberwithin{equation}{section} % Nummerierungen der Gleichungen, die durch equation erstellt werden, sind gebunden an die section
\newcommand{\HRule}{\rule{\linewidth}{0.7mm}}
\newcommand{\pu}[1]{\ensuremath{\mathrm{#1}}}

% disable commands
\renewcommand{\[}{} % math block start
\renewcommand{\]}{\noindent} % math block end
\newcommand{\tightlist}{} % created in enumerations

\hypersetup{
  pdftitle={Lernkarten zu B2.4},
  pdfcreator={LaTeX via pandoc}}

\begin{document}
\begin{titlepage}
	\pagestyle{empty}

	\begin{center}

	\textsc{\LARGE Universität zu Köln }\\ [0.4cm]
	\textsc{Mathematisch-Naturwissenschaftliche Fakultät} \\[1.5cm]

	\includegraphics[width=0.45\textwidth]{uni}\\[1.5cm]  % Uni-Logo wird geladen

	\textsc{\Large Praktikum~B}\\[2mm]
	\textsc{}\\[10mm]
	\HRule \\[0.4cm]

		{	\Huge \bfseries Lernkarten zu B2.4}\\[0.4cm]
			{	\huge \bfseries Magnetisierung eines Ferrits}\\[0.3cm]
	
	\HRule \\[3cm]

			\textsc{\Large Catherine Tran } \\[3pt]
		\textsc{\Large Carlo Kleefisch } \\[3pt]
		\textsc{\Large Oliver Filla } \\[3pt]
		
% 	\begin{center}
% 	\textsc{\Large Catherine~Tran } \\[3pt]
% 	\textsc{\Large Carlo~Kleefisch } \\[3pt]
% 	\textsc{\Large Oliver~Filla } \\[3pt]
% 	\end{center}
	\end{center}
\end{titlepage}

\newpage
\tableofcontents
\newpage

\hypertarget{theoretische-grundlagen}{%
\section{Theoretische Grundlagen}\label{theoretische-grundlagen}}

\hypertarget{grundlagen-des-magnetismus}{%
\subsection{Grundlagen des
Magnetismus}\label{grundlagen-des-magnetismus}}

\hypertarget{magnetismus}{%
\subsubsection{Magnetismus}\label{magnetismus}}

Das Phänomen des Magnetismus hängt wesentlich von der Quantenmechanik
ab, da sich das magnetische Moment eines Systems klassisch nicht
erklären lässt.\footnote{Quelle: Kittel, Einführung in die
  Festkörperphysik} Die Erzeuger des Magnetfeldes sind bewegte
elektrische Ladungen bzw. Ströme.

Die wichtigsten Größen sind die magnetische Feldstärke \(\vec H\), die
magnetische Flussdichte \(\vec B\) und die Magnetisierung \(\vec M\).

\[
\begin{eqnarray}
    \vec{B} &=& \mu_0
        \left(
            \vec H + \vec M
        \right)
\end{eqnarray}
\]

\hypertarget{magnetische-flussdichte}{%
\subsubsection{magnetische Flussdichte}\label{magnetische-flussdichte}}

element \(A\) hindurch tritt. Sie wird aus dem Vektorpotential
\(\vec A\) hergeleitet.

\[
\begin{eqnarray}
    B &=& \frac{\mathrm d\Phi}{\mathrm dA}
\end{eqnarray}
\]

Im Allgemeinen wird die magnetische Flussdichte durch die magnetische
Feldkonstante \(\mu_0\), die magnetische Feldstärke \(\vec H\) und die
Magnetisierung \(\vec M\) beschrieben.

\[
\begin{eqnarray}
    \vec B &=& \mu_0 \cdot \left(\vec H + \vec M\right)
\end{eqnarray}
\]

Für kleine Feldstärken kann man dies linear mit der Permeabilität
\(\mu\) annähern. Dies ist z.B. bei Paramagneten oder Diamagneten
anwendbar.

\[
\begin{eqnarray}
    \vec B &=& \mu \cdot \vec H
\end{eqnarray}
\]

\hypertarget{magnetisierung}{%
\subsubsection{Magnetisierung}\label{magnetisierung}}

Die Magnetisierung \(\vec M\) charakterisiert den magnetischen Zustand
eines Materials. Sie ist ein Vektorfeld, das die Dichte von (permanenten
oder induzierten) magnetischen Dipolen in einem magnetischen Material
beschreibt und berechnet sich als das magnetische Dipolmoment \(\vec m\)
pro Volumen \(V\). Weiterhin beschreibt sie die Dichte desselben.

\[
\begin{eqnarray}
    \vec M &=& \frac{\mathrm d\vec m}{\mathrm dV} \\
    \vec M &=& \frac{\vec \mu}{V}
\end{eqnarray}
\]

In einem Festkörper ist die Magnetisierung die Summe aller magnetischen
Dipolmomente \(\vec m\).

Die Magnetisierung beschreibt den Zusammenhang zwischen der magnetischen
Flussdichte \(\vec B\) und der magnetischen Feldstärke \(\vec H\)
mithilfe der magnetischen Feldkonstante \(\mu_0\).

\[
\begin{eqnarray}
 \vec B &=& \mu_0 \cdot \left(\vec H + \vec M\right)
\end{eqnarray}
\]

\hypertarget{magnetische-feldstuxe4rke}{%
\subsubsection{magnetische Feldstärke}\label{magnetische-feldstuxe4rke}}

Die magnetische Feldstärke \(\vec H\), auch als magnetische Erregung
bezeichnet, ordnet als vektorielle Größe jedem Raumpunkt eine Stärke und
Richtung des durch die magnetische Spannung erzeugten Magnetfeldes zu.
Die SI-Einheit der magnetischen Feldstärke ist Ampere pro Meter.

Oft benutzt man stattdessen die magnetische Flussdichte \(\vec B\), die
über die magnetische Feldkonstante \(\mu_0\) und die Magnetisierung
\(\vec M\) ermittelt wird.

\[
\begin{eqnarray}
    \vec B &=& \mu_0 \cdot \left(\vec H + \vec M\right)
\end{eqnarray}
\]

\hypertarget{magnetische-feldkonstante}{%
\subsubsection{magnetische
Feldkonstante}\label{magnetische-feldkonstante}}

Die magnetische Feldkonstante \(\mu_0\), auch magnetische Permeabilität
des Vakuums oder Induktionskonstante, ist eine physikalische Konstante,
die eine Rolle bei der Beschreibung von Magnetfeldern spielt. Sie gibt
das Verhältnis der magnetischen Flussdichte zur magnetischen Feldstärke
im Vakuum an. Der Kehrwert der magnetischen Feldkonstanten tritt mit
einem Vorfaktor \(4\pi\) als Proportionalitätskonstante im
magnetostatischen Kraftgesetz auf.

\[
\begin{eqnarray}
    \mu_0 &\approx& 4\pi \cdot \pu{1.0 \cdot 10^{-7} \frac{N}{A^2}} \\
    \mu_0 &\approx& 4\pi \cdot \pu{1.0 \cdot 10^{-7} \frac{Vs}{Am}}
\end{eqnarray}
\]

Die Einheit der magnetischen Feldkonstante ist Newton pro Quadratmeter.

\hypertarget{permeabilituxe4t}{%
\subsubsection{Permeabilität}\label{permeabilituxe4t}}

Die magnetische Permeabilität \(\mu\) beschreibt die Ordnung der
magnetischen Momente und bestimmt die Magnetisierung eines Materials in
einem äußeren Magnetfeld. Es bestimmt daher die Durchlässigkeit von
Materie für magnetische Felder und ist das Verhältnis der magnetischen
Flussdichte \(B\) zur magnetische Feldstärke \(H\).

\[
\begin{eqnarray}
    \mu = \frac{B}{H}
\end{eqnarray}
\]

Wenn keine Magnetisierung vorliegt, dann ist die Permeabilität identisch
mit der magnetischen Feldkonstante \(\mu_0\), die deshalb auch als
\emph{magnetische Permeabilität des Vakuums} bezeichnet wird. Ansonsten
wird sie mit der Permeabilitätszahl \(\mu_r\) ermittelt, die durch die
Suszeptibilität \(\chi\) beschrieben wird.

\[
\begin{eqnarray}
    \mu &=& \mu_0\mu_r \\
    \mu &=& \mu_0(1+\chi)
\end{eqnarray}
\]

\hypertarget{suszeptibilituxe4t}{%
\subsubsection{Suszeptibilität}\label{suszeptibilituxe4t}}

Die Suszeptibilität \(\chi\) ist eine dimensionslose Größe, welche die
Magnetisierbarkeit von Materie beschreibt. Sie beschreibt die Änderung
der Magnetisierung \(\vec M\) durch die Änderung der magnetischen
Feldstärke \(\vec H\).

\[
\begin{eqnarray}
    \chi &=& \frac{\mathrm dM}{\mathrm dH}
\end{eqnarray}
\]

Paramagnetische Stoffe haben eine positive Suszeptibilität, während
diamagnetische Stoffe eine negative Suszeptibilität haben.

Falls die Feldstärke \(\vec H\) und die Magnetisierung \(\vec M\)
parallel sind, kann die magnetische Flussdichte \(\vec B\) mithilfe der
Suszeptibilität \(\chi\) beschrieben werden.

\[
\begin{eqnarray}
    \vec B &=& \mu_0\cdot (1+\chi) \cdot \vec H
\end{eqnarray}
\]

Bei konstanter Magnetisierung \(M\) kann die Suszeptibilität \(\chi\)
als Quotient beschrieben werden, wobei \(H\) der Betrag der Feldstärke
zur Sättigungsmagnetisierung \(M\) ist.

\[
\begin{eqnarray}
    \chi &=& \frac{M}{H}
\end{eqnarray}
\]

\hypertarget{scheinbare-suszeptibilituxe4t}{%
\paragraph{scheinbare
Suszeptibilität}\label{scheinbare-suszeptibilituxe4t}}

Falls beispielsweise ein Luftspalt im Medium vorliegt, müssen die
effektive Feldstärke \(H_E\) im Kern und die Feldstärke \(H_L\) um
Luftspalt betrachtet werden. Daher gibt es eine \emph{scheinbare
Suszeptibilität} \(\chi_\mathrm{Schein}\) und eine \emph{wahre
Suszeptibilität} \(\chi_\mathrm{wahr}\). Nur die wahre Suszeptibilität
\(\chi_\mathrm{wahr}\) wirkt auf das effektiv im Medium wirkende Feld
\(H_E\).

\[
\begin{eqnarray}
    \chi_\mathrm{Schein} &=& \frac{M}{H} \\
    \chi_\mathrm{wahr} &=& \frac{M}{H_E}
\end{eqnarray}
\]

\hypertarget{magnetisches-dipolmoment}{%
\subsubsection{magnetisches
Dipolmoment}\label{magnetisches-dipolmoment}}

Das magnetische Dipolmoment \(\vec \mu\) tritt auf, wenn elektrische
Ladungen sich auf Kreisbahnen bewegen. Es lässt sich über das auf einen
magnetischen Dipol wirkende Drehmoment \(\vec \tau\) in einem Magnetfeld
\(\vec B\) definieren.

Für eine ebene Leiterschleife ist es folgendermaßen beschrieben. Damit
ist das Dipolmoment \(\vec \mu\) parallel zum Drehimpuls \(\vec{L}\).

\[
\begin{eqnarray}
    \vec \tau &=& \vec \mu \times \vec B
\end{eqnarray}
\]

Die Magnetisierung beschreibt die Dichte des magnetischen Momentes.

\hypertarget{ohne-ordnungsphuxe4nomene}{%
\subsection{ohne Ordnungsphänomene}\label{ohne-ordnungsphuxe4nomene}}

\hypertarget{bahnmagnetismus-und-spinmagnetismus}{%
\subsubsection{Bahnmagnetismus und
Spinmagnetismus}\label{bahnmagnetismus-und-spinmagnetismus}}

Besitzt ein Teilchen sowohl ein Bahndrehimpuls \(\vec L\) als auch einen
Spin \(\vec S\), so lässt sich das gesamte magnetische Moment
\(\vec{m}\) dieses Teilchens ausdrücken als:

\[
\begin{eqnarray}
    \vec m &=& \vec m_l + \vec m_s
\end{eqnarray}
\]

Alternativ sind das magnetische Moment \(\vec{m}\) und der
Gesamtdrehimpuls \(\vec{J}\) über das gyromagnetische Verhältnis
\(\gamma\) miteinander verknüpft.

\hypertarget{bahnmagnetismus}{%
\paragraph{Bahnmagnetismus}\label{bahnmagnetismus}}

Bahnmagnetismus beschreibt das magnetische Moment \(\vec m_l\) eines
Teilchens aufgrund seiner Bahnbewegung durch den Bahndrehimpuls. Analog
gibt es Spinmagnetismus.

\hypertarget{spinmagnetismus}{%
\paragraph{Spinmagnetismus}\label{spinmagnetismus}}

Spinmagnetismus beschreibt analog zum Bahnmagnetismus das magnetische
Moment \(\vec{m_{s}}\) eines Teilchens aufgrund seines Spins.

\hypertarget{gyromagnetisches-verhuxe4ltnis}{%
\subsubsection{gyromagnetisches
Verhältnis}\label{gyromagnetisches-verhuxe4ltnis}}

Das magnetische Moment \(\vec{m}\) und der Gesamtdrehimpuls \(\vec{J}\)
sind über das gyromagnetische Verhältnis \(\gamma\) miteinander
verknüpft. Dabei wird \(\gamma\) durch das Planck'sche Wirkungsquantum
\(\hbar\), das Bohr'sche Magneton \(\mu_{B}\) und den Lande-Faktor \(g\)
bestimmt.

\[
\begin{eqnarray}
    \gamma &=& \frac{g\mu_B}{\hbar} \\
    \vec{m} &=& \gamma \vec{J}
\end{eqnarray}
\]

Besitzt ein Teilchen sowohl ein Bahndrehimpuls \(\vec L\) als auch einen
Spin \(\vec S\), so lässt sich das gesamte magnetische Moment
\(\vec{m}\) durch Bahnmagnetismus und Spinmagnetismus darstellen.

\[
\begin{eqnarray}
    \vec m &=& \vec m_l + \vec m_s
\end{eqnarray}
\]

\hypertarget{lande-faktor}{%
\subsubsection{Lande-Faktor}\label{lande-faktor}}

Der Lande-Faktor \(g\) bestimmt das gyromagnetische Verhältnis
\(\gamma\).

Für ein Elektron ist der Lande-Faktor \(g\) durch den Gesamtdrehimpuls
\(J\), den Spin \(S\) und den Bahndrehimpuls \(L\) bestimmt. Im Falle
reinen Spinmagnetismus gilt \(g_{s} \approx 2\), ebenso bei reinem
Bahnmagnetismus.

\[
\begin{eqnarray}
    g &=& 1 + \frac{J(J+1) + S(S+1) + L(L+1)}{2J(J+1)}
\end{eqnarray}
\]

Die einzelnen Terme, z.B. \(J(J+1)\), sind proportional zu dem
Erwartungswert der quadrierten Drehimpulsoperatoren, in diesem Beispiel
\(\hat J^2\).

\hypertarget{bohrsches-magneton}{%
\subsubsection{Bohr'sches Magneton}\label{bohrsches-magneton}}

Das Bohr'sche Magneton \(\mu_B\) beschreibt das magnetische Moment, das
ein Elektron durch seine Rotation um den Atomkern erzeugt. Seine Einheit
ist Energie pro Tesla.

\[
\begin{eqnarray}
    \mu_B &=& \frac{e\hbar}{2m_e} \\
    \mu_B &\approx& \pu{5.788 \cdot 10^{-5} \frac{eV}{T}} \\
    \mu_B &\approx& \pu{9.274 \cdot 10^{-24} \frac{J}{T}}
\end{eqnarray}
\]

\hypertarget{diamagnetismus}{%
\subsubsection{Diamagnetismus}\label{diamagnetismus}}

Ein diamagnetischer Festkörper besitzt keine inneren magnetischen
Momente. Durch ein äußeres Magnetfeld werden aber magnetische Momente im
Festkörper induziert. Diese sind aufgrund der Lenz'schen Regel dem
induzierenden Magnetfeld entgegengesetzt, weshalb die magnetische
Suszeptibilität von diamagnetischen Festkörpern \(\mu_\mathrm{dia}\)
negativ ist.

Für Isolatoren ist \(\mu_\mathrm{dia}\) außerdem von der Temperatur des
Isolators unabhängig.

Perfekter Diamagnetismus ist bei Supraleitern zu finden. Diese weisen
eine magnetische Suszeptibilität von \(\chi_\mathrm{supra} = -1\) auf.

Deshalb wird sich ein beweglicher Diamagnet in einem inhomogenen
Magnetfeld aus diesem herausbewegen, was einen Unterschied zu
Paramagneten darstellt.

\hypertarget{paramagnetismus}{%
\subsubsection{Paramagnetismus}\label{paramagnetismus}}

In einem paramagnetischen Festkörper liegen innere magnetische
Dipolmomente vor, welche z.B. durch Spin und Bahndrehimpuls der
Elektronen herrühren. Diese wechselwirken allerdings nicht miteinander,
wodurch sie nur in einem äußeren Magnetfeld in Richtung des Feldes
ausgerichtet werden. Die magnetische Suszeptibilität
\(\chi_\mathrm{para}\) eines paramagnetischen Festkörpers~ist daher
positiv.

\hypertarget{langevin-paramagnetismus}{%
\paragraph{Langevin-Paramagnetismus}\label{langevin-paramagnetismus}}

Der Langevin-Paramagnetismus liefert Beschreibungen für paramagnetische
Isolatoren. Die magnetische Suszeptibilität \(\chi_\mathrm{langevin}\)
dieser ist durch das Curie-Gesetz durch die Curie-Konstante \(C\) und
die Temperatur \(T\) beschrieben.

\[
\begin{eqnarray}
    \chi_\mathrm{langevin} &=& \frac{C}{T}
\end{eqnarray}
\]

\hypertarget{pauli-paramagnetismus}{%
\paragraph{Pauli-Paramagnetismus}\label{pauli-paramagnetismus}}

Der Pauli-Paramagnetismus beschreibt die Eigenschaften von
paramagnetischen Metallen. Die magnetische Suszeptibilität
\(\chi_\mathrm{pauli}\) eines solchen Metalle ist konstant.

\[
\begin{eqnarray}
    \chi_\mathrm{pauli} &=& \mathrm{const}
\end{eqnarray}
\]

Hier tritt keine Temperaturabhängigkeit mehr auf, da aufgrund der
Fermi-Statistik nur grob \(\frac{T}{T_F}\) Elektronen in einem
Energieintervall um die Fermi-Energie ihre Energie ändern können, wobei
\(T_{f}\) die Fermi-Temperatur ist. Dadurch stellt sich ein
temperaturunabhängiger Beitrag
\(\frac{C}{T} \frac{T}{T_F} = \frac{C}{T_F}\) ein.

Der Pauli-Paramagnetismus ist deutlich schwächer als der
Langevin-Paramagnetismus.

\[
\begin{eqnarray}
    \frac{\chi_\mathrm{pauli}}{\chi_\mathrm{langevin}}
        &\propto& \frac{T}{T_F} \ll 1
\end{eqnarray}
\]

\hypertarget{mit-ordnungsphuxe4nomenen}{%
\subsection{mit Ordnungsphänomenen}\label{mit-ordnungsphuxe4nomenen}}

\hypertarget{magnetische-ordnung}{%
\subsubsection{magnetische Ordnung}\label{magnetische-ordnung}}

Die magnetische Ordnung wird durch die Austauschwechselwirkung
verursacht.

Die Austauschwechelwirkung bestimmt die magnetischen Eigenschaften von
Festkörpern. Sie ist eine Kombination der Coulomb-Wechselwirkung und der
Paulirepulsion.

Wenn sich die Orbitale zweier Atome überlagern, bestimmt das
Pauli-Prinzip, ob sich Elektronen in diesem Bereich aufhalten dürfen:
Haben die Elektronen den gleichen Spin, dann dürfen sie es nicht. Daher
kann es energetisch sinnvoll sein, dass magnetische Dipolmomente
miteinander wechselwirken.

Unterhalb der Curie-Temperatur ist die Austauschwechelwirkung stark
genug, um die magnetischen Momente miteinander wechelwirken zu lassen,
dann handelt es sich um einen Ferromagneten. Oberhalb der
Curie-Temperatur überwiegen thermische Bewegungen, die diese
Wechselwirkung stören, dann handelt es sich um einen Paramagneten.

\hypertarget{magnetische-anisotropie}{%
\subsubsection{magnetische Anisotropie}\label{magnetische-anisotropie}}

Anisotropie ist die Eigenschaft eines Material, die von der Ausrichtung
der magnetischen Drehmomenten abhängt. Dabei bezeichnet man die
energetisch günstigere Ausrichtung als \emph{Achse leichter
Magnetisierung}. Die Kristallstruktur bestimmt über die Art der
Anisotropie. Man unterscheidet zwischen Formanisotropie,
Spannungsanisotropie und Kristallanisotropie.

Im Allgemeinen ist Anisotropie die Richtungsabhängigkeit einer
Eigenschaft oder eines Vorgangs.

\hypertarget{ferromagnetismus}{%
\subsubsection{Ferromagnetismus}\label{ferromagnetismus}}

Bei Ferromagnetismus richten sich die magnetische Momente parallel
zueinander aus. Im \emph{ferrimagnetischen} Material dagegen sind die
Momente abwechselnd antiparallel zueinander ausgerichtet, dennoch heben
sich die Beträge sich nicht auf wie in einem
\emph{antiferromagnetischen} Material.

\hypertarget{domuxe4nen-weiuxdfsche-bezirke}{%
\subsubsection{Domänen / Weiß'sche
Bezirke}\label{domuxe4nen-weiuxdfsche-bezirke}}

Weiß'sche Bezirke sind Bereiche mit gleichartige
Magnetisierungsausrichrung. Der Grenzbereich zwischen zwei Weißschen
Bezirke nennt man \emph{Bloch-Wand}. Auch wenn kein äußeres Magnetfeld
angelegt ist, existieren vereinzelte Bereiche mit parallelen Spins, die
man \emph{Domänen} nennt.

Legt man nun ein Magnetfeld an, verschmelzen kleine Domänen zu größeren
und die Magnetisierung der Stoffes ist messbar. Bei kleiner Feldstärke
findet reversible \emph{Wandverschiebungen} statt.

Schaltet man die Feldstärke hoch finden sogenannte
\emph{Barkhausen-Sprünge} statt. Hierbei ändert sich die Ausrichtung
alle magnetischen Momente ganzer Weiß'scher Bezirke schlagartig, so dass
es zu einer deutlichen Änderung in der Magnetisierungskurve kommt. Dies
geschieht, wenn Defekte in Kristallen zunächst nicht von der
Verschiebung der Bloch-Wände betroffen sind. Sind sie fast umringt, so
schließt sich die Domäne um den Defekt, wodurch die Magnetisierung
sprunghaft ansteigt.

Kurz vor der Sättigung finden \emph{Rotationsprozesse} statt, wo dann
alle magnetische Momente in Richtung des äußeren Feldes zeigen.

\hypertarget{hysteresekurve}{%
\subsubsection{Hysteresekurve}\label{hysteresekurve}}

Die Hysteresekurve beschreibt das Verhalten eines Materials im äußeren
Magnetfeld.

Die Kurve startet im Ursprung und steigt (durch
\emph{Wandverschiebungen}) leicht an, dreht man das Feld auf wird die
Kurve wegen der \emph{Barkhausen-Sprünge} steiler. Bei größer werdenden
Feldstärken verläuft sie durch die Rotation wieder flacher zu. Dann
findet die \emph{Sättigung} statt, wo die maximale Magnetisierung
erreicht ist. Diese Kurve bezeichnet man als \emph{Neukurve}.

Entfernt man das Magnetfeld bzw. schaltet man es ab, sinkt die
Magnetisierung nicht automatisch auf null, sondern eine
Restmagnetisierung bleibt übrig, sogenanntes \emph{Remanenz}.

Sollt auch diese verschwinden, muss man ein negatives Feld anlegen und
die \emph{Koerzitivfeldstärke} erreichen, der Stoff ist dann vollständig
entmagnetisiert. Wird die Feldstärke weiter erhöht, magnetisiert der
Stoff in die entgegengesetzte Richtung, bis die Sättigung wieder
auftritt.

Die \emph{Kommutierungskurve} ist die Verbindungskurve der
Hystereseschleifen-Umkehrpunkte. Die Fläche, die die Hysteresekurve
umschließt, entspricht dem Energiegehalt, das erbracht werden muss,
messbar als Wärme.

Anhand der Hysterese kann man erkennen ob eine Probe weichmagnetisch
oder hartmagnetisch ist. \emph{Weichmagnetische} Materialien haben eine
kleine Koerzitivfeldstärke und eine hohe Sättigungsmagnetisierung, sie
sind also leicht zu magnetisieren. Man verwendet diese oft für
Transformatoren und Sensoren.

\emph{Hartmagnetische} Materialien dagegen haben eine große
Koerzitivfeldstärke und einen niedrigen Sättigungspunkt. Sie sind schwer
zu magnetisieren, daher baut man daraus oft Dauermagnete.

\hypertarget{temperaturabhuxe4ngigkeit}{%
\subsubsection{Temperaturabhängigkeit}\label{temperaturabhuxe4ngigkeit}}

Magnetische Eigenschaften hängen von der Temperatur ab. Steigt diese,
dann nimmt die Permeabilität ab, also die Ordnung der magnetische
Momente. Durch Erhöhung der Temperatur fügt man dem System Energie zu
und die Austauschwechselwirkung wird dadurch schwächer, bis sie
irgendwann komplett überwunden wird. Dieser Punkt nennt man
\emph{Curie-Temperatur}, nur unter diese ist ein ferromagnetischer Stoff
einsetzbar. Ab der Curie-Temperatur zeigt der Stoff paramagnetische
Verhalten.

\hypertarget{phasenuxfcberguxe4nge}{%
\subsubsection{Phasenübergänge}\label{phasenuxfcberguxe4nge}}

\emph{Ordnungsparameter} beschreibt den Zustand eines System beim
\emph{Phasenübergang}.

Bei Ferromagneten ist der Parameter die Magnetisierung. Beträgt der
Parameter Null, so ist das System völlig ungeordnet. Verläuft ein
Phasenübergang sprunghaft (z.B. vom Wasser zu Eis), klassifiziert man
ihn als \emph{Übergang 1. Ordnung}. Ist der Verlauf kontinuierlich (z.B.
von ferromagnetisch zu paramagnetisch) spricht man von einem
\emph{Übergang 2. Ordnung}.

Hierbei sind sprunghaft und kontinuierlich wie folgt definiert. Die 1.
partielle Ableitung der Enthalpie \(G(T,p)\) nach der Temperatur \(T\)
ist unstetig bzw. stetig.

\emph{Latente Wärme} ist die Wärme, die dazu führt, dass ein Stoff
seinen Aggregatzustand ändert, sie führt deshalb nicht zu einer
Temperaturerhöhung.

\hypertarget{entmagnetisierung}{%
\subsection{Entmagnetisierung}\label{entmagnetisierung}}

Laufen die Feldlinien eines äußeren Magnetfeldes durch die Flächen eines
Kristalls, so induzieren sie magnetische Dipolmomente in diesem. Diesen
kann man einen magnetischen Nordpol und einen magnetischen Südpol
zuweisen.

Nach der Lenz'schen Regel wirkt das auf diese Weise induzierte
Magnetfeld dem äußeren Feld entgegen. Dadurch wird das äußere Feld
abgeschwächt, daher nennt man das induzierte Feld auch
\emph{Entmagnetisierungsfeld}.

Soll das Entmagnetisierungsfeld das innere Magnetfeld komplett
auslöschen, ist es proportional zur Magnetisierung. Den
Proportionalitätsfaktor nennt man dann Entmagnetisierungsfaktor.

\hypertarget{entmagnetisierungsfaktor}{%
\subsubsection{Entmagnetisierungsfaktor}\label{entmagnetisierungsfaktor}}

Um einen Stoff mit der Magnetisierung \(M\) zu entmagnetisieren, muss
ein Entmagnetisierungsfeld \(H_\mathrm{ent}\) angelegt werden. Der
Entmagnetisierungsfaktor \(N\) ist der Proportionalitätsfaktor, der den
Zusammenhang zwischen dem Entmagnetisierungsfeldes \(H_\mathrm{ent}\)
und der Magnetisierung \(M\) eines Materials beschreibt.

\[
\begin{eqnarray}
    N &\equiv& \frac{H_\mathrm{ent}}{M}
\end{eqnarray}
\]

Um die Entmagnetisierung zu erreichen, ohne die interne Magnetisierung
\(M\) zu verändern, muss das magnetische Feld \(H_E\) aus dem Medium in
die Luft verdrängt werden. Dazu kann ein Luftspalt im Medium erzeugt
werden. Das Magnetfeld \(H_L\) im Luftspalt wird um den Betrag erhöht,
um den das Feld \(H_E\) im Medium verringert wird.

Dies lässt sich mit einem Ringkern besonders gut realisieren. Für einen
Ringkern mit einer mittleren Länge \(l\) und einem Luftspalt der Länge
\(l_L\ll l\) kann der Entmagnetisierungsfaktor \(N\) folgendermaßen
bestimmt werden.

\[
\begin{eqnarray}
    N &=& \frac{l_L}{l}
\end{eqnarray}
\]

\hypertarget{gescherte-hystereseschleife}{%
\subsubsection{gescherte
Hystereseschleife}\label{gescherte-hystereseschleife}}

Es werde ein Ringkern mit einem Luftspalt durch ein äußeres Magnetfeld
\(H\) entmagnetisiert, ohne die innere Magnetisierung \(M\) zu
verändern. Dazu muss das Magnetfeld \(H\) um ein Entmagnetisierungsfeld
\(H_\mathrm{ent}\) erhöht werden, um das innere Magnetfeld \(H_E\) des
Kerns zu negieren. Die benötigte Stärke von \(H_\mathrm{ent}\) hängt von
der Breite des Luftspalts ab, was aus dem Entmagnetisierungsfaktor \(N\)
hervorgeht.

In einem Ringkern ohne Luftspalt entspricht die Stärke des äußeren
Magnetfeldes \(H\) der des inneren Magnetfeldes \(H_E\). Mit einem
Luftspalt steigt \(H\) an, somit wird die Hystereseschleife nach außen
geschert. Daher lässt sich das Entmagnetisierungsfeld \(H_\mathrm{ent}\)
durch die Scherung der Hystereseschleife bestimmen.

\[
\begin{eqnarray}
    H_\mathrm{ent} &=& H - H_E
\end{eqnarray}
\]

\end{document}
