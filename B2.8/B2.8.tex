% based on a template made by the university of cologne
% http://www.mi.uni-koeln.de/wp-MIEDV/wp-content/uploads/2016/07/LaTeX-Vorlage.zip - 2023-11-02
\documentclass[12pt,a4paper]{scrartcl}

\addtokomafont{sectioning}{\rmfamily}
\usepackage[ngerman]{babel}% deutsches Sprachpaket wird geladen
\usepackage[T1]{fontenc} % westeuropäische Codierung wird verlangt
\usepackage[utf8]{inputenc}% Umlaute werden erlaubt
\usepackage[usenames]{color} % Erlaubt die Benutzung der namen im Farbpaket und deren Änderung
\usepackage{amsmath} % Erweiterung für den Mathe-Satz
\usepackage{amssymb} % alle Zeichen aus msam und msmb werden dargestellt
\usepackage{graphicx} % Graphiken und Bilder können eingebunden werden
%\usepackage{multirow} % erlaubt in einer Spalte einer Tabelle die Felder in mehreren Zeilen zusammenzufassen
\usepackage{enumerate} % erlaubt Nummerierungen
\usepackage{xurl} % Dient zur Auszeichnung von URLs; setzt die Adresse in Schreibmaschinenschrift.
\usepackage[center]{caption}  % Bildunterschrift wird zentriert
%\usepackage{subfigure} % mehrere Bilder können in einer fugure-Umgebung verwendet werden
%\usepackage{longtable} % Diese Umgebung ist ähnlich definiert wie die tabular-Umgebung, erlaubt jedoch mehrseitige Tabellen.
%\usepackage{paralist} % Modifikation der bereits bestehenden Listenumgebungen
\usepackage{lmodern}% Für die Schrift
\usepackage[hidelinks]{hyperref} % Links und Verweise werden innerhalb von PDF Dokumenten erzeugt
%\usepackage{wrapfig} % Das Paket ermöglicht es von Schrift umflossene Bilder und Tabellen einzufügen.
\usepackage{latexsym} % LaTeX-Symbole werden geladen
\usepackage{tikz} % Erlaubt es mit tikz zu zeichnen
\usepackage{tabularx} % Erlaubt Tabellen
\usepackage{algorithm} % Erlaubt Pseudocode
\usepackage{color} % Farbpaket wird geladen
%\usepackage{stmaryrd} % St Mary Road Symbole werden geladen
\usepackage{physics}

\numberwithin{equation}{section} % Nummerierungen der Gleichungen, die durch equation erstellt werden, sind gebunden an die section
\newcommand{\HRule}{\rule{\linewidth}{0.7mm}}
\newcommand{\pu}[1]{\ensuremath{\mathrm{#1}}}

\hyphenation{An-samm-lung }
\hyphenation{auf-wen-dig}
\hyphenation{Brems-me-di-ums}
\hyphenation{be-schreibt}
\hyphenation{Cu-rie-Tem-pe-ra-tur}
\hyphenation{Dop-pel-ätz-me-tho-de}
\hyphenation{ei-ner-seits}
\hyphenation{ein-ge-stellt}
\hyphenation{e-lek-tro-mag-ne-tische}
\hyphenation{E-lek-tro-nik}
\hyphenation{Ener-gie-stragg-ling}
\hyphenation{Ent-mag-neti-sier-ungs-fak-tor}
\hyphenation{Ent-mag-neti-sier-ungs-feld-stär-ke}
\hyphenation{Ent-mag-neti-sier-ungs-ver-hal-ten}
\hyphenation{Erd-be-schleu-ni-gung}
\hyphenation{Er-eig-nis-se}
\hyphenation{er-kenn-bar}
\hyphenation{er-rei-chen}
\hyphenation{ge-ra-der}
\hyphenation{Ge-schoss-e-ner-gi-en}
\hyphenation{Ge-schwin-dig-keit}
\hyphenation{Grenz-be-reich}
\hyphenation{Koh-len-stoff-ket-ten}
\hyphenation{Kom-mu-tie-rungs-kur-ve}
\hyphenation{konn-te}
\hyphenation{kor-ri-gier-te}
\hyphenation{Li-te-ra-tur-wer-te}
\hyphenation{Li-thi-um-fluo-rid-Kris-tal-len}
\hyphenation{Mag-ne-ti-sier-ungs-aus-rich-tung}
\hyphenation{nach-ge-wie-sen}
\hyphenation{nächs-ten}
\hyphenation{nä-he-rungs-wei-se}
\hyphenation{Nor-mal-ver-tei-lung}
\hyphenation{or-ga-ni-schen}
\hyphenation{Pri-mär-elek-tron}
\hyphenation{Scha-len-mo-dells}
\hyphenation{Schnitt-flä-che}
\hyphenation{Se-kun-där-elek-tron-en}
\hyphenation{Se-kun-där-elek-tron-en-ver-viel-fach-er}
\hyphenation{statt-fin-det}
\hyphenation{sys-te-ma-tisch-en}
\hyphenation{Ü-ber-gän-ge}
\hyphenation{Ver-ar-mungs-zo-ne}
\hyphenation{vi-su-a-li-siert}
\hyphenation{Wahr-schein-lich-keit}
\hyphenation{Wie-der-ho-lung}
\hyphenation{zu-sätz-lich}


\hypersetup{
  pdftitle={B 2.8},
  pdfcreator={LaTeX via pandoc}}

\setcounter{secnumdepth}{6}
\setcounter{tocdepth}{6}

\begin{document}
\begin{titlepage}
	\pagestyle{empty}

	\begin{center}

	\textsc{\LARGE Universität zu Köln }\\ [0.4cm]
	\textsc{Mathematisch-Naturwissenschaftliche Fakultät} \\[1.5cm]

	\includegraphics[width=0.45\textwidth]{../media/uni.jpg}\\[1.5cm]  % Uni-Logo wird geladen

	\textsc{\Large Praktikum~B}\\[2mm]
	\textsc{}\\[10mm]
	\HRule \\[0.4cm]

		{	\Huge \bfseries B 2.8}\\[0.4cm]
			{	\huge \bfseries Versetzungen in Lithiumfluorid}\\[0.3cm]
	
	\HRule \\[3cm]

		\textsc{\Large Catherine Tran } \\[3pt]
		\textsc{\Large Carlo Kleefisch } \\[3pt]
		\textsc{\Large Oliver Filla } \\[3pt]
	\end{center}
\end{titlepage}

\newpage
\tableofcontents
\newpage

\hypertarget{einleitung}{%
\section{Einleitung}\label{einleitung}}

Alle Kristalle haben Kristallbaufehler, die ihre Struktur beeinflussen. Punktuelle Kristallbaufehler sind für elektrische Eigenschaften relevant, lineare Kristallbaufehler wie Versetzungen für mechanische Eigenschaften wie Plastizität und die kritische Fließspannung.

In diesem Versuch werden Versetzungen an der Oberfläche von Lithiumfluorid--Kristal-\newline len mithilfe der Doppelätzmethode sichtbar gemacht. Versetzungsbewegungen sind besonders in der Festkörper--Forschung interessant, da diese Wanderung eine plastische Verformung ist, aus der man Eigenschaften wie die Plastizität des Kristalls ableiten kann.

\clearpage
\hypertarget{theoretische-grundlagen}{%
\section{Theoretische Grundlagen}\label{theoretische-grundlagen}}

\hypertarget{lithiumfluorid}{%
	\subsection{Lithiumfluorid}\label{lithiumfluorid}}

Lithiumfluorid ($\mathrm{LiF}$) ist ein ionischer Kristall. Dies bedeutet, der Kristall wird durch elektrostatische Wechselwirkung entgegengesetzt geladener Ionen zusammengehalten. $\mathrm{LiF}$ hat eine sehr hohe Gitterenergie von $1034\mathrm{\,\frac{kJ}{mol}}$, was an den kleinen Ionenradien des Lithiumkations und des Fluoridanions liegt.

Daher ist es gut für diesen Versuch geeignet, weil Versetzungswanderungen bei großem Druck durch eine Presse auftreten, ohne dass die Gitterstruktur großflächig zerstört wird. Stattdessen gleitet nur ein Teil der Struktur.

Die Gitterstruktur von Lithiumfluorid entspricht der von Natriumchlorid mit einer flächenzentriert--kubischen Gitterstruktur und einer zweiatomigen Basis. Die Gitterkonstante von $\mathrm{LiF}$ beträgt $a=0.402\mathrm{\,nm}\pm 1 \cdot 10^{-7} \mathrm{\, \mu m}$. \cite{Uni}

\hypertarget{kuxfcrzester-burgers-vektor-in-mathrmlif}{%
\subsection{\texorpdfstring{kürzester Burgers--Vektor in
$\mathrm{LiF}$}{kürzester Burgers-Vektor in \textbackslash mathrm\{LiF\}}}\label{kuxfcrzester-burgers-vektor-in-mathrmlif}}

Der kürzeste Burgers--Vektor $\vec{b}_\mathrm{min}$ in $\mathrm{LiF}$ ist der $[\frac{1}{2}\,0\,\frac{1}{2}]$ Vektor in der $\expval{1\,1\,0}$ Richtung. Seine Länge bestimmt sich wie folgt durch die Gitterkonstante $a$. \cite{Uni}

\begin{eqnarray}
    \left|\vec{b}_\mathrm{min}\right|
        &=& \frac{1}{2} \cdot \sqrt{a^2 + a^2} \\
        &=& \frac{1}{2} \sqrt{2} a \\
        &=& \frac{a}{\sqrt{2}}
\end{eqnarray}

\noindent
Die Energie einer Versetzung ist proportional zum Quadrat des Burgers--Vektors. Ein Defekt mit einem längeren Burgers--Vektor benötigt daher eine viel größeren Energie, was Versetzungen mit längeren Burgers--Vektoren sehr viel unwahrscheinlicher macht.

Aufgrund der sich abwechselnden Lithium- und Fluoratome in den Richtungen $\expval{1\,0\,0}$ und $\expval{1\,1\,1}$ sind die Burgers--Vektoren in diesen Richtungen doppelt so lang, da sie immer zwischen gleichen Atomen liegen. Der Burgers--Vektor in der $\expval{1\,0\,0}$--Richtung ist davon nicht betroffen und stellt sich dadurch als der kürzeste heraus. Dies ist in Abbildung \ref{abb:LiF Kristallgitter} ersichtlich.

\begin{figure}[ht]
	\centering
	\includegraphics[width=0.5\textwidth]{../media/B2.8/LiF.pdf}
	\caption{LiF Kristallgitter \cite[S. 9]{Newey}}
	\label{abb:LiF Kristallgitter}
\end{figure}

\hypertarget{doppeluxe4tzmethode}{%
	\subsection{Doppelätzmethode}\label{doppeluxe4tzmethode}}

Zur Messung von Versetzungsbewegungen wird in diesem Versuch die Doppelätzmethode angewandt. Sie besteht aus zwei Ätzdurchgängen, um eine erste Versetzung und anschließend eine Versetzungsbewegung verzeichnen zu können.

Versetzungen verschiedener Orientierung oder Typen behindern sich gegenseitig, die reduzierte Mobilität resultiert in einer Ansammlung von Versetzungshaufen. Diese Ansammlung staut sich an Korngrenzen auf.

Alte Versetzungsstellen sind daran zu erkennen, dass sie dichter und definierter sind. Die Bewegung wird durch die Messung des Abstand des letzten Versetzungspunktes nach der ersten Messung und der neuen Versetzungsstellen bestimmt.

Mithilfe der Bewegungsrichtung und Geschwindigkeit können die Gleitebene, das Gleitsystem und somit die Versetzungsart bestimmt werden.

\hypertarget{winkel-zwischen-den-kristalliten-einer-kleinwinkelkorngrenze}{%
\subsection{Winkel zwischen den Kristalliten einer
Kleinwinkelkorngrenze}\label{winkel-zwischen-den-kristalliten-einer-kleinwinkelkorngrenze}}

Für den Winkel $\theta$ zwischen den Kristalliten, die in einer
Korngrenze aufeinander treffen, lässt sich geometrisch folgender
Zusammenhang zum Abstand $d$ zweier Ätzgrübchen und dem Betrag $b$ des Burgers--Vektors finden.
\begin{eqnarray}
    \sin(\theta) &=& \frac{b}{d}
\end{eqnarray}

\noindent
Ist die Korngrenze eine Kleinwinkelkorngrenze ($\theta < 15^\circ$), so lässt sich die Kleinwinkelnäherung für den Sinus nutzen.

\begin{eqnarray}
    \sin(\theta) &\approx& \theta \\
    \Rightarrow \theta &\approx& \frac{b}{d}
\end{eqnarray}

\noindent
Der wahrscheinlichste Burgers--Vektor $\vec{b}$ für $\mathrm{LiF}$ ist wie beschrieben der Vektor $[\frac{1}{2}\,0\,\frac{1}{2}]$ mit einer Länge von $b = \frac{a}{\sqrt{2}}$.

\begin{eqnarray}
    \theta &\approx& \frac{a}{\sqrt{2} \cdot d} \\
        &=& \frac{0.402\,\mathrm{nm}}{\sqrt{2} d}\\
    \theta &\approx& \frac{0.284\,\mathrm{nm}}{d} \label{theta}
\end{eqnarray}

\hypertarget{gleitsysteme-bei-mechanischer-belastung}{%
\subsection{Gleitsysteme bei mechanischer
Belastung}\label{gleitsysteme-bei-mechanischer-belastung}}

Aufgrund der Empfindlichkeit des Materials erzeugt jede mechanische Belastung Gitterfehler in einem Kristall. Bei einem Nadeleindruck werden
Gleitsysteme mit bestimmten Ebenen sowie Richtungen aktiviert, dabei entsteht eine Rosettenform, siehe Abbildung \ref{abb:Rosette}. Die Gleitebenen sind dabei die $\lbrace1\,1\,0\rbrace$--Ebenen, was in Abbildung \ref{abb:Schema Rosette} dargestellt ist. Zusammen mit dem Gleitvektor in $\expval{1\,1\,0}$--Richtungen bilden sie bei Lithiumfluorid insgesamt sechs Gleitsysteme.

Die Arme der Rosette bestehen aus Versetzungen und zwar sowohl Stufen- als auch Schraubenversetzungen. In $\expval{1\,0\,0}$--Richtungen werden Schraubenversetzungen und in $\expval{1\,1\,0}$--Richtungen Stufenversetzungen produziert. Die Arme in $[1\,1\,0]$--Richtung sind, wie in Abbildung \ref{abb:Schema Rosette} ersichtlich, länger und ausgeprägter, weil der Burgersvektor in dieser Richtung am kürzesten ist.

\begin{figure}[ht]
	\begin{minipage}[t]{.3\linewidth}
		\includegraphics[width=\textwidth]{../media/B2.8/Nadeleindruck.pdf}
		\caption{Eine Rosette nach Nadeleindruck \cite[S. 18]{Newey}}
		\label{abb:Rosette}
	\end{minipage}
	\begin{minipage}[t]{.3\linewidth}
		\includegraphics[width=\textwidth]{../media/B2.8/Rosette_Schema.pdf}
		\caption{Schematische Darstellung einer Rosette \cite[S. 21]{Newey}}
		\label{abb:Schema Rosette}
	\end{minipage}
	\begin{minipage}[t]{.3\linewidth}
		\includegraphics[width=\textwidth]{../media/B2.8/Gleitsysteme_Druck.pdf}
		\caption{Erwartetes Abgleiten unter Druck \cite[S. 2]{Newey}}
		\label{abb:Gleitebene}
	\end{minipage}

	\begin{minipage}[t]{\linewidth}
		\centering
		\includegraphics[width=0.9\linewidth]{../media/B2.8/Gleitsysteme.pdf}
		\caption{Gleitebenen in LiF \cite[S. 9]{Newey} }
		\label{abb:Gleitebenen LiF}
	\end{minipage}
\end{figure}

Wird die Probe axial unter Druck gesetzt, z.B. in der $[0\,0\,1]$--Richtung, gibt es die vier potentiellen Gleitebenen $(0\,\bar{1}\,1)$, $(0\,1\,1)$, $(\bar{1}\,0\,1)$ und $(1\,0\,1)$, entlang derer Versetzungen laufen. Dies ist in Abbildung \ref{abb:Gleitebenen LiF} ersichtlich.

Auf der $(\bar{1}\,1\,0)$--Ebene sowie der $(1\,1\,0)$--Ebene gibt es keine Schubspannung, weil beide Ebenen parallel zur Hauptspannungsachse liegen. Stattdessen ist ein Abgleiten in den $\lbrace1\,1\,0\rbrace$--Ebenen zu erwarten, was in Abbildung \ref{abb:Gleitebene} dargestellt ist. Ist der Druck auf die Probe bekannt, lässt sich die Schubspannung im Gleitsystem bestimmen.

\clearpage
\hypertarget{durchfuxfchrung}{%
\section{Durchführung}\label{durchfuxfchrung}}

\hypertarget{vorbereitung-der-proben}{%
\subsection{Vorbereitung der Proben}\label{vorbereitung-der-proben}}

Eine Probe $\mathrm{LiF}$--Kristall mit äußeren Abmessungen von etwa $15 \times 3 \times 3 \,\mathrm{mm^3}$ wird zur Verfügung gestellt. Es wurde zuvor von einem größeren Kristall durch Spalten abgetrennt und dann für $48$ Stunden bei $650\,^\circ\mathrm C$ getempert und langsam abgekühlt.

Eine zweite Probe wird zu Beginn des Versuchs von einem größeren Block abgespalten. Diese wird nicht getempert, allerdings chemisch poliert und geätzt. Das Abspalten erfolgt mit einem Beitel, der parallel zu einer Seitenfläche des großen Blocks angesetzt wird.

\hypertarget{polieren-und-uxe4tzen}$ aus Tetrafluoroborsäure $(\mathrm{HBF_4})$, zu $30\,\mathrm{Vol\,\%}$ aus Salpetersäure $(\mathrm{HNO_3})$ und zu $60\,\mathrm{Vol\,\%}$ aus Wasser $(\mathrm{H_2O})$. Das Ätzmittel ist $50\,\mathrm{ppm}$ Eisen(III)--Chlorid $(\mathrm{FeCl_3})$ in destiliertem Wasser.

Beide Proben wurden für $19\mathrm{\,min}$ in das Poliermittel gegeben. Nach $13.5\mathrm{\,min}$ wurden sie umgedreht, sodass alle Seiten poliert wurden. Daraufhin wurden sie mit Ethanol abgespült, jeweils $6.5\mathrm{\,min}$ geätzt und wiederum mit Ethanol abgespült.

\hypertarget{uxe4tzgruxfcbchendichten-kleinwinkelkorngrenzen}{%
\subsection{Ätzgrübchendichten \&
Kleinwinkelkorngrenzen}\label{uxe4tzgruxfcbchendichten-kleinwinkelkorngrenzen}}

Daraufhin wurden die Proben daraufhin untersucht, welche Seite die wenigsten Schäden und Stufen hat. Diese Seite wurde jeweils für die folgenden Messungen verwendet.

Dann wurden je Probe zwei Aufnahmen aufgenommen, auf denen Ätzgrübchen erkennbar sind. Bei der getemperten Probe wurden je eine Aufnahme bei $2000$--facher und bei $1000$--facher Vergrößerung gemacht, bei der ungetemperten Probe je eine Aufnahme bei $1000$--facher und $500$--facher Vergrößerung.

Dann wurden drei Kleinwinkelkorngrenzen auf der getemperten Probe gesucht und bei $2000$--facher Vergrößerung vermessen. Hierzu wurden jeweils $4$ bis $6$ Ätzgrübchen abgezählt.

\hypertarget{nadeldruckrosetten}{%
\subsection{Nadeldruckrosetten}\label{nadeldruckrosetten}}

Auf der getemperten Probe wurden an drei möglichst defektfreien Stellen Eindrücke mit einer Nadel gemacht. Daraufhin wurde die Probe erneut für $6\mathrm{\,min}$ geätzt und danach mit Ethanol abgespült.

Die dabei entstandenen Rosetten wurden fotografiert. Dazu wurde zunächst eine Übersicht bei $50$--facher Vergrößerung aufgenommen und dann Einzelaufnahmen bei $200$-- bzw. $150$--fachen Vergrößerungen aufgenommen. Hierbei fiel direkt auf, dass die Rosette im Zentrum der Probe größere Ausmaße hatte, weswegen eine geringere Vergrößerung als für die anderen beiden Rosetten notwendig war, um die gesamte Rosette abzubilden.

\hypertarget{eingespannte-probe}{%
\subsection{Eingespannte Probe}\label{eingespannte-probe}}

Nun wurde die Probe hochkant so in eine Presse eingespannt, dass die vermessene Seite senkrecht war. Die Spannung wurde durch ein Gesamtgewicht von $1.789\mathrm{\,kg}$ erzeugt, das durch die obere Hälfte der Presse und ein Zusatzgewicht von $841\mathrm{\,g}$ erzeugt wurde.

Nach $2\mathrm{\,min}$ wurde die Probe herausgeholt und erneut für $6\mathrm{\,min}$ geätzt. Nach erneuten Abspülen mit Ethanol wurden die Rosetten erneut fotografiert. Hierbei wurde darauf geachtet, dass sowohl alte Ätzgrübchen als auch durch Wandern neu entstandene Ätzgrübchen auf den Fotos sichtbar waren. Dies lässt sich durch die Größe und Form der Ätzgrübchen unterscheiden.

Zuletzt wurde eingespannte Fläche vermessen. Dazu wurde die Probe senkrecht auf das Mikroskop gestellt, sodass die Orientierung der Probe der Orientierung beim Einspannen entsprach. Mithilfe des Mikroskops wurde die Oberfläche der Probenschmalseite vermessen.

\clearpage
\hypertarget{auswertung}{%
\section{Auswertung}\label{auswertung}}

\hypertarget{uxe4tzgruxfcbchendichte}{%
\subsection{Ätzgrübchendichte}\label{uxe4tzgruxfcbchendichte}}

\hypertarget{berechnung}{%
\subsubsection{Berechnung}\label{berechnung}}

Die Ätzgrübchendichte ist an verschiedene Stellen der jeweiligen Proben sehr unterschiedlich, da durch die Ätzgrübchendichte lokale Unebenheiten dargestellt werden.

Um die insgesamte Ätzgrübchendichte der beiden Proben abzuschätzen, wurden für jede Probe an zwei repräsentativen Stellen Ätzgrübchen aufgenommen, wobei die Ätzgrübchendichte $N$ durch die Anzahl $n$ der Ätzgrübchen und die Fläche $F$ bestimmt ist. Die Ungenauigkeit $\Delta N$ der Ätzgrübchendichte ist nach Gauß'scher Fehlerfortpflanzung zu bestimmen.

\begin{eqnarray}
    N &=& \frac{n}{F} \label{N}\\
    \Delta N &=& \sqrt{
        \left(\frac{\Delta n}{F}\right)^2
        + \left(\frac{n}{F^2} \cdot \Delta F\right)^2} \label{DeltaN}
\end{eqnarray}

\noindent
Der Fehler der Anzahl der Ätzgrübchendichte $\Delta n$ hängt neben menschlicher Ablesefähigkeit auch von der Schärfe des jeweiligen Bildes ab. Wir schätzen ihn dennoch für alle Bilder auf $10\,\%$, da in unserem Fall die Bilder mit höherer Unschärfe auch die kleinere Anzahl an Ätzgrübchen haben. Der Fehler der Fläche $\Delta F$ ist durch die Genauigkeit der Streckenangabe des Mikroskops gegeben und beträgt $\Delta F=(0.005 \mathrm{\, \mu m})^2$.

Um die durchschnittliche Ätzgrübchendichte der Proben zu bestimmen, wird der Mittelwert der Dichten an den zwei Stellen ermittelt, die Ungenauigkeit ist durch die Standardabweichung gegeben.

\begin{eqnarray}
    \bar{N} &=& \frac{N_1 + N_2}{2} \label{Nbar}\\
    \Delta \bar N &=& \left| \frac{N_1 - N_2}{2} \right| \label{DeltaNbar}
\end{eqnarray}

\hypertarget{nicht-getemperte-probe}{%
\subsubsection{Nicht--getemperte Probe}\label{nicht-getemperte-probe}}

Wie in Abbildung \ref{abb:dichte_nt_1} zu sehen, sind in den inneren sechs blauen Quadraten insgesamt $5\pm1$ Ätzgrübchen zu finden, die Quadrate haben jeweils eine Seitenlänge von $200.00 \mathrm{\,\mu m}$.

An der zweiten Stelle der nicht--getemperten Probe (siehe Abbildung \ref{abb:dichte_nt_2}) sind in den inneren sechs blauen Quadraten $16\pm1$ Ätzgrübchen zu finden, die Quadrate hier je eine Seitenlänge von $100.00 \mathrm{\, \mu m}$.

\begin{figure}[ht]
	\begin{minipage}[t]{.5\linewidth}
		\includegraphics[width=\textwidth]{../media/B2.8/Dichte1_not_tempered.pdf}
		\caption{Ätzgrübchendichte nicht--getemperte Probe, Aufnahme $1$}
		\label{abb:dichte_nt_1}
	\end{minipage}
	\begin{minipage}[t]{.5\linewidth}
		\includegraphics[width=\textwidth]{../media/B2.8/Dichte2_not_tempered.pdf}
		\caption{Ätzgrübchendichte nicht--getemperte Probe, Aufnahme $2$}
		\label{abb:dichte_nt_2}
	\end{minipage}
\end{figure}

Mithilfe der Gleichungen $\eqref{N}$ bis $\eqref{DeltaNbar}$ folgt die mittlere Ätzgrübchendichte $\bar N_\mathrm{nt}$ der nicht--getemperten Probe.

\begin{eqnarray}
    N_\mathrm{nt,1} &=& (2.1 \pm 0.4) \cdot 10^3 \mathrm{\, cm^{-2}} \\
    N_\mathrm{nt,2} &=& (2.7 \pm 0.3) \cdot 10^{4} \mathrm{\, cm^{-2}} \\
    \bar N_\mathrm{nt}
        &=& (1.4 \pm 1.2) \cdot 10^4 \mathrm{\, cm^{-2}}
        \quad(\pm 85.52\,\%)
\end{eqnarray}

\hypertarget{getemperte-probe}{%
\subsubsection{getemperte Probe}\label{getemperte-probe}}

In Abbildung \ref{abb:dichte_t_1} sind insgesamt $50$ Ätzgrübchen in den inneren acht blauen Quadraten zu entdecken. Jedes dieser Quadrate hat eine Seitenlänge von $150.00 \mathrm{\, \mu m}$.

Die zweite Stelle der getemperten Probe ist in Abbildung \ref{abb:dichte_t_2} zu sehen. Dort sind $81\pm8$ Ätzgrübchen in den inneren zwölf blauen Quadraten zu finden. Jedes der Quadrate hat eine Seitenlänge von $70.00 \mathrm{\, \mu m}$.

\begin{figure}[ht]
	\begin{minipage}[t]{.5\linewidth}
		\includegraphics[width=\textwidth]{../media/B2.8/Dichte1_tempered.pdf}
		\caption{Ätzgrübchendichte getemperte Probe, Aufnahme $1$}
		\label{abb:dichte_t_1}
	\end{minipage}
	\begin{minipage}[t]{.5\linewidth}
		\includegraphics[width=\textwidth]{../media/B2.8/Dichte2_tempered.pdf}
		\caption{Ätzgrübchendichte getemperte Probe, Aufnahme $2$}
		\label{abb:dichte_t_2}
	\end{minipage}
\end{figure}

Wieder folgt aus den Gleichungen $\eqref{N}$ bis $\eqref{DeltaNbar}$ die mittlere Ätzgrübchendichte $\bar N_\mathrm{t}$ der getemperten Probe.

\begin{eqnarray}
    N_\mathrm{t,1} &=& (2.8 \pm 0.3) \cdot 10^4 \mathrm{\, cm^{-2}} \\
    N_\mathrm{t,2} &=& (1.4 \pm 0.1) \cdot 10^5 \mathrm{\, cm^{-2}}
    \\
    \bar N_\mathrm{t}
        &=& (8.3 \pm 5.5) \cdot 10^4 \mathrm{\, cm^{-2}}
        \quad(\pm 66.44\,\%)
\end{eqnarray}

\hypertarget{kleinwinkelkorngrenze}{%
\subsection{Kleinwinkelkorngrenze}\label{kleinwinkelkorngrenze}}

Es wurden drei Bilder von Kleinwinkelkorngrenzen aufgenommen. Nun werden die Winkel $\theta$ zwischen den Kristalliten an diesen Kleinwinkelkorngrenzen bestimmen. Dieser durch Gleichung \eqref{theta} gegeben, wobei $d$ den Abstand zweier Ätzgrübchen beschreibt.

Die Ungenauigkeit der Gitterkonstante $\Delta a=1 \cdot 10^{-7} \mathrm{\, \mu m}$ $[5]$ und die des Mikroskops $\Delta d_m = 5 \cdot 10^{-3} \mathrm{\, \mu m}$ ergeben mittels Gauß'scher Fehlerfortpflanzung die Ungenauigkeit des Winkels $\Delta \theta$. Wird der Abstand $d_n$ von $n$ Grübchen gemessen, so sinkt die Ungenauigkeit $\Delta d$ des Abstandes zweier Grübchen.

\begin{eqnarray}
    \Delta d &=& \frac{\Delta d_m}{n} \\
    \Delta \theta &=& \sqrt{
        \left(\frac{\Delta a}{\sqrt{2} \cdot d} \right)^2
            + \left( \frac{a}{\sqrt{2} \cdot d^2} \cdot \Delta d \right)^2 }
            \label{DeltaTheta}
\end{eqnarray}

\begin{figure}[h]
	\begin{minipage}[t]{.31\linewidth}
		\includegraphics[width=\textwidth]{../media/B2.8/KWK1_tempered.pdf}
		\caption{Messung 1}
		\label{abb:KWK1}
	\end{minipage}
	\begin{minipage}[t]{.31\linewidth}
		\includegraphics[width=\textwidth]{../media/B2.8/KWK2_tempered.pdf}
		\caption{Messung 2}
		\label{abb:KWK2}
	\end{minipage}
	\begin{minipage}[t]{.31\linewidth}
		\includegraphics[width=\textwidth]{../media/B2.8/KWK3_tempered.pdf}
		\caption{Messung 3}
		\label{abb:KWK3}
	\end{minipage}
	\caption*{Messungen von Kleinwinkelkorngrenzen}
\end{figure}

\hypertarget{erste-messung}{%
\subsubsection{erste Messung}\label{erste-messung}}

In Abbildung \ref{abb:KWK1} ist die erste Kleinwinkelkorngrenze zu sehen. In rot ist der Abstand $18.28 \mathrm{\, \mu m}$ für $5$ Ätzgrübchen eingetragen.

Aus $\eqref{theta}$ und $\eqref{DeltaTheta}$ folgt der erste Abstand $d_1$, woraus der Winkel $\theta_1$ bestimmt wird.

\begin{eqnarray}
    d_1 &=& (3.656 \pm 0.002) \mathrm{\, \mu m} \\
    \theta_1 &=& (7.775 \pm 0.005) \cdot 10^{-5} \mathrm{\, rad} \\
        &=& (4.455 \pm 0.003) \cdot 10^{-3\ \ \circ}
\end{eqnarray}

\hypertarget{zweite-messung}{%
\subsubsection{zweite Messung}\label{zweite-messung}}

Die zweite von uns aufgenommene Kleinwinkelkorngrenze ist in Abbildung \ref{abb:KWK2} zu sehen. Hier ist wieder der Abstand von $5$ Ätzgrübchen mit $14.53 \mathrm{\, \mu m}$ eingetragen.

Der Abstand zweier Ätzgrübchen $d_2$ sowie der Winkel $\theta_2$ werden analog bestimmt.

\begin{eqnarray}
    d_2 & = & (2.096\pm 0.002) \cdot 10^{-3} \mathrm{\, \mu m} \\
    \theta_2 &=& (9.782 \pm 0.008) \cdot 10^{-5} \mathrm{\, rad} \\
        &=& (5.605 \pm 0.005) \cdot 10^{-3\ \circ}
\end{eqnarray}

\hypertarget{dritte-messung}{%
\subsubsection{dritte Messung}\label{dritte-messung}}

An der letzten untersuchten Stelle ist der Abstand für $3$ Ätzgrübchen von $16.76 \mathrm{\, \mu m}$ zu sehen, siehe Abb. \ref{abb:KWK3}.

Erneut werden der Abstand zweier Ätzgrübchen $d_2$ sowie der Winkel $\theta_2$ analog bestimmt.

\begin{eqnarray}
    d_3 &=& (5.587 \pm 0.003) \mathrm{\, \mu m} \\
    \theta_3 &=& (5.088 \pm 0.003) \mathrm{\, rad} \\
        &=& (2.915 \pm 0.002) \cdot 10^{-3\ \circ}
\end{eqnarray}

\hypertarget{nadeldruckrosetten-1}{%
\subsection{Nadeldruckrosetten}\label{nadeldruckrosetten-1}}

Die Mikroskopbilder von allen drei Rosetten sind zusammen und einzeln in den Abbildungen \ref{abb:Rosetten all} -- \ref{abb:Rosetten 3} zu sehen.

\begin{figure}[ht]
	\begin{minipage}[t]{.5\linewidth}
		\includegraphics[width=\textwidth]{../media/B2.8/Rosetten_uebersicht.pdf}
		\caption{Nadeldruckrosetten}
		\label{abb:Rosetten all}
	\end{minipage}
	\begin{minipage}[t]{.5\linewidth}
		\includegraphics[width=\textwidth]{../media/B2.8/Rosetten_1.pdf}
		\caption{Rosette $1$: obere Mitte }
		\label{abb:Rosetten 1}
	\end{minipage}
	
	\vspace{8pt}
	\begin{minipage}[t]{.5\linewidth}
		\includegraphics[width=\textwidth]{../media/B2.8/Rosetten_2.pdf}
		\caption{Rosette $2$: Ecke}
		\label{abb:Rosetten 2}
	\end{minipage}
	\begin{minipage}[t]{.5\linewidth}
	\includegraphics[width=\textwidth]{../media/B2.8/Rosetten_3.pdf}
	\caption{Rosette $3$: Zentrum}
	\label{abb:Rosetten 3}
\end{minipage}
\end{figure}

Entsprechend der Theorie kann man erkennen, dass die Ätzgrübchen entlang bestimmten Richtungen verlaufen. Zudem sieht man in Abbildung \ref{abb:Rosetten 3}, dass die Nadel an der dritten Stelle im Zentrum abgerutscht ist.

\hypertarget{versetzungswanderung}{%
\subsection{Versetzungswanderung}\label{versetzungswanderung}}

Nachdem die Probe für zwei Minuten in der Presse stand, werden die $\lbrace1\,1\,0\rbrace$--Ebenen als Gleitebenen aktiviert und man erkennt die Versetzungslinien in verschiedene Richtungen in Abbildung \ref{abb:Druck all}.

Allerdings verlaufen auch einige dieser Versetzungslinien parallel zu unseren Versetzungsarmen der Rosetten und man kann nicht eindeutig darüber aussagen, ob die Ätzgrüchen nun zu Versetzungslinien oder zu Versetzungsarmen gehören. Vergleicht man die Bilder von vorher und nachher kann man dennoch sagen, dass die Versetzung bei einigen Armen weiter gewandert ist. Beispielsweise ist der untere Arm der oberen Diagonalen und der obere Arm der unteren Diagonalen gewandert.

Die Arme in $\expval{1\,1\,0}$--Richtungen sind mehr gewachsen als die in $\expval{1\,0\,0}$--Richtungen, was man deutlich in den Abbildungen sieht, dies lässt sich auch mit dem Burgersvektor vereinbaren.

Die Arme zeigen jeweils zur Mitte hin und von der Mitte weg, denn sonst würden sie sich gegenseitig auslöschen und der Kristall würde sich nicht erkürzen. Nur in den Bereichen, wo die Versetzung wandert, existieren eingeschobene Halbebenen, die das Wandern ermöglichen.

Nach der Theorie müssten die Arme aber andersherum wachsen,\footnote{vergleiche Abschnitt \ref{gleitsysteme-bei-mechanischer-belastung}} also wie in der Abbildung \ref{abb:Rosette}. Es ist jedoch schwierig die, Grübchen richtig zuzuordnen, und möglicherweise müsste man die Probe länger in der Presse lassen, um die Effekte deutlicher sehen zu können.

\begin{figure}[ht]
	\begin{minipage}[t]{\linewidth}
		\centering
		\includegraphics[width=0.7\textwidth]{../media/B2.8/Druck_uebersicht.pdf}
		\caption{Abb. 12}
		\label{abb:Druck all}
	\end{minipage}
	
	\vspace{8pt}
	\begin{minipage}[t]{.5\linewidth}
		\includegraphics[width=\textwidth]{../media/B2.8/Druck1.pdf}
		\caption{Versetzungswanderung $1$}
		\label{abb:Druck 1}
	\end{minipage}
	\begin{minipage}[t]{.5\linewidth}
		\includegraphics[width=\textwidth]{../media/B2.8/Druck2.pdf}
		\caption{Versetzungswanderung $2$ und $3$}
		\label{abb:Druck 2}
	\end{minipage}
\end{figure}

Die alten Ätzgrübchen kann man von den Neuen unterscheiden, indem man auf die Größendifferenz achtet. Die alten Grübchen existieren länger und wurden somit länger geätzt, während die neuen Grübchen später dazugekommen sind. Dies bedeutet, dass man große und kleine Grübchen unterscheiden kann.

Diesen Unterschied erkennt man in den Abbildungen \ref{abb:Druck 1} und \ref{abb:Druck 2}. So kann man den Abstand der Versetzungswanderung messen und die Geschwindigkeit berechnen.

Die Probe stand für $t=2\mathrm{\,min}$ in einer Presse. Die Strecke $s$ der Versetzungswanderung wurde durch das Mikroskop ermittelt. Daraus folgt die Geschwindigkeit $v$ der Versetzungsbewegung.

Der Messfehler für die Zeit durch Start--Stopp--Fehler wird auf $\Delta t=5\mathrm s$ geschätzt. Weil man die Ätzgrübchen nicht eindeutig zuordnen und treffen kann, wird der Fehler auf $0.1\mathrm{\,\mu m}$ geschätzt. Der Fehler $\Delta v$ der Geschwindigkeit wird durch Gauß'sche Fehlerfortpflanzung ermittelt.

\begin{eqnarray}
    v &=& \frac{s}{t} \\
    \Delta v &=&
        \sqrt{
            \left(\frac{\Delta s}{t}\right)^2
            + \left(\frac{-s \Delta t}{t^2}\right)^2
        }
\end{eqnarray}

\noindent
Auf diese Weise werden die Geschwindigkeiten $v_1$, $v_2$ und $v_3$ aus den gemessenen Strecken $s_1=982.12\mathrm{\,\mu m}$, $s_2=90.52\mathrm{\,\mu m}$ und $s_3=242.07\mathrm{\,\mu m}$ ermittelt.

\begin{eqnarray}
    v_1 &=& (8.2 \pm 0.3) \mathrm{\,\frac{\mu m}{s}} \\
    v_2 &=& (0.75 \pm 0.03) \mathrm{\,\frac{\mu m}{s}} \\
    v_3 &=& (2.02 \pm 0.08) \mathrm{\,\frac{\mu m}{s}}
\end{eqnarray}

\noindent
Aufgrund der stark unterschiedlichen Geschwindigkeiten $v_2$ und $v_3$, die den selben Arm der Rosette beschreiben, werden diese gemittelt. Die Gewichte $w_i$ folgen aus den Unsicherheiten $\Delta v_i$. Die resultierende Unsicherheit $\Delta v$ ist entweder durch die Gauß'sche Fehlerfortpflanzung $\Delta v_\mathrm{G}$ oder durch die mittlere Quadratsumme der Residuen $\Delta v_\mathrm{MQR}$ zu bestimmen. Hierbei ist der größere Fehler zu wählen.

\begin{eqnarray}
    w_i &=& \frac{1}{(\Delta v_i)^2} \\
    \bar{v} &=& \frac{\sum_{i=2}^{3} v_i\cdot w_i}{\sum_{i=2}^{3} w_i} \\
    \Delta \bar{v}_\mathrm{G} &=&
        \sqrt{\frac{1}{2-1}\frac{\sum_{i=2}^{3}w_i(v_i-\bar{v})^2}{\sum_{i=2}^{3}w_i}} \\
    \Delta \bar{v}_\mathrm{MQR} & = & \frac{1}{\sqrt{\sum_{i=2}^{3}w_i}} \\
    \Delta \bar{v} &=& \max\{\bar{v}_\mathrm{G}, \bar{v}_\mathrm{MQR}\}
\end{eqnarray}

\noindent
Daraus ergibt sich für die Geschwindigkeit $\bar v$ der Rosettenarme aus Abbildung \ref{abb:Druck 2} wie folgt.

\begin{eqnarray}
    \bar{v} &=& 9.088 \cdot 10^{-7} \mathrm{\,\frac{m}{s}} \\
    \Delta v_\mathrm{G} &=& 4.143 \cdot 10^{-7} \mathrm{\,\frac{m}{s}} \\
    \Delta v_\mathrm{MQR} &=& 8.403 \cdot 10^{-10} \mathrm{\,\frac{m}{s}} \\
    \Rightarrow \bar{v} &=& (0.91 \pm 0.41) \mathrm{\,\frac{\mu m}{s}} \\
    v_1 &=& (8.2 \pm 0.3) \mathrm{\,\frac{\mu m}{s}}
\end{eqnarray}

\hypertarget{schubspannung}{%
\subsection{Schubspannung}\label{schubspannung}}

Zuletzt soll die Schubspannung im Gleitsystem der Versetzungen durch den Druck bestimmt werden.

Dazu wird der Druck $p$ der Presse, welcher auf die Probe wirkte, benötigt. Dieser wird durch die Gewichtskraft $F_G=mg$ beschrieben, wobei $m$ die Masse und $g$ die Erdbeschleunigung sind. Desweiteren ist die Oberfläche $A=9,073,890.26\mathrm{\,\mu m^2}$, auf die diese Kraft wirkte, benötigt. Diese wurde durch das Mikroskop vermessen, siehe Abbildung \ref{abb:Querschnitt}.

\begin{eqnarray}
    p &=& \frac{mg}{A} \label{Druck}
\end{eqnarray}

\begin{figure}[ht]
	\centering
	\includegraphics[width=0.7\textwidth]{../media/B2.8/Querschnittsflaeche_tempered.pdf}
	\caption{Oberfläche der eingespannten Probe}
	\label{abb:Querschnitt}
\end{figure}

\noindent
Der Deckel der Presse hat ein Gewicht von $948\mathrm{\,g}$, zusätzlich wurde ein $841\mathrm{\,g}$ schweres Gewicht darauf gelegt. Diese Werte wurden als fehlerfrei angenommen. Da die Probe nicht gerade in der Presse stand und man möglicherweise die Fläche nicht exakt trifft, wird der Fehler der Fläche als $100\mathrm{\,\mu m^2}$ abgeschätzt. Dadurch kann die Unsicherheit $\Delta p$ durch Gleichung $\eqref{DeltaDruck}$ beschrieben werden.

\begin{eqnarray}
    \Delta p &=& \frac{mg \Delta A}{A^2} \label{DeltaDruck}
\end{eqnarray}

\noindent
Die Schubspannung $\sigma$ setzt sich aus der tangentialen Kraftkomponente und der Schnittfläche der $\lbrace1\,1\,0\rbrace$--Ebenen zusammen, vergleiche Abbildung \ref{abb:Gleitebene}. Der Kristall verschiebt sich in einem Winkel von $\varphi=45\,^\circ=\frac{\pi}{4}\mathrm{\,rad}$ entlang der $\lbrace1\,1\,0\rbrace$--Ebenen. Daher kann die Schubspannung $\sigma$ aus dem Druck $p$ berechnet werden.

\begin{eqnarray}
    \sigma &=& \frac{F_G \cdot \cos(\varphi)}{\frac{A}{\cos(\varphi)}}\\
        &=&\frac{F_G}{A} \cdot \cos^2(\varphi) \\
        &=& \frac{1}{2}\frac{F_G}{A}\\
    \sigma &=& \frac{p}{2} \\
    \Delta \sigma &=& \frac{\Delta p}{2}
\end{eqnarray}

\noindent
Mithilfe der Gleichungen $\eqref{Druck}$ und $\eqref{DeltaDruck}$ lässt sich die Schubspannung $\sigma$ ermitteln.

\begin{eqnarray}
    p &=& (1,934.13 \pm 0.02) \mathrm{\,kPa} \\
    \sigma &=& 967.07 \pm 0.01 \mathrm{\,kPa}
\end{eqnarray}

\clearpage
\hypertarget{fazit}{%
\section{Fazit}\label{fazit}}

Es ist schwierig, die Ergebnisse konkret zu bewerten, denn es gibt keine Literaturwerte zum Vergleich. Daher kann die Bewertung nur qualitativ erfolgen. Dies macht eine Bewertung der Schubspannung extrem schwierig, weshalb diese hier nicht erfolgt.

Alle drei ermittelten Kleinwinkelkorngrenzen auf der getemperten Probe hatten einen Winkel im Bereich von einigen Milligrad. Damit fallen sie alle deutlich in den Bereich einer Kleinwinkelkorngrenze, wodurch die hier getätigte Kleinwinkelannahme gerechtfertigt ist.

\hypertarget{uxe4tzgruxfcbchendichte-1}{%
\subsection{Ätzgrübchendichte}\label{uxe4tzgruxfcbchendichte-1}}

Die großen Fehler in den Mittelwerten der Ätzgrübchendichten folgen daraus, dass nur zwei typische Stellen pro Probe verwenden um die durchschnittliche Ätzgrübchendichte der gesamten Probe zu bestimmen. Die Ergebnisse sind trotzdem gut interpretierbar.

Es wurde erwartet, dass die Ätzgrübchendichte und damit die Versetzungsdichte der getemperten Probe niedriger ist als die der nicht--getemperten Probe, da sich die Versetzungen in der Probe beim Temperungsprozess auslöschen. \cite{Newey}

Diese Messungen stellen aber das Gegenteil dar. Die Ätzgrübchendichte der getemperten Probe ist mit ca. $\bar N_\mathrm{t}=(8.3\pm 5.5) \cdot 10^4 \mathrm{\, cm^{-2}}$ selbst innerhalb der Fehlergrenzen größer als die Ätzgrübchendichte der nicht getemperten Probe $\bar N_\mathrm{nt}=(1.4 \pm 1.2) \cdot 10^4 \mathrm{\, cm^{-2}}$.

Dies kann mehrere Ursachen haben. Die erste Möglichkeit ist, dass unsere ausgesuchten repräsentativen Stellen doch nicht so repräsentativ sind wie wir dachten. Wir hätten z.B. aus Versehen eine oder beide Stellen der getemperten Probe innerhalb einer stark beschädigten Zone wählen können.

Dies führt direkt zur zweiten Möglichkeit. Es kann sein, dass die getemperte Probe nach dem Tempern beschädigt wurde und so neue Versetzungen hinzukamen. Dies kann beispielsweise durch Abrutschen der Probe aus der Pinzette, insbesondere beim Spülen, passiert sein.

In beiden Fällen ist das gemessene Ergebnis nicht undenkbar, wenn auch unerwartet. Der Effekt des Temperns auf die Ätzgrübchendichte der Probe konnte somit zwar nicht beobachtet werden. Dennoch konnte ein Gefühl für die Menge an Versetzungen in $\mathrm{LiF}$ entwickelt werden.

\hypertarget{rosetten}{%
\subsection{Rosetten}\label{rosetten}}

Auf den Horizontalen und Vertikalen sind die Versetzungsarme deutlich weniger ausgeprägt als auf den Diagonalen. Dies liegt dran, dass der kürzeste Burgersvektor in den $\expval{1\,1\,0}$--Richtungen zeigt. In diesen Richtungen treten auch die Stufenversetzungen auf. Die weniger ausgeprägte Arme zeigen in den $\expval{1\,0\,0}$--Richtungen, wo sich Schraubenversetzungen bewegen. Diese Richtungen haben den zweitkürzesten Burgersvektor und haben damit die zweitniedrigste Bindungsenergie.

Die Ergebnisse entsprechen die Theorie und unsere Erwartung. Man sieht, dass die Versetzung von dem Arm in Abbildung \ref{abb:Druck 1} mit Geschwindigkeit $v_1=(8.2 \pm 0.3) \,\frac{\mu m}{s}$ deutlich schneller als die von dem Arm in Abbildung \ref{abb:Druck 2} mit der Geschwindigkeit $\bar{v}=(0.91 \pm 0.41) \,\frac{\mu m}{s}$ gewandert ist.

\clearpage
\hypertarget{literaturverzeichnis}{%
	\section{Literaturverzeichnis}\label{literaturverzeichnis}}
\renewcommand{\section}[2]{} % remove extra title
\begin{thebibliography}{9}
\bibitem{Newey}
	 C. Newey und R. Davidge, ``Dislocations in Lithiumfluoride'', editiert von
	A. Bailey, Online verfügbar unter
	\url{https://ph2.uni-koeln.de/fileadmin/Lehre/PraktikumB/Dislocations_in_Lithium_Fluoride.pdf}, 1965
	\bibitem{Kittel}
	C. Kittel, ``Einführung in die Festkörperphysik'', Oldenbourg Verlag, 2005
\bibitem{Hunklinger}
	S. Hunklinger, ``Festkörperphysik'', Oldenbourg Verlag, 2011
\bibitem{Gross}
	R. Gross und A. Marx, ``Festkörperphysik'', Oldenbourg Verlag, 2012
\bibitem{Uni}
	Universität zu Köln, ``Anleitung zum Versuch 2.8 -- Versetzungen in
	LiF'', Online verfügbar unter
	\url{https://ph2.uni-koeln.de/fileadmin/Lehre/PraktikumB/B28-LiF_tutorial_de.pdf}, Juni 2013
\end{thebibliography}
\end{document}